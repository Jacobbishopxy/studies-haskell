\documentclass[./main.tex]{subfiles}

\begin{document}

\subsection*{加载模块}

Haskell 模块是一系列关联的函数,类型以及 typeclasses 的集合。一个 Haskell 的程序是模块的集合,其中主模块加载若干模块并使用这些模块。

Haskell 的标准库也被分成了若干模块,每个模块所包含的函数与类型服务于某些共同的目的。迄今为止我们所有接触到的函数,类型,typeclasses
都位于\acode{Prelude}模块,也是默认加载的模块。

Haskell 加载模块的语法就是\acode{import <module name>}。现在尝试一下加载\acode{Data.List}这个模块:

\begin{lstlisting}[language=Haskell]
  import Data.List

  numUniques :: (Eq a) => [a] -> Int
  numUniques = length . nub
\end{lstlisting}

当\acode{import Data.List}后,所有在\acode{Data.List}中的函数在公共命名空间中变为可用,也就是说可以在脚本中任意处调用这些函数。
\acode{nub}是一个定义在\acode{Data.List}中的函数,接受一个列表并去除其中的重复元素。将\acode{length}与\acode{num}组合起来
\acode{length . nub}产生的函数等同于\acode{\xs -> length (nub xs)}。

在使用 GHCI 的时候同样也可以加载模块至全局命名空间内,只需:

\begin{lstlisting}[language=Haskell]
  ghci> :m + Data.List
\end{lstlisting}

如果想要加载多个模块,仅需:

\begin{lstlisting}[language=Haskell]
  ghci> :m + Data.List Data.Map Data.Set
\end{lstlisting}

如果加载的脚本(通过\acode{:l <xxx script>})中已经加载过模块了,那么便不再需要通过\acode{:m +}进行加载。

如果只是想要从模块中加载几个函数,那么可以这样做:

\begin{lstlisting}[language=Haskell]
  import Data.List (nub, sort)
\end{lstlisting}

如果想加载模块,却不包括某些函数,可以这样:

\begin{lstlisting}[language=Haskell]
  import Data.List hiding (nub)
\end{lstlisting}

为了避免重名,可以使用\acode{qualified},譬如这样:

\begin{lstlisting}[language=Haskell]
  import qualified Data.Map
\end{lstlisting}

这样的话如果想要调用\acode{Data.Map}的\acode{filter}函数时,就必须要\acode{Data.Map.filter},这样的话\acode{filter}仍然会引用
普通的\acode{filter}。不过每次都要写\acode{Data.Map}就很麻烦,因此可以这么写:

\begin{lstlisting}[language=Haskell]
  import qualified Data.Map as M
\end{lstlisting}

现在再要调用\acode{Data.Map}的\acode{filter}时,仅需\acode{M.filter}。

可以在\href{https://downloads.haskell.org/ghc/latest/docs/libraries/}{这里}找到标准库中有哪些模块。

通过\href{https://hoogle.haskell.org/}{Hoogle}可以查看函数的位置,这是一个非常棒的 Haskell 搜索引擎,可以通过名称,模块名称甚至
是类型签名来进行搜索。

\subsection*{Data.List}

\acode{Data.List}模块显然都是关于列表的,它提供了些很有用的函数用于列表处理。我们已经学习到了一些(例如\acode{map}与\acode{filter})
这是因为\acode{Prelude}已经从\acode{Data.List}中加载了不少函数。无需再通过 qualified import 导入\acode{Data.List},因为它并不会
与任何的\acode{Prelude}名称重名(除开那些已经被\acode{Prelude}从\acode{Data.List}中偷走的)。现在让我们看一下其中一些还没有用到过的
函数:

\acode{intersperse}接受一个元素以及一个列表,将该元素插入至列表中每个元素之间:

\begin{lstlisting}[language=Haskell]
  ghci> intersperse '.' "MONKEY"
  "M.O.N.K.E.Y"
  ghci> intersperse 0 [1..6]
  [1,0,2,0,3,0,4,0,5,0,6]
\end{lstlisting}

\acode{intercalate}接受一个列表以及一个列表的列表,将前者插入至后者之间在打平结果:

\begin{lstlisting}[language=Haskell]
  ghci> intercalate " " ["hey","there","guys"]
  "hey there guys"
  ghci> intercalate [0,0,0] [[1,2,3],[4,5,6],[7,8,9]]
  [1,2,3,0,0,0,4,5,6,0,0,0,7,8,9]
\end{lstlisting}

\acode{transpose}接受一个列表的列表,如果将其视为一个二维的矩阵,那么就是列变为行,行变位列:

\begin{lstlisting}[language=Haskell]
  ghci> transpose [[1,2,3],[4,5,6],[7,8,9]]
  [[1,4,7],[2,5,8],[3,6,9]]
  ghci> transpose ["hey", "there", "guys"]
  ["htg","ehu","yey","rs","e"]
\end{lstlisting}

假设我们有一些多项式 $3x^2 + 5x + 9, 10x^3 + 9$ 以及 $ 8x^3 + 5x^2 + x - 1$,我们希望对它们求和,那么可以使用列表\acode{[0,3,5,9]},
\acode{[10,0,0,9]}以及\acode{[8,5,1,-1]}来代表它们的系数(分别为 $x^3,x^2,x^1,x^0$ 的系数),那么可以这样:

\begin{lstlisting}[language=Haskell]
  ghci> map sum $ transpose [[0,3,5,9],[10,0,0,9],[8,5,1,-1]]
  [18,8,6,17]
\end{lstlisting}

首先转置它们,代表着每个同幂的系数在一个列表中,在将\acode{sum}映射至这个转置后的列表中的每个元素(列表)。

% TODO

\end{document}
