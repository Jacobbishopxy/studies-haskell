\documentclass[./main.tex]{subfiles}

\begin{document}

\subsection*{加载模块}

Haskell 模块是一系列关联的函数,类型以及 typeclasses 的集合。一个 Haskell 的程序是模块的集合,其中主模块加载若干模块并使用这些模块。

Haskell 的标准库也被分成了若干模块,每个模块所包含的函数与类型服务于某些共同的目的。迄今为止我们所有接触到的函数,类型,typeclasses
都位于\acode{Prelude}模块,也是默认加载的模块。

Haskell 加载模块的语法就是\acode{import <module name>}。现在尝试一下加载\acode{Data.List}这个模块:

\begin{lstlisting}[language=Haskell]
  import Data.List

  numUniques :: (Eq a) => [a] -> Int
  numUniques = length . nub
\end{lstlisting}

当\acode{import Data.List}后,所有在\acode{Data.List}中的函数在公共命名空间中变为可用,也就是说可以在脚本中任意处调用这些函数。
\acode{nub}是一个定义在\acode{Data.List}中的函数,接受一个列表并去除其中的重复元素。将\acode{length}与\acode{num}组合起来
\acode{length . nub}产生的函数等同于\acode{\xs -> length (nub xs)}。

在使用 GHCI 的时候同样也可以加载模块至全局命名空间内,只需:

\begin{lstlisting}[language=Haskell]
  ghci> :m + Data.List
\end{lstlisting}

如果想要加载多个模块,仅需:

\begin{lstlisting}[language=Haskell]
  ghci> :m + Data.List Data.Map Data.Set
\end{lstlisting}

如果加载的脚本(通过\acode{:l <xxx script>})中已经加载过模块了,那么便不再需要通过\acode{:m +}进行加载。

如果只是想要从模块中加载几个函数,那么可以这样做:

\begin{lstlisting}[language=Haskell]
  import Data.List (nub, sort)
\end{lstlisting}

如果想加载模块,却不包括某些函数,可以这样:

\begin{lstlisting}[language=Haskell]
  import Data.List hiding (nub)
\end{lstlisting}

为了避免重名,可以使用\acode{qualified},譬如这样:

\begin{lstlisting}[language=Haskell]
  import qualified Data.Map
\end{lstlisting}

这样的话如果想要调用\acode{Data.Map}的\acode{filter}函数时,就必须要\acode{Data.Map.filter},这样的话\acode{filter}仍然会引用
普通的\acode{filter}。不过每次都要写\acode{Data.Map}就很麻烦,因此可以这么写:

\begin{lstlisting}[language=Haskell]
  import qualified Data.Map as M
\end{lstlisting}

现在再要调用\acode{Data.Map}的\acode{filter}时,仅需\acode{M.filter}。

可以在\href{https://downloads.haskell.org/ghc/latest/docs/libraries/}{这里}找到标准库中有哪些模块。

通过\href{https://hoogle.haskell.org/}{Hoogle}可以查看函数的位置,这是一个非常棒的 Haskell 搜索引擎,可以通过名称,模块名称甚至
是类型签名来进行搜索。

\subsection*{Data.List}

\acode{Data.List}模块显然都是关于列表的,它提供了些很有用的函数用于列表处理。我们已经学习到了一些(例如\acode{map}与\acode{filter})
这是因为\acode{Prelude}已经从\acode{Data.List}中加载了不少函数。无需再通过 qualified import 导入\acode{Data.List},因为它并不会
与任何的\acode{Prelude}名称重名(除开那些已经被\acode{Prelude}从\acode{Data.List}中偷走的)。现在让我们看一下其中一些还没有用到过的
函数:

\acodered{intersperse}接受一个元素以及一个列表,将该元素插入至列表中每个元素之间:

\begin{lstlisting}[language=Haskell]
  ghci> intersperse '.' "MONKEY"
  "M.O.N.K.E.Y"
  ghci> intersperse 0 [1..6]
  [1,0,2,0,3,0,4,0,5,0,6]
\end{lstlisting}

\acodered{intercalate}接受一个列表以及一个列表的列表,将前者插入至后者之间在打平结果:

\begin{lstlisting}[language=Haskell]
  ghci> intercalate " " ["hey","there","guys"]
  "hey there guys"
  ghci> intercalate [0,0,0] [[1,2,3],[4,5,6],[7,8,9]]
  [1,2,3,0,0,0,4,5,6,0,0,0,7,8,9]
\end{lstlisting}

\acodered{transpose}接受一个列表的列表,如果将其视为一个二维的矩阵,那么就是列变为行,行变位列:

\begin{lstlisting}[language=Haskell]
  ghci> transpose [[1,2,3],[4,5,6],[7,8,9]]
  [[1,4,7],[2,5,8],[3,6,9]]
  ghci> transpose ["hey", "there", "guys"]
  ["htg","ehu","yey","rs","e"]
\end{lstlisting}

假设我们有一些多项式 $3x^2 + 5x + 9, 10x^3 + 9$ 以及 $ 8x^3 + 5x^2 + x - 1$,我们希望对它们求和,那么可以使用列表\acode{[0,3,5,9]},
\acode{[10,0,0,9]}以及\acode{[8,5,1,-1]}来代表它们的系数(分别为 $x^3,x^2,x^1,x^0$ 的系数),那么可以这样:

\begin{lstlisting}[language=Haskell]
  ghci> map sum $ transpose [[0,3,5,9],[10,0,0,9],[8,5,1,-1]]
  [18,8,6,17]
\end{lstlisting}

上述做法就是先转置整个列表\acode{transpose},这就代表着每个同幂的系数在一个列表中,再将\acode{sum}映射至这个转置后的列表中的每个元素(列表)。

\acodered{foldl'}与\acodered{foldl1'}相对于它们的懒加载的原版是更严格的版本,当使用原版的懒加载 fold 用于一个很大的列表,很有可能就会得到
一个堆栈溢出的错误。罪魁祸首就是因为 fold 的懒加载特性,累加器的值并不会在 folding 时真正的去计算,相反是在需要结果时才去进行计算(这也被称为
一个型实转换程序 thunk)。这发生在每个中间累加器上,且所有的这些 thunk 都会导致堆栈溢出。strict 版本由于不是懒加载的,是真实的计算中间值而不是
一直在栈上堆叠。因此当使用懒加载的 folds 时遇到了堆栈溢出,可以尝试一下用它们的严格版本。

\acodered{concat}打平一个列表的列表:

\begin{lstlisting}[language=Haskell]
  ghci> concat ["foo", "bar", "car"]
  "foobarcar"
  ghci> concat [[3,4,5],[2,3,4],[2,1,1]]
  [3,4,5,2,3,4,2,1,1]
\end{lstlisting}

\acodered{concatMap}首先映射一个函数至列表接着将该列表进行\acode{concat}:

\begin{lstlisting}[language=Haskell]
  ghci> concatMap (replicate 4) [1..3]
  [1,1,1,1,2,2,2,2,3,3,3,3]
\end{lstlisting}

\acodered{and}接受一个布尔值的列表,当所有值皆为\acode{True}时返回\acode{True}:

\begin{lstlisting}[language=Haskell]
  ghci> and $ map (>4) [5,6,7,8]
  True
  ghci> and $ map (==4) [4,4,3,4,4]
  False
\end{lstlisting}

\acodered{or}与\acode{and}类似,只不过是任意元素为\acode{True}时返回\acode{True}:

\begin{lstlisting}[language=Haskell]
  ghci> or $ map (==4) [2,3,4,5,6,7]
  True
  ghci> or $ map (>4) [1,2,3]
  False
\end{lstlisting}

\acodered{any}与\acodered{all}接受一个子句,然后检查列表中的所有元素,通常而言我们会使用这两个函数而不是像上面那样先\acode{map}接着
\acode{and}或是\acode{or}。

\begin{lstlisting}[language=Haskell]
  ghci> any (==4) [2,3,5,6,1,4]
  True
  ghci> all (>4) [6,9,10]
  True
  ghci> all (`elem` ['A'..'Z']) "HEYGUYSwhatsup"
  False
  ghci> any (`elem` ['A'..'Z']) "HEYGUYSwhatsup"
  True
\end{lstlisting}

\acodered{iterate}接受一个函数以及一个起始值,将函数应用在这个起始值得到结果后,再将函数应用至该结果,以此类推,返回一个无限列表。

\begin{lstlisting}[language=Haskell]
  ghci> take 10 $ iterate (*2) 1
  [1,2,4,8,16,32,64,128,256,512]
  ghci> take 3 $ iterate (++ "haha") "haha"
  ["haha","hahahaha","hahahahahaha"]
\end{lstlisting}

\acodered{splitAt}接受一个数值与一个列表,将列表从数值作为的索引出切分为两部分,返回一个包含了切分后两个列表的二元元组:

\begin{lstlisting}[language=Haskell]
  ghci> splitAt 3 "heyman"
  ("hey","man")
  ghci> splitAt 100 "heyman"
  ("heyman","")
  ghci> splitAt (-3) "heyman"
  ("","heyman")
  ghci> let (a,b) = splitAt 3 "foobar" in b ++ a
  "barfoo"
\end{lstlisting}

\acodered{takeWhile}是一个非常有用的小函数,它从一个列表中从头开始获取元素,直到元素不再满足子句的条件,返回之前所有满足元素的列表:

\begin{lstlisting}[language=Haskell]
  ghci> takeWhile (>3) [6,5,4,3,2,1,2,3,4,5,4,3,2,1]
  [6,5,4]
  ghci> takeWhile (/=' ') "This is a sentence"
  "This"
\end{lstlisting}

假设我们想要所有三次方都小于 10,000 的值之和,将\acode{(^3)}映射至\acode{[1..]},然后应用一个过滤函数再进行求和这是做不到的,
因为过滤一个无限长的列表将永远都不会结束。虽然我们知道列表里的元素是递增的,但是 Haskell 并不知道,因此需要这么做:

\begin{lstlisting}[language=Haskell]
  ghci> sum $ takeWhile (<10000) $ map (^3) [1..]
  53361
\end{lstlisting}

将\acode{(^3)}应用至一个无限列表,然后一旦元素超过 10,000 那么列表将被截断,这样才可以求和。

\acodered{dropWhile}也类似,它扔掉所有子句中条件判断为真的元素,一旦子句返回\acode{False},则返回剩余的列表。

\begin{lstlisting}[language=Haskell]
  ghci> dropWhile (/=' ') "This is a sentence"
  " is a sentence"
  ghci> dropWhile (<3) [1,2,2,2,3,4,5,4,3,2,1]
  [3,4,5,4,3,2,1]
\end{lstlisting}

\acodered{span}有点像\acode{takeWhile},不过它返回的是一对列表。第一个列表包含了所有\acode{takeWhile}所返回的元素,第二个列表
则是剩余的部分:

\begin{lstlisting}[language=Haskell]
  ghci> let (fw, rest) = span (/=' ') "This is a sentence" in "First word:" ++ fw ++ ", the rest:" ++ rest
  "First word: This, the rest: is a sentence"
\end{lstlisting}

\acode{break}则是列表中第一个元素令子句条件为真时,切分并返回一对列表。\acode{break p}等同于\acode{span (not . p)}。

\begin{lstlisting}[language=Haskell]
  ghci> break (==4) [1,2,3,4,5,6,7]
  ([1,2,3],[4,5,6,7])
  ghci> span (/=4) [1,2,3,4,5,6,7]
  ([1,2,3],[4,5,6,7])
\end{lstlisting}

使用\acodered{break}时,第一个满足条件的元素将会放在第二个列表的头部。

\acodered{sort}则是对一个列表排序。元素的类型必须属于\acode{Ord} typeclass:

\begin{lstlisting}[language=Haskell]
  ghci> sort [8,5,3,2,1,6,4,2]
  [1,2,2,3,4,5,6,8]
  ghci> sort "This will be sorted soon"
  "    Tbdeehiillnooorssstw"
\end{lstlisting}

\acodered{group}接受一个列表,并将相邻的相同的元素组成子列表:

\begin{lstlisting}[language=Haskell]
  ghci> group [1,1,1,1,2,2,2,2,3,3,2,2,2,5,6,7]
  [[1,1,1,1],[2,2,2,2],[3,3],[2,2,2],[5],[6],[7]]
\end{lstlisting}

如果在 group 一个列表之前进行排序,那么我们就可以知道每个元素在列表中出现了多少次:

\begin{lstlisting}[language=Haskell]
  ghci> map (\l@(x:xs) -> (x,length l)) . group . sort $ [1,1,1,1,2,2,2,2,3,3,2,2,2,5,6,7]
  [(1,4),(2,7),(3,2),(5,1),(6,1),(7,1)]
\end{lstlisting}

\acodered{inits}与\acodered{tails}类似于\acode{init}与\acode{tail},不同之处在于前者们会递归的应用至一个列表直到空:

\begin{lstlisting}[language=Haskell]
  ghci> inits "w00t"
  ["","w","w0","w00","w00t"]
  ghci> tails "w00t"
  ["w00t","00t","0t","t",""]
  ghci> let w = "w00t" in zip (inits w) (tails w)
  [("","w00t"),("w","00t"),("w0","0t"),("w00","t"),("w00t","")]
\end{lstlisting}

尝试一下使用 fold 来实现查找子列表:

\begin{lstlisting}[language=Haskell]
  search :: (Eq a) => [a] -> [a] -> Bool
  -- search needle haystack =
  --   let nlen = length needle
  --    in foldl (\acc x -> if take nlen x == needle then True else acc) False (tails haystack)
  search needle haystack =
    let nlen = length needle
     in foldl (\acc x -> (take nlen x == needle) || acc) False (tails haystack)
\end{lstlisting}

首先调用\acode{tails}来处理需要查找的列表,接着查找每个 tail 直到该 tail 的头是期望找到的。

上述代码的行为实际上就是\acodered{isInfixOf}函数,该函数在一个列表中查找子列表,在发现了该子列表时返回\acode{True}:

\begin{lstlisting}[language=Haskell]
  ghci> "cat" `isInfixOf` "im a cat burglar"
  True
  ghci> "Cat" `isInfixOf` "im a cat burglar"
  False
  ghci> "cats" `isInfixOf` "im a cat burglar"
  False
\end{lstlisting}

\acodered{isPrefixOf}与\acodered{isSuffixOf}则是从前往后进行查找与从后往前进行查找:

\begin{lstlisting}[language=Haskell]
  ghci> "hey" `isPrefixOf` "hey there!"
  True
  ghci> "hey" `isPrefixOf` "oh hey there!"
  False
  ghci> "there!" `isSuffixOf` "oh hey there!"
  True
  ghci> "there!" `isSuffixOf` "oh hey there"
  False
\end{lstlisting}

\acodered{elem}与\acodered{notElem}则是查找元素是否在列表中。

\acodered{partition}接受一个列表以及一个子句,返回一对列表,第一个列表包含了所有符合子句条件的元素,其余的在第二个列表:

\begin{lstlisting}[language=Haskell]
  ghci> partition (`elem` ['A'..'Z']) "BOBsidneyMORGANeddy"
  ("BOBMORGAN","sidneyeddy")
  ghci> partition (>3) [1,3,5,6,3,2,1,0,3,7]
  ([5,6,7],[1,3,3,2,1,0,3])
\end{lstlisting}

理解\acode{partition}有别于\acode{span}与\acode{break}是非常重要的:

\begin{lstlisting}[language=Haskell]
  ghci> span (`elem` ['A'..'Z']) "BOBsidneyMORGANeddy"
  ("BOB","sidneyMORGANeddy")
\end{lstlisting}

因为\acode{span}与\acode{break}两者是一旦元素满足子句条件便停止,而\acode{partition}则是根据子句判断遍历整个列表。

\acodered{find}接受一个子句以及一个列表,返回首个满足子句条件的元素。不过它返回的元素是被包含在\acode{Maybe}值之中。
我们将会在下一节深入讲解代数数据类型,不过现在我们需要知道的是:一个\acode{Maybe}值可以是\acode{Just something}或是
\acode{Nothing}。这就像是一个列表既可以是空列表也可以是带有元素的列表,而\acode{Maybe}即可以是无元素又可以是一个元素。

\begin{lstlisting}[language=Haskell]
  ghci> find (>4) [1,2,3,4,5,6]
  Just 5
  ghci> find (>9) [1,2,3,4,5,6]
  Nothing
  ghci> :t find
  find :: (a -> Bool) -> [a] -> Maybe a
\end{lstlisting}

这里需要注意的是\acode{find}的类型,它返回的是\acode{Maybe a}。

\acode{elemIndex}有点像\acode{elem},不过它返回的不是布尔值,而是一个由\acode{Maybe}包含的索引,如果元素不在列表中,
则返回\acode{Nothing}:

\begin{lstlisting}[language=Haskell]
  ghci> :t elemIndex
  elemIndex :: Eq a => a -> [a] -> Maybe Int
  ghci> 4 `elemIndex` [1,2,3,4,5,6]
  Just 3
  ghci> 10 `elemIndex` [1,2,3,4,5,6]
  Nothing
\end{lstlisting}

\acodered{elemIndices}类似于\acode{elemIndex},不过它返回的是索引的列表,因为是列表,所以即使找不到元素也可以返回一个空列表,
这样就不需要一个\acode{Maybe}类型了:

\begin{lstlisting}[language=Haskell]
  ghci> ' ' `elemIndices` "Where are the spaces?"
  [5,9,13]
\end{lstlisting}

\acodered{finxIndex}像\acode{find},不过返回的是索引,而\acodered{findIndices}返回的是全部匹配元素的索引:

\begin{lstlisting}[language=Haskell]
  ghci> findIndex (==4) [5,3,2,1,6,4]
  Just 5
  ghci> findIndex (==7) [5,3,2,1,6,4]
  Nothing
  ghci> findIndices (`elem` ['A'..'Z']) "Where Are The Caps?"
  [0,6,10,14]
\end{lstlisting}

我们已经尝试过了\acode{zip}与\acode{zipWith},那么对于多个列表就可以使用\acodered{zip3},\acodered{zip4}等等,以及
\acodered{zipWith3},\acodered{zipWith4}等等,这里最高可以是 zip 七个列表。不过有更好的办法可以 zip 无穷多个列表,
只不过我们现有的知识暂时还没有办法到那儿。

\begin{lstlisting}[language=Haskell]
  ghci> zipWith3 (\x y z -> x + y + z) [1,2,3] [4,5,2,2] [2,2,3]
  [7,9,8]
  ghci> zip4 [2,3,3] [2,2,2] [5,5,3] [2,2,2]
  [(2,2,5,2),(3,2,5,2),(3,2,3,2)]
\end{lstlisting}

跟普通的\acode{zip}一样,计算到最短的列表结束为止。

当处理文件或者是其他地方而来的输入,\acodered{lines}则是一个非常有用的函数:

\begin{lstlisting}[language=Haskell]
  ghci> lines "first line\nsecond line\nthird line"
  ["first line","second line","third line"]
\end{lstlisting}

\acode{'\\n'}是 unix 的换行符。

\acodered{unlines}则是与\acode{lines}相反,将字符串列表变回一个由\acode{'\\n'}分隔的大字符串:

\begin{lstlisting}[language=Haskell]
  ghci> unlines ["first line", "second line", "third line"]
  "first line\nsecond line\nthird line\n"
\end{lstlisting}

\acodered{words}与\acodered{unwords}则是分隔与组装一行字符串:

\begin{lstlisting}[language=Haskell]
  ghci> words "hey these are the words in this sentence"
  ["hey","these","are","the","words","in","this","sentence"]
  ghci> words "hey these           are    the words in this\nsentence"
  ["hey","these","are","the","words","in","this","sentence"]
  ghci> unwords ["hey","there","mate"]
  "hey there mate"
\end{lstlisting}

\acodered{nub}我们已经见识过了,移除列表中的重复元素:

\begin{lstlisting}[language=Haskell]
  ghci> nub [1,2,3,4,3,2,1,2,3,4,3,2,1]
  [1,2,3,4]
  ghci> nub "Lots of words and stuff"
  "Lots fwrdanu"
\end{lstlisting}

\acodered{delete}接受一个元素以及一个列表,删除列表中第一个出现的元素。

\begin{lstlisting}[language=Haskell]
  ghci> delete 'h' "hey there ghang!"
  "ey there ghang!"
  ghci> delete 'h' . delete 'h' $ "hey there ghang!"
  "ey tere ghang!"
  ghci> delete 'h' . delete 'h' . delete 'h' $ "hey there ghang!"
  "ey tere gang!"
\end{lstlisting}

\acodered{\\\\}是一个列表比较函数,类似于集合比较,根据右侧的列表中的元素,移除左边列表中匹配的元素。

\begin{lstlisting}[language=Haskell]
  ghci> [1..10] \\ [2,5,9]
  [1,3,4,6,7,8,10]
  ghci> "Im a big baby" \\ "big"
  "Im a  baby"
\end{lstlisting}

处理\acode{[1..10] \\\\ [2,5,9]}类似于\acode{delete 2 . delete 5 . delete 9 \$ [1..10]}。

\acodered{union}类似于集合的并集,它返回的是两个列表的集合,做法是将第二个列表中没有在第一个列表中出现的元素,
添加至第一个列表的尾部。注意第二个列表中重复的元素会被移除。

\begin{lstlisting}[language=Haskell]
  ghci> "hey man" `union` "man what's up"
  "hey manwt'sup"
  ghci> [1..7] `union` [5..10]
  [1,2,3,4,5,6,7,8,9,10]
\end{lstlisting}

\acodered{intersect}类似于集合的交集,它返回两个列表同时存在的元素:

\begin{lstlisting}[language=Haskell]
  ghci> [1..7] `intersect` [5..10]
  [5,6,7]
\end{lstlisting}

\acodered{insert}接受一个元素以及一个可以被排序的列表,将该元素插入到最后一个仍然小于或等于下一个元素的位置,\acode{insert}将会
从列表的头开始,直到找到一个元素大于等于它,接着插入到找到的这个元素之前并结束。

\begin{lstlisting}[language=Haskell]
  ghci> insert 4 [3,5,1,2,8,2]
  [3,4,5,1,2,8,2]
  ghci> insert 4 [1,3,4,4,1]
  [1,3,4,4,4,1]
\end{lstlisting}

如果我们对一个排序后的列表使用\acode{insert},那么返回的列表仍然是排序好的:

\begin{lstlisting}[language=Haskell]
  ghci> insert 4 [1,2,3,5,6,7]
  [1,2,3,4,5,6,7]
  ghci> insert 'g' $ ['a'..'f'] ++ ['h'..'z']
  "abcdefghijklmnopqrstuvwxyz"
  ghci> insert 3 [1,2,4,3,2,1]
  [1,2,3,4,3,2,1]
\end{lstlisting}

因为历史的原因\acode{length},\acode{take},\acode{drop},\acode{splitAt},\acode{!!}以及\acode{replicate}接受的都是\acode{Int}
类型,即使它们可以更加的泛用接受\acode{Integral}或\acode{Num} typeclasses(取决于函数本身),但是修改它们则会影响大量已经存在的代码。
这就是为什么在\acode{Data.List}中引入了\acodered{genericLength},\acodered{genericTake},\acodered{genericDrop},
\acodered{genericSplitAt},\acodered{genericIndex}以及\acodered{genericReplicate}作为泛化的版本。例如\acode{length}的类型签名是
\acode{length :: [a] -> Int},那么如果想要一个列表的均值\acode{let xs = [1..6] in sum xs / length xs},这么做会得到一个类型错误,
因为我们不能对\acode{Int}使用\acode{/}。而\acode{genericLength}的类型签名是\acode{genericLength :: (Num a) => [b] -> a}。
因为一个\acode{Num}可以是一个浮点数,那么\acode{let xs = [1..6] in sum xs / genericLength xs}这样做就没有问题了。

对于\acode{nub},\acode{delete},\acode{union},\acode{intersect}以及\acode{group}而言,它们的泛化版本则是\acodered{nubBy},
\acodered{deleteBy},\acodered{unionBy},\acodered{intersectBy}以及\acodered{groupBy}。它们的不同之处在于前一批使用的函数是
\acode{==}用于比较,而后一批则是使用输入的比较函数进行比较。\acode{group}等同于\acode{groupBy (==)}。

例如,假设我们有一个描述函数美妙值的列表。我们希望基于正负数,将其它们分别分段形成子列表。如果用普通的\acode{group}那就只能将相同的相邻元素
分组在一起,而使用\acode{groupBy}则可以通过\textit{By}函数进行同样类型的判断,当认为是同样类型的即返回\acode{True}:

\begin{lstlisting}[language=Haskell]
  ghci> let values = [-4.3, -2.4, -1.2, 0.4, 2.3, 5.9, 10.5, 29.1, 5.3, -2.4, -14.5, 2.9, 2.3]
  ghci> groupBy (\x y -> (x > 0) == (y > 0)) values
  [[-4.3,-2.4,-1.2],[0.4,2.3,5.9,10.5,29.1,5.3],[-2.4,-14.5],[2.9,2.3]]
\end{lstlisting}

一个更简单的方法是从\acode{Data.Function}中加载\acode{on}函数,其定义:

\begin{lstlisting}[language=Haskell]
  ghci> :t on
  on :: (b -> b -> c) -> (a -> b) -> a -> a -> c
\end{lstlisting}

那么使用\acode{(==) `on` (> 0)}返回的函数类似于\acode{\\x y -> (x > 0) == (y > 0)}。\acodered{on}在\textit{By}函数中运用的很广:

\begin{lstlisting}[language=Haskell]
  ghci> groupBy ((==) `on` (> 0)) values
  [[-4.3,-2.4,-1.2],[0.4,2.3,5.9,10.5,29.1,5.3],[-2.4,-14.5],[2.9,2.3]]
\end{lstlisting}

确实非常有可读性!我们可以大声的念出来:根据元素是否大于零进行相等分组。

\acode{sort},\acode{insert},\acode{maximum}以及\acode{minimum}也同样拥有更泛用的版本:\acodered{sortBy},\acodered{insertBy},
\acodered{maximumBy}以及\acodered{minimumBy}。\acode{sortBy}的类型签名是\acode{sortBy :: (a -> a -> Ordering) -> [a] -> [a]}。
如果还记得之前的\acode{Ordering}类型可以是\acode{LT}\acode{EQ}或\acode{GT}。\acode{sort}等同于\acode{sortBy compare},因为
compare 只接受两个类型为 Ord typeclass,并返回它们的排序关系。

列表可以被比较,一旦可以比较,它们则是根据字典顺序进行比较。那么如果我们有一个列表的列表,并想要不根据内部列表的内容,而是根据长度来进行排序呢?
这就可以使用\acode{sortBy}函数:

\begin{lstlisting}[language=Haskell]
  ghci> let xs = [[5,4,5,4,4],[1,2,3],[3,5,4,3],[],[2],[2,2]]
  ghci> sortBy (compare `on` length) xs
  [[],[2],[2,2],[1,2,3],[3,5,4,3],[5,4,5,4,4]]
\end{lstlisting}

这里的\acode{on}等同于\acode{\\x y -> length x `compare` length y}。也就是说处理\textit{By}函数时,需要等式时可以使用
\acode{(==) `on` something},需要排序时可以\acode{compare `on` something}。

\subsection*{Data.Char}

\acodered{Data.Char}正如其名,是一个用于处理字符的模块,同样对过滤或映射字符串有很大的帮助,因为字符串本身就是字符的列表。

\acodered{Data.Char}提供了很多关于字符范围的函数。也就是函数接受一个字符,并告诉我们哪些假设为真或为假:

\acodered{isControl}判断一个字符是否是控制字符

\acodered{isSpace}判断一个字符是否是空格字符,包括空格,tab,换行符等

\acodered{isLower}判断一个字符是否为小写

\acodered{isUpper}判断一个字符是否为大写

\acodered{isAlpha}判断一个字符是否为字母

\acodered{isAlphaNum}判断一个字符是否为字母或数字

\acodered{isPrint}判断一个字符是否可打印

\acodered{isDigit}判断一个字符是否为数字

\acodered{isOctDigit}判断一个字符是否为八进制数字

\acodered{isHexDigit}判断一个字符是否为十六进制数字

\acodered{isLetter}判断一个字符是否为字母

\acodered{isMark}判断一个字符是否为 unicode 注意字符(法语)

\acodered{isNumber}判断一个字符是否为数字

\acodered{isPuncuation}判断一个字符是否为标点符号

\acodered{isSymbol}判断一个字符是否为货币符号

\acodered{isSeparator}判断一个字符是否为 unicode 空格或分隔符

\acodered{isAscii}判断一个字符是否在 unicode 字母表的前 128 位

\acodered{isLatin1}判断一个字符是否为 unicode 字母表的前 256 位

\acodered{isAsciiUpper}判断一个字符是否为大写的 Ascii

\acodered{isAsciiLower}判断一个字符是否为小写的 Ascii

以上所有判断函数的类型声明都是\acode{Char -> Bool},大多数时候我们会用它们来过滤字符串或者其它。例如假设我们做一个程序用来获取用户名,
而用户名只能由字母与数字构成,我们可以使用\acode{Data.List}里的\acode{all}函数与\acode{Data.Char}里的子句用于判断用户名是否正确:

\begin{lstlisting}[language=Haskell]
  ghci> all isAlphaNum "bobby283"
  True
  ghci> all isAlphaNum "eddy the fish!"
  False
\end{lstlisting}

同样可以通过\acode{isSpace}来模拟\acode{Data.List}中的\acode{words}函数:

\begin{lstlisting}[language=Haskell]
  ghci> words "hey guys its me"
  ["hey","guys","its","me"]
  ghci> groupBy ((==) `on` isSpace) "hey guys its me"
  ["hey"," ","guys"," ","its"," ","me"]
\end{lstlisting}

呃这看起来像是\acode{words}不过我们留下来空格,那么再用一次\acode{filter}

\begin{lstlisting}[language=Haskell]
  ghci> filter (not . any isSpace) . groupBy ((==) `on` isSpace) $ "hey guys its me"
  ["hey","guys","its","me"]
\end{lstlisting}

\acode{Data.Char}同样提供了一个类似\acode{Ordering}的数据类型。\acode{Ordering}类型可以是\acode{LT},\acode{EQ}或\acode{GT},
类似于枚举。而\acode{GeneralCategory}同样也是枚举,它提供了字符所处的范围\acode{generalCategory :: Char -> GeneralCategory},
由 31 个种类,这里只列举部分:

\begin{lstlisting}[language=Haskell]
  ghci> generalCategory ' '
  Space
  ghci> generalCategory 'A'
  UppercaseLetter
  ghci> generalCategory 'a'
  LowercaseLetter
  ghci> generalCategory '.'
  OtherPunctuation
  ghci> generalCategory '9'
  DecimalNumber
  ghci> map generalCategory " \t\nA9?|"
  [Space,Control,Control,UppercaseLetter,DecimalNumber,OtherPunctuation,MathSymbol]
\end{lstlisting}

由于\acode{GeneralCategory}类型属于\acode{Eq} typeclass,也就可以这样进行测试\acode{generalCategory c == Space}。

\acodered{toUpper}转换一个字符成为大写。空格、数字等不变。

\acodered{toLower}转换一个字符成为小写。

\acodered{toTitle}转换一个字符成为 title-case,大多数字符就是大写。

\acodered{digitToInt}将一个字符转为 Int 值,这个字符必须在\acode{'1'..'9','a'..'f'}或是\acode{‘A’..'F'}的范围之内。

\begin{lstlisting}[language=Haskell]
  ghci> map digitToInt "34538"
  [3,4,5,3,8]
  ghci> map digitToInt "FF85AB"
  [15,15,8,5,10,11]
\end{lstlisting}

\acodered{intToDigit}则与\acode{digitToInt}相反。

\begin{lstlisting}[language=Haskell]
  ghci> intToDigit 15
  'f'
  ghci> intToDigit 5
  '5'
\end{lstlisting}

\acodered{ord}与\acode{chr}函数将字符转换成相应的数值,反之亦然:

\begin{lstlisting}[language=Haskell]
  ghci> ord 'a'
  97
  ghci> chr 97
  'a'
  ghci> map ord "abcdefgh"
  [97,98,99,100,101,102,103,104]
\end{lstlisting}

下面是\acode{encode}与\acode{decode}:

\begin{lstlisting}[language=Haskell]
  encode :: Int -> String -> String
  encode shift msg
      let ords = map ord msg
          shifted = map (+ shift) ords
      in  map chr shifted
\end{lstlisting}

\begin{lstlisting}[language=Haskell]
  ghci> encode 3 "Heeeeey"
  "Khhhhh|"
  ghci> encode 4 "Heeeeey"
  "Liiiii}"
  ghci> encode 1 "abcd"
  "bcde"
  ghci> encode 5 "Marry Christmas! Ho ho ho!"
  "Rfww~%Hmwnxyrfx&%Mt%mt%mt&"
\end{lstlisting}

\begin{lstlisting}[language=Haskell]
  decode :: Int -> String -> String
  decode shift msg = encode (negate shift) msg
\end{lstlisting}

\begin{lstlisting}[language=Haskell]
  ghci> encode 3 "Im a little teapot"
  "Lp#d#olwwoh#whdsrw"
  ghci> decode 3 "Lp#d#olwwoh#whdsrw"
  "Im a little teapot"
  ghci> decode 5 . encode 5 $ "This is a sentence"
  "This is a sentence"
\end{lstlisting}

\subsection*{Data.Map}

关联列表(通常也被称为字典)是一种用于存储键值对却不保证顺序的列表。

由于\acode{Data.Map}中的函数会与\acode{Prelude}中的\acode{Data.List}冲突,因此需要 qualified import:

\begin{lstlisting}[language=Haskell]
import qualified Data.Map as Map
\end{lstlisting}

现在让我们看一下\acode{Data.Map}中有什么好东西!

\acodered{fromList}函数接受一个关联列表,并返回一个 map:

\begin{lstlisting}[language=Haskell]
  ghci> Map.fromList [("betty","555-2938"),("bonnie","452-2928"),("lucille","205-2928")]
  fromList [("betty","555-2938"),("bonnie","452-2928"),("lucille","205-2928")]
  ghci> Map.fromList [(1,2),(3,4),(3,2),(5,5)]
  fromList [(1,2),(3,2),(5,5)]
\end{lstlisting}

如果有重复的键出现,那么之前的值会被丢弃,这里是\acode{fromList}的类型签名:

\begin{lstlisting}[language=Haskell]
  Map.fromList :: (Ord k) => [(k, v)] -> Map.Map k v
\end{lstlisting}

意为接受一个键值类型\acode{k}与\acode{v}对的列表,并返回一个 map。注意这里的键必须属于\acode{Eq}以及\acode{Ord} typeclass,
后者是因为需要将键值安排在树结构中。

在需要键值关联的类型时总是使用\acode{Data.Map},除非键不属于\acode{Ord} typeclass。

\acodered{empty}没有参数,返回一个空 map:

\begin{lstlisting}[language=Haskell]
  ghci> Map.empty
  fromList []
\end{lstlisting}

\acodered{insert}接受一键,一值,并返回 map:

\begin{lstlisting}[language=Haskell]
  ghci> Map.empty
  fromList []
  ghci> Map.insert 3 100 Map.empty
  fromList [(3,100)]
  ghci> Map.insert 5 600 (Map.insert 4 200 ( Map.insert 3 100  Map.empty))
  fromList [(3,100),(4,200),(5,600)]
  ghci> Map.insert 5 600 . Map.insert 4 200 . Map.insert 3 100 $ Map.empty
  fromList [(3,100),(4,200),(5,600)]
\end{lstlisting}

我们可以通过空 map,\acode{insert}以及\acode{foldr}来实现自己的\acode{fromList}:

\begin{lstlisting}[language=Haskell]
  fromList' :: (Ord k) => [(k, v)] -> Map.Map k v
  fromList' = foldr (\(k, v) acc -> Map.insert k v acc) Map.empty
\end{lstlisting}

\acodered{null}检查 map 是否为空:

\begin{lstlisting}[language=Haskell]
  ghci> Map.null Map.empty
  True
  ghci> Map.null $ Map.fromList [(2,3),(5,5)]
  False
\end{lstlisting}

\acodered{size}告知 map 大小:

\begin{lstlisting}[language=Haskell]
  ghci> Map.size Map.empty
  0
  ghci> Map.size $ Map.fromList [(2,4),(3,3),(4,2),(5,4),(6,4)]
  5
\end{lstlisting}

\acodered{singleton}创建一个只有一对键值的 map:

\begin{lstlisting}[language=Haskell]
  ghci> Map.singleton 3 9
  fromList [(3,9)]
  ghci> Map.insert 5 9 $ Map.singleton 3 9
  fromList [(3,9),(5,9)]
\end{lstlisting}

\acodered{lookup}类似于\acode{Data.List}的\acode{lookup},不过它作用于 map。如果找到了则返回\acode{Just something},
反之\acode{Nothing}

\acodered{member}是一个子句,接受一个键以及一个 map,查找该键是否在 map 中:

\begin{lstlisting}[language=Haskell]
  ghci> Map.member 3 $ Map.fromList [(3,6),(4,3),(6,9)]
  True
  ghci> Map.member 3 $ Map.fromList [(2,5),(4,5)]
  False
\end{lstlisting}

\acodered{map}与\acodered{filter}于列表的函数相似:

\begin{lstlisting}[language=Haskell]
  ghci> Map.map (*100) $ Map.fromList [(1,1),(2,4),(3,9)]
  fromList [(1,100),(2,400),(3,900)]
  ghci> Map.filter isUpper $ Map.fromList [(1,'a'),(2,'A'),(3,'b'),(4,'B')]
  fromList [(2,'A'),(4,'B')]
\end{lstlisting}

\acodered{toList}与\acode{fromList}相反:

\begin{lstlisting}[language=Haskell]
  ghci> Map.toList . Map.insert 9 2 $ Map.singleton 4 3
  [(4,3),(9,2)]
\end{lstlisting}

\acodered{keys}与\acodered{elems}分别返回键与值的列表。\acode{keys}等同于\acode{map fst . Map.toList},而\acode{elems}
等同于\acode{map snd . Map.toList}。

\acodered{fromListWith}这个函数很酷。它作用类似于\acode{fromList},不过不会丢弃重复键而是使用函数来决定去留。假设我们
有这样一个列表(注意在 ghci 中使用\acode{:\{}与\acode{:\}}作为多行输入的开头与结尾):

\begin{lstlisting}[language=Haskell]
  phoneBook =
      [("betty","555-2938")
      ,("betty","342-2492")
      ,("bonnie","452-2928")
      ,("patsy","493-2928")
      ,("patsy","943-2929")
      ,("patsy","827-9162")
      ,("lucille","205-2928")
      ,("wendy","939-8282")
      ,("penny","853-2492")
      ,("penny","555-2111")
      ]
\end{lstlisting}

如果使用\acode{fromList}来生成一个 map,我们将会丢弃一些号码!因此这里我们这么做:

\begin{lstlisting}[language=Haskell]
  phoneBookToMap :: (Ord k) => [(k, String)] -> Map.Map k String
  phoneBookToMap = Map.fromListWith (\number1 number2 -> number1 ++ ", " ++ number2)
\end{lstlisting}

\begin{lstlisting}[language=Haskell]
  ghci> Map.lookup "patsy" $ phoneBookToMap phoneBook
  "827-9162, 943-2929, 493-2928"
  ghci> Map.lookup "wendy" $ phoneBookToMap phoneBook
  "939-8282"
  ghci> Map.lookup "betty" $ phoneBookToMap phoneBook
  "342-2492, 555-2938"
\end{lstlisting}

如果遇到一个重复的键,传入的函数用于将这些值结合成为其他值。我们还可以首先将所有值映射为单例的列表后在使用\acode{++}来结合:

\begin{lstlisting}[language=Haskell]
  phoneBookToMap' :: (Ord k) => [(k, a)] -> Map.Map k [a]
  phoneBookToMap' = Map.fromListWith (++) . map (\(k, v) -> (k, [v]))
\end{lstlisting}

\begin{lstlisting}[language=Haskell]
  ghci> Map.lookup "patsy" $ phoneBookToMap' phoneBook
  Just ["827-9162","943-2929","493-2928"]
\end{lstlisting}

非常的精妙!另一个用例则是如果我们从一个数值关联的列表创建 map,当遇到重复键时,我们希望较大的键的值能保留:

\begin{lstlisting}[language=Haskell]
  ghci> Map.fromListWith max [(2,3),(2,5),(2,100),(3,29),(3,22),(3,11),(4,22),(4,15)]
  fromList [(2,100),(3,29),(4,22)]
\end{lstlisting}

或者将重复键的值相加:

\begin{lstlisting}[language=Haskell]
  ghci> Map.fromListWith (+) [(2,3),(2,5),(2,100),(3,29),(3,22),(3,11),(4,22),(4,15)]
  fromList [(2,108),(3,62),(4,37)]
\end{lstlisting}

\acodered{insertWith}则是插入一个键值对进入 map,如果键重复则使用传递的函数进行判断:

\begin{lstlisting}[language=Haskell]
  ghci> Map.insertWith (+) 3 100 $ Map.fromList [(3,4),(5,103),(6,339)]
  fromList [(3,104),(5,103),(6,339)]
\end{lstlisting}

根据官网的介绍现在最好是用
\href{https://downloads.haskell.org/ghc/latest/docs/libraries/containers-0.6.7/Data-Map.html}{Data.Map.Strict}
(2023/7/14)。

\subsection*{Data.Set}

由于\acode{Data.Set}中的名称会与\acode{Prelude}与\acode{Data.List}重复,因此需要 qualified import:

\begin{lstlisting}[language=Haskell]
  import qualified Data.Set as Set
\end{lstlisting}

\acodered{fromList}将列表转换为集合:

\begin{lstlisting}[language=Haskell]
  ghci> let set1 = Set.fromList text1
  ghci> let set2 = Set.fromList text2
  ghci> set1
  fromList " .?AIRadefhijlmnorstuy"
  ghci> set2
  fromList " !Tabcdefghilmnorstuvwy"
\end{lstlisting}

\acodered{intersection}取交集:

\begin{lstlisting}[language=Haskell]
  ghci> Set.intersection set1 set2
  fromList " adefhilmnorstuy"
\end{lstlisting}

\acodered{difference}取在第一个集合却不在第二个集合的元素:

\begin{lstlisting}[language=Haskell]
  ghci> Set.difference set1 set2
  fromList ".?AIRj"
  ghci> Set.difference set2 set1
  fromList "!Tbcgvw"
\end{lstlisting}

\acodered{union}取并集:

\begin{lstlisting}[language=Haskell]
  ghci> Set.union set1 set2
  fromList " !.?AIRTabcdefghijlmnorstuvwy"
\end{lstlisting}

\acodered{null},\acodered{size},\acodered{member},\acodered{empty},\acodered{singleton},\acodered{insert}以及
\acodered{delete}函数与预期一致:

\begin{lstlisting}[language=Haskell]
  ghci> Set.null Set.empty
  True
  ghci> Set.null $ Set.fromList [3,4,5,5,4,3]
  False
  ghci> Set.size $ Set.fromList [3,4,5,3,4,5]
  3
  ghci> Set.singleton 9
  fromList [9]
  ghci> Set.insert 4 $ Set.fromList [9,3,8,1]
  fromList [1,3,4,8,9]
  ghci> Set.insert 8 $ Set.fromList [5..10]
  fromList [5,6,7,8,9,10]
  ghci> Set.delete 4 $ Set.fromList [3,4,5,4,3,4,5]
  fromList [3,5]
\end{lstlisting}

也可以检查子集等:

\begin{lstlisting}[language=Haskell]
  ghci> Set.fromList [2,3,4] `Set.isSubsetOf` Set.fromList [1,2,3,4,5]
  True
  ghci> Set.fromList [1,2,3,4,5] `Set.isSubsetOf` Set.fromList [1,2,3,4,5]
  True
  ghci> Set.fromList [1,2,3,4,5] `Set.isProperSubsetOf` Set.fromList [1,2,3,4,5]
  False
  ghci> Set.fromList [2,3,4,8] `Set.isSubsetOf` Set.fromList [1,2,3,4,5]
  False
\end{lstlisting}

\acodered{map}与\acodered{filter}:

\begin{lstlisting}[language=Haskell]
  ghci> Set.filter odd $ Set.fromList [3,4,5,6,7,2,3,4]
  fromList [3,5,7]
  ghci> Set.map (+1) $ Set.fromList [3,4,5,6,7,2,3,4]
  fromList [3,4,5,6,7,8]
\end{lstlisting}

\acodered{toList}转为列表,按字典顺序进行排序:

\begin{lstlisting}[language=Haskell]
  ghci> let setNub xs = Set.toList $ Set.fromList xs
  ghci> setNub "HEY WHATS CRACKALACKIN"
  " ACEHIKLNRSTWY"
  ghci> nub "HEY WHATS CRACKALACKIN"
  "HEY WATSCRKLIN"
\end{lstlisting}

\subsection*{构建自己的模块}

首先我们创建一个名为\acode{Geometry.hs}的文件。

当我们提到模块\textit{exports}函数,即当导入一个模块时,可以使用该模块导出的函数。

在模块的开头,需要指定模块名称,接着指定想要导出的函数:

\begin{lstlisting}[language=Haskell]
  module Geometry
    ( sphereVolume,
      sphereArea,
      cubeVolume,
      cubeArea,
      cuboidArea,
      cuboidVolume,
    )
  where
\end{lstlisting}

接下来就是定义函数:

\begin{lstlisting}[language=Haskell]
  module Geometry
    ( sphereVolume,
      sphereArea,
      cubeVolume,
      cubeArea,
      cuboidArea,
      cuboidVolume,
    )
  where

  import Control.Monad

  sphereVolume :: Float -> Float
  -- sphereVolume radius = (4.0 / 3.0) * pi * (radius ^3)
  sphereVolume = (* (4.0 / 3.0 * pi)) . (^ 3)

  sphereArea :: Float -> Float
  -- sphereArea radius = 4 * pi * (radius ^ 2)
  sphereArea = (* (4 * pi)) . (^ 3)

  cubeVolume :: Float -> Float
  -- cubeVolume side = cuboidVolume side side side
  cubeVolume = join (join cuboidVolume)

  cubeArea :: Float -> Float
  -- cubeArea side = cuboidArea side side side
  cubeArea = join (join cuboidArea)

  cuboidVolume :: Float -> Float -> Float -> Float
  -- cuboidVolume a b c = rectangleArea a b * c
  cuboidVolume = ((*) .) . rectangleArea

  cuboidArea :: Float -> Float -> Float -> Float
  cuboidArea a b c = rectangleArea a b * 2 + rectangleArea a c * 2 + rectangleArea c b * 2

  rectangleArea :: Float -> Float -> Float
  -- rectangleArea a b = a * b
  rectangleArea = (*)
\end{lstlisting}

我们仅需像下面这样就能使用模块:

\begin{lstlisting}[language=Haskell]
import Geometry
\end{lstlisting}

需要注意的是\acode{Geometry.hs}需要在加载模块的程序同一文件夹下。

模块童谣可以被给予等级结构。每个模块可以拥有若干子模块,子模块里还可以有子模块。

首先创建一个名为\acode{Geometry}的文件夹,接着是下面的三个文件夹:\acode{Sphere.hs},\acode{Cuboid.hs}以及\acode{Cube.hs}

\acode{Sphere.hs}:

\begin{lstlisting}[language=Haskell]
  module Geometry.Sphere
  ( volume,
    area,
  )
  where

  volume :: Float -> Float
  volume = (* (4.0 / 3.0 * pi)) . (^ 3)

  area :: Float -> Float
  area = (* (4 * pi)) . (^ 3)
\end{lstlisting}

\acode{Cuboid.hs}:

\begin{lstlisting}[language=Haskell]
  module Geometry.Cuboid
    ( volume,
      area,
    )
  where

  volume :: Float -> Float -> Float -> Float
  volume = ((*) .) . rectangleArea

  area :: Float -> Float -> Float -> Float
  area a b c = rectangleArea a b * 2 + rectangleArea a c * 2 + rectangleArea c b * 2

  rectangleArea :: Float -> Float -> Float
  rectangleArea = (*)
\end{lstlisting}

以及\acode{Cube.hs}:

\begin{lstlisting}[language=Haskell]
  module Geometry.Cube
    ( volume,
      area,
    )
  where

  import Control.Monad
  import Geometry.Cuboid qualified as Cuboid

  volume :: Float -> Float
  volume = join (join Cuboid.volume)

  area :: Float -> Float
  area = join (join Cuboid.area)
\end{lstlisting}

那么在与文件夹同级的目录下就可以这样加载模块中的子模块:

\begin{lstlisting}[language=Haskell]
import Geometry.Sphere
\end{lstlisting}

为了避免命名冲突可以使用 qualified import:

\begin{lstlisting}[language=Haskell]
  import qualified Geometry.Sphere as Sphere
  import qualified Geometry.Cuboid as Cuboid
  import qualified Geometry.Cube as Cube
\end{lstlisting}

\end{document}
