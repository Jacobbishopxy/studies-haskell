\documentclass[./main.tex]{subfiles}

\begin{document}

\subsection*{第一个函数}

在 \acode{./test/} 文件夹下创建一个 \acode{baby.hs} 的文件,写入:

\begin{lstlisting}[language=Haskell]
  doubleMe x = x + x
\end{lstlisting}

使用 \acode{ghci} 加载该文件(在本项目根目录时使用 \acode{:l tests/baby}):

\begin{lstlisting}
  ghci> :l baby
  [1 of 1] Compiling Main             ( baby.hs, interpreted )
  Ok, modules loaded: Main.
  ghci> doubleMe 9
  18
  ghci> doubleMe 8.3
  16.6
\end{lstlisting}

一个带有 \acode{if} 的函数:

\begin{lstlisting}[language=Haskell]
  doubleSmallNumber x =
    if x > 100
      then x
      else x * 2
\end{lstlisting}

Haskell 中的 \acode{if} 声明是一个 \textit{表达式},那么 \acode{else} 是强制性的,因为表达式一定要有所返回。
因此加上述函数可以改写为:

\begin{lstlisting}[language=Haskell]
  doubleSmallNumber' x = (if x > 100 then x else x * 2) + 1
\end{lstlisting}

这里的 \acode{'} 符号是 Haskell 中的有效字符,且在 Haskell 中并没有特殊的意义,因此可以用作于函数名。通常情况下,
使用 \acode{'} 代表着一个函数(非懒加载的函数)的严格版本,或是一个有细微变化的函数或者变量。又因为 \acode{'} 是一个
有效字符,那么可以创建以下函数:

\begin{lstlisting}[language=Haskell]
  conanO'Brien = "It's a-me, Conan O'Brien!"
\end{lstlisting}

这里又有两点值得注意的地方。首先,函数名不能以大写开头,稍后会进行说明;其次,该函数并没有任何入参。当一个函数没有入参,
我们通常称其为一个 \textit{定义 definition},因为一旦定义了它便不能修改其名称,以及其返回。

\subsection*{list 的介绍}

Haskell 中的 list 是 \textbf{同质的 homogenous} 数据结构。

\begin{anote}
  在 \acode{GHCI} 中可以使用 \acode{let} 关键字定义一个名称。换言之,\acode{GHCI} 中的 \acode{let a = 1}
  等同于脚本中的 \acode{a = 1}。
\end{anote}

通常使用 \acode{++} 操作符将两个数组进行合并:

\begin{lstlisting}[language=Haskell]
  ghci> [1,2,3,4] ++ [9,10,11,12]
  [1,2,3,4,9,10,11,12]
  ghci> "hello" ++ " " ++ "world"
  "hello world"
  ghci> ['w','o'] ++ ['o','t']
  "woot"
\end{lstlisting}

可以使用 \acode{:} 操作符将元素直接添加至数组头部:

\begin{lstlisting}[language=Haskell]
  ghci> 'A':" SMALL CAT"
  "A SMALL CAT"
  ghci> 5:[1,2,3,4,5]
  [5,1,2,3,4,5]
\end{lstlisting}

实际上,\acode{[1, 2, 3]} 是 \acode{1:2:3:[]} 的语法糖,其中 \acode{[]} 为一个空数组。如果头部追加 \acode{3},
\acode{[]} 就变成了 \acode{[3]},再次进行头部追加 \acode{2},则变为 \acode{[2, 3]},以此类推。

如果希望通过索引获取数组中的元素,那么可以使用 \acode{!!} 操作符:

\begin{lstlisting}[language=Haskell]
  ghci> "Steve Buscemi" !! 6
  'B'
  ghci> [9.4,33.2,96.2,11.2,23.25] !! 1
  33.2
\end{lstlisting}

超出索引时则会报错。

数组还可以通过操作符 \acode{<},\acode{<=},\acode{==},\acode{>} 以及 \acode{>=} 操作符来进行比较,
而比较的方式则是顺序比较。当进行头部比较元素相等时,再进行下一个元素进行比较。

数组的四种基础操作 \acodered{head},\acodered{tail},\acodered{last} 以及 \acodered{init}:

\begin{lstlisting}[language=Haskell]
  ghci> head [5,4,3,2,1]
  5
  ghci> tail [5,4,3,2,1]
  [4,3,2,1]
  ghci> last [5,4,3,2,1]
  1
  ghci> init [5,4,3,2,1]
  [5,4,3,2]
\end{lstlisting}

当使用上述四种操作时,需要注意是否应用于空数组,这样的错误在编译期并不能被发现。

其它的操作:

\begin{enumerate}
  \item \acodered{length} 获取数组长度;
  \item \acodered{null} 检查数组是否为空;
  \item \acodered{reverse} 翻转数组;
  \item \acodered{take} 获取数组的头几个元素的数组;
  \item \acodered{drop} 移除数组的头几个元素,并返回剩余元素的数组;
  \item \acodered{maximum} 获取最大值;
  \item \acodered{minimum} 获取最小值;
  \item \acodered{sum} 求和;
  \item \acodered{product} 求积;
  \item \acodered{elem} 元素是否存在于数组中。
\end{enumerate}

\begin{lstlisting}[language=Haskell]
  ghci> length [5,4,3,2,1]
  5

  ghci> null [1,2,3]
  False
  ghci> null []
  True

  ghci> reverse [5,4,3,2,1]
  [1,2,3,4,5]

  ghci> take 3 [5,4,3,2,1]
  [5,4,3]
  ghci> take 1 [3,9,3]
  [3]
  ghci> take 5 [1,2]
  [1,2]
  ghci> take 0 [6,6,6]
  []

  ghci> drop 3 [8,4,2,1,5,6]
  [1,5,6]
  ghci> drop 0 [1,2,3,4]
  [1,2,3,4]
  ghci> drop 100 [1,2,3,4]
  []

  ghci> minimum [8,4,2,1,5,6]
  1
  ghci> maximum [1,9,2,3,4]
  9

  ghci> sum [5,2,1,6,3,2,5,7]
  31
  ghci> product [6,2,1,2]
  24
  ghci> product [1,2,5,6,7,9,2,0]
  0

  ghci> 4 `elem` [3,4,5,6]
  True
  ghci> 10 `elem` [3,4,5,6]
  False
\end{lstlisting}

\subsection*{Texas 排列}

\begin{lstlisting}[language=Haskell]
  ghci> [1..20]
  [1,2,3,4,5,6,7,8,9,10,11,12,13,14,15,16,17,18,19,20]
  ghci> ['a'..'z']
  "abcdefghijklmnopqrstuvwxyz"
  ghci> ['K'..'Z']
  "KLMNOPQRSTUVWXYZ"
\end{lstlisting}

带有 step 的排列:

\begin{lstlisting}[language=Haskell]
  ghci> [2,4..20]
  [2,4,6,8,10,12,14,16,18,20]
  ghci> [3,6..20]
  [3,6,9,12,15,18]
\end{lstlisting}

而对于浮点数的排列需要注意精度问题:

\begin{lstlisting}[language=Haskell]
  ghci> [0.1, 0.3 .. 1]
  [0.1,0.3,0.5,0.7,0.8999999999999999,1.0999999999999999]
\end{lstlisting}

以下是若干用于生产无限长度数组的函数:

\acodered{cycle} 循环周期:

\begin{lstlisting}[language=Haskell]
  ghci> take 10 (cycle [1,2,3])
  [1,2,3,1,2,3,1,2,3,1]
  ghci> take 12 (cycle "LOL ")
  "LOL LOL LOL "
\end{lstlisting}

\acodered{repeat} 重复:

\begin{lstlisting}[language=Haskell]
  ghci> take 10 (repeat 5)
  [5,5,5,5,5,5,5,5,5,5]
\end{lstlisting}

另外就是 \acodered{replicate} 函数可以重复单个元素:

\begin{lstlisting}[language=Haskell]
  ghci> replicate 3 10
  [10,10,10]
\end{lstlisting}

\subsection*{列表表达式}

数学里的 \textit{集合表达式 set comprehensions} 例如 $S = { 2 \cdot x | x \in \mathbb{N}, x \le 10 }$;
Haskell 中的列表表达式,例如 1 至 10 数组中每个元素乘以 2:

\begin{lstlisting}[language=Haskell]
  ghci> [x*2 | x <- [1..10]]
  [2,4,6,8,10,12,14,16,18,20]
\end{lstlisting}

为列表表达式添加条件(或称谓语 predicate):

\begin{lstlisting}[language=Haskell]
  ghci> [x*2 | x <- [1..10], x*2 >= 12]
  [12,14,16,18,20]
  ghci> [ x | x <- [50..100], x `mod` 7 == 3]
  [52,59,66,73,80,87,94]
\end{lstlisting}

将列表表达式置于一个函数中便于复用:

\begin{lstlisting}[language=Haskell]
  ghci> boomBangs xs = [ if x < 10 then "BOOM!" else "BANG!" | x <- xs, odd x]
  ghci> boomBangs [7..13]
  ["BOOM!","BOOM!","BANG!","BANG!"]
\end{lstlisting}

多个谓语也是可以的:

\begin{lstlisting}[language=Haskell]
  ghci> [ x | x <- [10..20], x /= 13, x /= 15, x /= 19]
  [10,11,12,14,16,17,18,20]
\end{lstlisting}

除此之外,还可以处理若干数组:

\begin{lstlisting}[language=Haskell]
  ghci> [ x*y | x <- [2,5,10], y <- [8,10,11]]
  [16,20,22,40,50,55,80,100,110]
\end{lstlisting}

当然也可以加上谓语:

\begin{lstlisting}[language=Haskell]
  ghci> [ x*y | x <- [2,5,10], y <- [8,10,11], x*y > 50]
  [55,80,100,110]
\end{lstlisting}

那么对于字符串也可以使用列表表达式:

\begin{lstlisting}[language=Haskell]
  ghci> let nouns = ["hobo","frog","pope"]
  ghci> let adjectives = ["lazy","grouchy","scheming"]
  ghci> [adjective ++ " " ++ noun | adjective <- adjectives, noun <- nouns]
  ["lazy hobo","lazy frog","lazy pope","grouchy hobo","grouchy frog",
  "grouchy pope","scheming hobo","scheming frog","scheming pope"]
\end{lstlisting}

现在让我们编写一个自己的 \acode{length},命名 \acode{length'}(这里的 \acode{_} 意为无需使用的变量):

\begin{lstlisting}[language=Haskell]
  length' xs = sum [1 | _ <- xs]
\end{lstlisting}

由于字符串是数组,因此我们可以使用列表表达式处理并生产字符串。以下是一个移除所有字符但保留大写字符的函数:

\begin{lstlisting}[language=Haskell]
  removeNonUppercase st = [ c | c <- st, c `elem` ['A'..'Z']]
  removeUppercase st = [ c | c <- st, c `notElem` ['A'..'Z']]
\end{lstlisting}

\subsection*{元组}

在某种程度上,元组类似于数组 -- 存储若干值至单个变量上。然而有一些基础的差异:数组长度可以无限,元组长度固定;数组中元素类型
是同质的,而元组则可以是异质的 heterogenous。

对于对元组(当且仅当包含两个元素)有以下操作:

\acodered{fst} 获取对元组的第一个元素:

\begin{lstlisting}[language=Haskell]
  ghci> fst (8,11)
  8
  ghci> fst ("Wow", False)
  "Wow"
\end{lstlisting}

\acodered{snd} 获取对元组的第二个元素:

\begin{lstlisting}[language=Haskell]
  ghci> snd (8,11)
  11
  ghci> snd ("Wow", False)
  False
\end{lstlisting}

另外一个有意思的函数则是 \acodered{zip},它可以将两个数组按对拼接成对元组的数组

\begin{lstlisting}[language=Haskell]
  ghci> zip [1,2,3,4,5] [5,5,5,5,5]
  [(1,5),(2,5),(3,5),(4,5),(5,5)]
  ghci> zip [1 .. 5] ["one", "two", "three", "four", "five"]
  [(1,"one"),(2,"two"),(3,"three"),(4,"four"),(5,"five")]
\end{lstlisting}

当两个数组的长度不一时,\acode{zip} 则按最短的那个进行对齐,长的数组剩余部分则被丢弃,这是因为 Haskell 是懒加载的缘故。

\begin{lstlisting}[language=Haskell]
  ghci> zip [5,3,2,6,2,7,2,5,4,6,6] ["im","a","turtle"]
  [(5,"im"),(3,"a"),(2,"turtle")]
  ghci> zip [1..] ["apple", "orange", "cherry", "mango"]
  [(1,"apple"),(2,"orange"),(3,"cherry"),(4,"mango")]
\end{lstlisting}

\end{document}
