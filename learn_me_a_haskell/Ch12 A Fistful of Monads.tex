\documentclass[./main.tex]{subfiles}

\begin{document}

当我们第一次谈及函子时,得知这是一个用于映射的抽象概念。接着就是高级函子,让我们将某些类型视为在某些 contexts 中,保留 contexts
的同时还能应用普通函数在 contexts 中的值上。

本章我们开始学习单子 monads,它是一个增强版的高级函子,正如是一个高级函子是函子的增强版那样。

单子是高级函子的自然演进,可以这样考虑:如果有一个带 context 的值,\acode{m a},该如何将其应用至一个接受普通\acode{a}并返回一个
context 的函数?也就是说,如何应用一个类型为\acode{a -> m b}的函数至一个类型为\acode{m a}的值?概括一下就是想要一个这样的函数:

\begin{lstlisting}[language=Haskell]
  (>>=) :: (Monad m) => m a -> (a -> m b) -> m b
\end{lstlisting}

\textbf{当我们拥有一个漂亮的值以及一个接受普通值但返回漂亮值的函数,我们如何才能将漂亮值喂给这个函数?}这就是单子处理的主要问题。
这里用\acode{m a}而不是\acode{f a}是因为\acode{m}意为\acode{Monad},而单子就是支持\acode{>>=}操作的高级函子,此处的
\acode{>>=}函数读作\textit{绑定 bind}。

\subsection*{从处理\acode{Maybe}开始}

不出意外,\acode{Maybe}就是一个单子,让我看看它是如何以单子运作的。

\acode{Maybe a}代表一个类型\acode{a}的值处于有可能失败的 context 中。\acode{Just "dharma"}意为字符串\acode{"dharma"}
同时\acode{Nothing}代表着缺值,或者说如果将字符串作为计算的结果,那么\acode{Nothing}则代表计算失败了。

将\acode{Maybe}视为一个函子时,想要的是\acode{fmap}一个函数在其值,也就是\acode{Just}中的元素,否则保留\acode{Nothing}
状态,即内部没有任何元素。

\begin{lstlisting}[language=Haskell]
  ghci> fmap (++"!") (Just "wisdom")
  Just "wisdom!"
  ghci> fmap (++"!") Nothing
  Nothing
\end{lstlisting}

将\acode{Maybe}视为一个高级函子时,applicatives 同样包装了函数。使用\acode{<*>}应用一个函数在\acode{Maybe}内的值,
它们都必须是包在\acode{Just}来代表值存在,否则返回\acode{Nothing}。

\begin{lstlisting}[language=Haskell]
  ghci> Just (+3) <*> Just 3
  Just 6
  ghci> Nothing <*> Just "greed"
  Nothing
  ghci> Just ord <*> Nothing
  Nothing
\end{lstlisting}

当我们使用 applicative 风格让普通函数作用在\acode{Maybe}值时,所有的值都需要时\acode{Just}值,否则返回\acode{Nothing}!

\begin{lstlisting}[language=Haskell]
  ghci> max <$> Just 3 <*> Just 6
  Just 6
  ghci> max <$> Just 3 <*> Nothing
  Nothing
\end{lstlisting}

现在让我们想想该如何使用\acode{>>=}在\acode{Maybe}上。正如我们所说的,\acode{>>=}接受一个 monadic 值以及一个接受普通值并
返回一个 monadic 值的函数。

\acode{>>=}将接受一个\acode{Maybe a}值以及一个类型\acode{a -> Maybe a}的函数,以某种方式应用该函数至\acode{Maybe a}。
假设我们有一个函数\acode{\\x -> Just (x+1)}。它接受一个值,加\acode{1}并将其包装在一个\acode{Just}内。

\begin{lstlisting}[language=Haskell]
  ghci> (\x -> Just (x+1)) 1
  Just 2
  ghci> (\x -> Just (x+1)) 100
  Just 101
\end{lstlisting}

现在的问题是如何将一个\acode{Maybe}值喂给该函数?如果思考了\acode{Maybe}是如何作为一个高级函子的,那么该问题的答案就很简单了。
如果拿到一个\acode{Just},就把包在\acode{Just}内的值喂给函数;如果拿到一个\acode{Nothing}则返回\acode{Nothing}。

现在我们不调用\acode{>>=}而是调用\acode{applyMaybe},那么它接受一个\acode{Maybe a}以及一个返回\acode{Maybe b}的函数,
并将该函数应用至\acode{Maybe a}:

\begin{lstlisting}[language=Haskell]
  applyMaybe :: Maybe a -> (a -> Maybe b) -> Maybe b
  applyMaybe Nothing f  = Nothing
  applyMaybe (Just x) f = f x
\end{lstlisting}

现在让我们试试:

\begin{lstlisting}[language=Haskell]
  ghci> Just 3 `applyMaybe` \x -> Just (x+1)
  Just 4
  ghci> Just "smile" `applyMaybe` \x -> Just (x ++ " :)")
  Just "smile :)"
  ghci> Nothing `applyMaybe` \x -> Just (x+1)
  Nothing
  ghci> Nothing `applyMaybe` \x -> Just (x ++ " :)")
  Nothing
\end{lstlisting}

上述例子中,可以看到当使用\acode{applyMaybe}于一个\acode{Just}值以及一个函数,该函数很容易的就被应用到了\acode{Just}
内的值;而但作用于一个\acode{Nothing},那么结果就是\acode{Nothing}。那么如果函数返回的是\acode{Nothing}?

\begin{lstlisting}[language=Haskell]
  ghci> Just 3 `applyMaybe` \x -> if x > 2 then Just x else Nothing
  Just 3
  ghci> Just 1 `applyMaybe` \x -> if x > 2 then Just x else Nothing
  Nothing
\end{lstlisting}

符合预期。如果左侧的 monadic 值是\acode{Nothing},那么整个结果就是\acode{Nothing};如果右侧函数返回的是\acode{Nothing},
结果还是\acode{Nothing}。这非常像是使用\acode{Maybe}作为高级函子时,过程中有任何一个\acode{Nothing}时,整个结果就会是
\acode{Nothing}。

你或许会问,这样有用么?看起来高级函子比单子更强,因为高级函子允许我们用一个普通函数应用至 contexts 中的值上。而单子可以这么做
是因为它们是升级版的高级函子,那肯定是可以做高级函子不能做的事情。

稍后我们再来讨论\acode{Maybe},现在让我们看看属于单子的那些 typeclass。

\subsection*{单子 typeclass}

正如函子拥有\acode{Functor} typeclass,高级函子拥有\acode{Applicative} typeclass,单子也有自己的 typeclass:
\acode{Monad}!

\begin{lstlisting}[language=Haskell]
  class Monad m where
    return :: a -> m a

    (>>=) :: m a -> (a -> m b) -> m b

    (>>) :: m a -> m b -> m b
    x >> y = x >>= \_ -> y

    fail :: String -> m a
    fail msg = error msg
\end{lstlisting}

首先从第一行开始,\acode{class Monad m where}。等等之前不是提到过单子是高级函子的演进吗?那么不应该有一个类约束像是
\acode{class (Applicative m) => Monad m where},也就是一个类型必须先是高级函子才能是单子么?的确要有,但是 Haskell
被创造的早期,人们没有想到高级函子适合被放进语言中,所以最后并没有这个限制。但的确每个单子都是高级函子,即使\acode{Monad}
并没有这么宣告。

\acode{Monad} typeclass 的第一个函数定义就是\acode{return}。它等同于\acode{pure},只不过名字不同,其类型为
\acode{(Monad m) => a -> m a}。接受一个值,将其放入最小默认 context 中。对于\acode{Maybe}而言,就是将值放入
\acode{Just}中。

接下来的是函数\acode{>>=},或绑定。它像是函数应用那样,只不过它接受的不是普通值而是一个 monadic 值(即具有 context 的值)
并把该值喂给一个接受普通值的函数,最后返回一个 monadic 值。

接下来就是函数\acode{>>},现在无需太多的关注因为它拥有一个默认实现,同时在构造\acode{Monad}实例时我们几乎永远不用考虑去实现它。

最后的函数就是\acode{Monad} typeclass 的\acode{fail}。我们永远不会显式的在代码中用到,而是会被 Haskell 用在处理语法错误。
目前不需要太在意\acode{fail}。

现在来看一下\acode{Maybe}的\acode{Monad}实例。

\begin{lstlisting}[language=Haskell]
  instance Monad Maybe where
    return x = Just x
    Nothing >>= f = Nothing
    Just x >>= f  = f x
    fail _ = Nothing
\end{lstlisting}

\acode{return}与\acode{pure}等价。

\acode{>>=}与\acode{applyMaybe}是一样的。当\acode{Maybe a}喂给我们函数时,我们保留 context,并在左值为\acode{Nothing}
时返回\acode{Nothing},左值为\acode{Just}时将\acode{f}应用至其内部值。

\begin{lstlisting}[language=Haskell]
  ghci> return "WHAT" :: Maybe String
  Just "WHAT"
  ghci> Just 9 >>= \x -> return (x*10)
  Just 90
  ghci> Nothing >>= \x -> return (x*10)
  Nothing
\end{lstlisting}

我们已经在\acode{Maybe}使用过\acode{pure}了,这里的\acode{return}就是\acode{pure}。

注意是如何把\acode{Just 9}喂给\acode{\\x -> return (x*10)}的。在函数中\acode{x}绑定到\acode{9}。它看起来不用模式匹配
就能从\acode{Maybe}中抽取值,且并没有丢失\acode{Maybe}的 context。

\subsection*{走钢丝}

现在我们知道了如何将一个\acode{Maybe a}值喂给\acode{a -> Maybe b}类型的函数。现在看看我们如何重复使用\acode{>>=}来处理多个
\acode{Maybe a}值。

...省略原文故事。

我们用一对整数来代表平衡杆的状态。第一个位置代表左侧的鸟的数量,第二个位置代表右侧的鸟的数量。

\begin{lstlisting}[language=Haskell]
  type Birds = Int
  type Pole = (Birds,Birds)
\end{lstlisting}

接下来定义两个函数,它们接受一个代表鸟的数量的数值以及分别放在杆子的左右侧。

\begin{lstlisting}[language=Haskell]
  landLeft :: Birds -> Pole -> Pole
  landLeft n (left,right) = (left + n,right)

  landRight :: Birds -> Pole -> Pole
  landRight n (left,right) = (left,right + n)
\end{lstlisting}

测试:

\begin{lstlisting}[language=Haskell]
  ghci> landLeft 2 (0,0)
  (2,0)
  ghci> landRight 1 (1,2)
  (1,3)
  ghci> landRight (-1) (1,2)
  (1,1)
\end{lstlisting}

鸟儿飞走只需要用负值表达即可。由于函数的输入与返回都是\acode{Pole}的缘故,我们可以串联\acode{landLeft}与\acode{landRight}:

\begin{lstlisting}[language=Haskell]
  ghci> landLeft 2 (landRight 1 (landLeft 1 (0,0)))
  (3,1)
\end{lstlisting}

如果编写这样的一个函数:

\begin{lstlisting}[language=Haskell]
  x -: f = f x
\end{lstlisting}

那么就可以先写参数,然后再是函数:

\begin{lstlisting}[language=Haskell]
  ghci> 100 -: (*3)
  300
  ghci> True -: not
  False
  ghci> (0,0) -: landLeft 2
  (2,0)
\end{lstlisting}

使用这样的方法我们可以用更加可读的方式来重复之前的表达:

\begin{lstlisting}[language=Haskell]
  ghci> (0,0) -: landLeft 1 -: landRight 1 -: landLeft 2
  (3,1)
\end{lstlisting}

那么如果一次性飞入 10 只鸟儿呢?

\begin{lstlisting}[language=Haskell]
  ghci> landLeft 10 (0,3)
  (10,3)
\end{lstlisting}

平衡杆会失去平衡(超过 4 只鸟)!这是显而易见的,但是如果在一系列的操作当中:

\begin{lstlisting}[language=Haskell]
  ghci> (0,0) -: landLeft 1 -: landRight 4 -: landLeft (-1) -: landRight (-2)
  (0,2)
\end{lstlisting}

这看起来没问题,但是实际上中间有一刻是右侧有 4 只鸟,而左侧没有鸟!修复这个问题,我们需重审\acode{landLeft}与\acode{landRight}
函数。这些函数是需要返回失败的。这就是使用\acode{Maybe}的绝佳时刻!

\begin{lstlisting}[language=Haskell]
  landLeft :: Birds -> Pole -> Maybe Pole
  landLeft n (left, right)
    | abs ((left + n) - right) < 4 = Just (left + n, right)
    | otherwise = Nothing

  landRight :: Birds -> Pole -> Maybe Pole
  landRight n (left, right)
    | abs (left - (right + n)) < 4 = Just (left, right + n)
    | otherwise = Nothing
\end{lstlisting}

测试:

\begin{lstlisting}[language=Haskell]
  ghci> landLeft 2 (0,0)
  Just (2,0)
  ghci> landLeft 10 (0,3)
  Nothing
\end{lstlisting}

现在我们需要一种方法,接受一个\acode{Maybe Pole},将其喂给接受\acode{Pole}并返回\acode{Maybe Pole}的函数。幸运的是,我们
拥有\acode{>>=}:

\begin{lstlisting}[language=Haskell]
  ghci> landRight 1 (0,0) >>= landLeft 2
  Just (2,1)
\end{lstlisting}

注意,\acode{landLeft 2}的类型是\acode{Pole -> Maybe Pole}。我们无法将\acode{landRight 1 (0,0)}的\acode{Maybe Pole}
结果喂给它,因此使用了\acode{>>=}来将带有 context 的值给到了\acode{landLeft 2}。\acode{>>=}确实允许我们将\acode{Maybe}
值视为一个带有 context 的值,因为如果将一个\acode{Nothing}喂给\acode{landLeft 2},那么结果就是\acode{Nothing}且失败被
传递下去:

\begin{lstlisting}[language=Haskell]
  ghci> Nothing >>= landLeft 2
  Nothing
\end{lstlisting}

测试一系列的操作:

\begin{lstlisting}[language=Haskell]
  ghci> return (0,0) >>= landRight 2 >>= landLeft 2 >>= landRight 2
  Just (2,4)
\end{lstlisting}

开始时通过\acode{return}将一个 pole 包装进一个\acode{Just}。

稍早之前的一系列操作:

\begin{lstlisting}[language=Haskell]
  ghci> return (0,0) >>= landLeft 1 >>= landRight 4 >>= landLeft (-1) >>= landRight (-2)
  Nothing
\end{lstlisting}

符合预期,最后的情形代表了失败的情况。

如果只把\acode{Maybe}当做高级函子使用的话是没有办法达到我们想要的效果。因为高级函子并不允许 applicative 值之间有弹性的交互。
它们最多就是让我们可以用 applicative 风格来传递参数至函数。applicative 操作符拿到它们的结果并用 applicative 的方式喂给
另一个函数,最后将最终的 applicative 值放在一起。这里面每一步之间并没有多少操作空间。而我们的这个例子需要的是每一步都依赖前一步的
结果。

我们也可以写一个\acode{banana}函数必定返回失败:

\begin{lstlisting}[language=Haskell]
  banana :: Pole -> Maybe Pole
  banana _ = Nothing
\end{lstlisting}

将该函数置于整个过程中,不管前面的状态如何,都会产生失败:

\begin{lstlisting}[language=Haskell]
  ghci> return (0,0) >>= landLeft 1 >>= banana >>= landRight 1
  Nothing
\end{lstlisting}

\acode{Just (1,0)}被喂给\acode{banana}产生\acode{Nothing}之后所有的结果便是\acode{Nothing}了。

除开构建一个忽略输入并返回一个预先设定好的 monadic 值,还可以使用\acode{>>}函数,其默认实现如下:

\begin{lstlisting}[language=Haskell]
  (>>) :: (Monad m) => m a -> m b -> m b
  m >> n = m >>= \_ -> n
\end{lstlisting}

一般而言,碰到一个完全忽略前面状态的函数,它就只会返回它想返回的值。然而碰到单子时,它们的 context 还是必须要被考虑到的。看一下
\acode{>>}串联\acode{Maybe}的情况。

\begin{lstlisting}[language=Haskell]
  ghci> Nothing >> Just 3
  Nothing
  ghci> Just 3 >> Just 4
  Just 4
  ghci> Just 3 >> Nothing
  Nothing
\end{lstlisting}

如果把\acode{>>}换成\acode{>>= \\_ ->},就很容易理解了。

将\acode{banana}改用\acode{>>}与\acode{Nothing}来表达:

\begin{lstlisting}[language=Haskell]
  ghci> return (0,0) >>= landLeft 1 >> Nothing >>= landRight 1
  Nothing
\end{lstlisting}

注意\acode{Maybe}对\acode{>>=}的实现,它其实就是在遇到\acode{Nothing}时返回\acode{Nothing},遇到\acode{Just}值时
继续用\acode{Just}传值。

\subsection*{do 表示法}

Haskell 中的单子非常的有用,它们得到了属于自己的特殊语法,即\acode{do}表示法。我们已经学习到了\acode{do}标记法用作 I/O,
同时将若干 I/O actions 粘合成为一个。实际上\acode{do}表示法不仅仅作用于\acode{IO},而是可以用于任意单子。其原则仍然不变:
顺序的粘合其他的 monadic 值。现在让我们看看\acode{do}表示法是如何工作的,以及为什么这么有用。

考虑以下 monadic 应用的例子:

\begin{lstlisting}[language=Haskell]
  ghci> Just 3 >>= (\x -> Just (show x ++ "!"))
  Just "3!"
\end{lstlisting}

那么如果还有另一个\acode{>>=}在函数内呢?

\begin{lstlisting}[language=Haskell]
  ghci> Just 3 >>= (\x -> Just "!" >>= (\y -> Just (show x ++ y)))
  Just "3!"
\end{lstlisting}

一个嵌套的\acode{>>=}!在最外层的 lambda 函数,将\acode{Just "!"}喂给 lambda \acode{\\y -> Just (show x ++ y)}。
在内部的 lambda,\acode{y}变成\acode{"!"}。\acode{x}仍然是\acode{3}因为是从外层的 lambda 取值的。这些行为让我们想到了
下列式子:

\begin{lstlisting}[language=Haskell]
  ghci> let x = 3; y = "!" in show x ++ y
  "3!"
\end{lstlisting}

差别在于前者的值是 monadic,带有可能失败的 context。我们可以把其中任何一步替换成失败的状态:

\begin{lstlisting}[language=Haskell]
  ghci> Nothing >>= (\x -> Just "!" >>= (\y -> Just (show x ++ y)))
  Nothing
  ghci> Just 3 >>= (\x -> Nothing >>= (\y -> Just (show x ++ y)))
  Nothing
  ghci> Just 3 >>= (\x -> Just "!" >>= (\y -> Nothing))
  Nothing
\end{lstlisting}

为了解释的更清楚,我们来编写一个自己的\acode{Maybe}脚本:

\begin{lstlisting}[language=Haskell]
  foo :: Maybe String
  foo = Just 3 >>= (\x -> Just "!" >>= (\y -> Just (show x ++ y)))
\end{lstlisting}

为了避免这些麻烦的 lambda,Haskell 允许我们使用\acode{do}表示法:

\begin{lstlisting}[language=Haskell]
  foo :: Maybe String
  foo = do
    x <- Just 3
    y <- Just "!"
    Just (show x ++ y)
\end{lstlisting}

看起来好像是不必每一步都去检查\acode{Maybe}值是\acode{Just}或\acode{Nothing}。如果任意步骤取出了\acode{Nothing},
那么整个\acode{do}的结果就会是\acode{Nothing}。我们把所有责任都交给\acode{>>=},它来处理所有的\acode{context}问题。
这里的\acode{do}表示法就是另一种语法的形式来串联所有的 monadic 值。

在\acode{do}表达式中,每一行都是一个 monadic 值。想要获取结果,需要\acode{<-}。如果有一个\acode{Maybe String},
同时我们通过\acode{<-}绑定它到一个变量上,那么该变量将变为一个\acode{String},就像是我们使用\acode{>>=}将 monadic
值喂给 lambdas 那样。\acode{do}表达式中,最后一个 monadic 值,例如这里的\acode{Just (show x ++ y)},不可以将它
绑定至一个结果,因为这样的写法转换成\acode{>>=}的结果时不合理。它必须是所有 monadic 值粘合后最终的结果,因此需要考虑前面
所有可能失败的情景。

例如以下:

\begin{lstlisting}[language=Haskell]
  ghci> Just 9 >>= (\x -> Just (x > 8))
  Just True
\end{lstlisting}

因为\acode{>>=}的左参是一个\acode{Just}值,lambda 被应用至\acode{9},同时返回是一个\acode{Just True}。如果将上述
重写为\acode{do}表示法,可得:

\begin{lstlisting}[language=Haskell]
  marySue :: Maybe Bool
  marySue = do
      x <- Just 9
      Just (x > 8)
\end{lstlisting}

比较这两种写法,很容易看出来为什么整个 monadic 值的结果会是在\acode{do}表示法中最后一个,因为它串联了前面所有的结果。

用\acode{do}表示法来改写 routine 函数:

\begin{lstlisting}[language=Haskell]
  routine :: Maybe Pole
  routine = do
    start <- return (0,0)
    first <- landLeft 2 start
    second <- landRight 2 first
    landLeft 1 second
\end{lstlisting}

测试:

\begin{lstlisting}[language=Haskell]
  ghci> routine
  Just (3,2)
\end{lstlisting}

中间插入香蕉皮:

\begin{lstlisting}[language=Haskell]
  routine :: Maybe Pole
  routine = do
      start <- return (0,0)
      first <- landLeft 2 start
      Nothing
      second <- landRight 2 first
      landLeft 1 second
\end{lstlisting}

在\acode{do}表示法中没有通过\acode{<-}绑定 monadic 值时,就像是将\acode{>>}放在 monadic 值之后,即忽略计算的结果。
我们只是要让它们有序,而不是要它们的结果,而且这比写\acode{_ <- Nothing}要好看。

何时使用\acode{do}表示法或是\acode{>>=}取决于你。

在\acode{do}表示法中,我们还可以用模式匹配来绑定 monadic 值,就好像我们在\acode{let}表达式以及函数参数那样:

\begin{lstlisting}[language=Haskell]
  justH :: Maybe Char
  justH = do
    (x : xs) <- Just "hello"
    return x
\end{lstlisting}

如果这个模式匹配失败了呢?函数中的模式匹配失败后会去匹配下一个模式,如果所有的模式都匹配不上,那么错误将被抛出,然后程序就
挂掉了。另一方面,如果在\acode{let}中进行模式匹配失败会直接抛出错误。毕竟在\acode{let}表达式的情况下没有失败就跳至下一个
选项的设计。至于在\acode{do}表示法中模式匹配失败时,就会调用\acode{fail}函数。它是\acode{Monad} typeclass 中的一部分,
它运行在现在的 context 下失败时不会挂掉程序。它的默认实现如下:

\begin{lstlisting}[language=Haskell]
  fail :: (Monad m) => String -> m a
  fail msg = error msg
\end{lstlisting}

默认情况下确实是让程序崩溃,不过一些单子(例如\acode{Maybe})表示可能失败的 context 时,通常会实现自己的失败函数。例如
\acode{Maybe}的实现如下:

\begin{lstlisting}[language=Haskell]
  fail _ = Nothing
\end{lstlisting}

它忽视所有错误并创造一个\acode{Nothing}。

\begin{lstlisting}[language=Haskell]
  justH :: Maybe Char
  justH = do
    (x : _) <- Just "hello";
    return x
\end{lstlisting}

这里模式匹配失败,产生的影响等同于一个\acode{Nothing}。测试:

\begin{lstlisting}[language=Haskell]
  ghci> wopwop
  Nothing
\end{lstlisting}

这样模式匹配的失败只会限制在单子的 context 中,而不会让程序崩溃,这样的处理方式好很多。

\subsection*{列表单子}

% TODO

% \begin{lstlisting}[language=Haskell]

% \end{lstlisting}

\end{document}
