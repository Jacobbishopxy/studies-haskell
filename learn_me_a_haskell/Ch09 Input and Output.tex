\documentclass[./main.tex]{subfiles}

\begin{document}

\subsection*{Hello, world!}

在文件编辑器中敲下:

\begin{lstlisting}[language=Haskell]
  main = putStrLn "hello, world"
\end{lstlisting}

我们定义了一个\acode{main},其调用了一个名为\acode{putStrLn}的函数,其参数为\acode{"hello, world"}。将其保存为\acode{helloworld.hs}。
打开终端并 cd 到该文件处,输入命令:

\begin{lstlisting}
  $ ghc --make helloworld
  [1 of 1] Compiling Main             ( helloworld.hs, helloworld.o )
  Linking helloworld ...
\end{lstlisting}

执行:

\begin{lstlisting}
  $ ./helloworld
  hello, world
\end{lstlisting}

我们第一个编译的程序好了!

测试一下我们所写的。首先是查看\acode{putStrLn}的类型:

\begin{lstlisting}[language=Haskell]
  ghci> :t putStrLn
  putStrLn :: String -> IO ()
  ghci> :t putStrLn "hello, world"
  putStrLn "hello, world" :: IO ()
\end{lstlisting}

我们可以这样解读\acode{putStrLn}:\acode{putStrLn}接受一个字符串并返回一个\textbf{I/O action},其返回值的类型是\acode{()}(也就是
空元组,或是 unit 形态)。一个 I/O action 就是在执行时会造成副作用的动作,常常指读取输入或输出到屏幕,同时也代表返回某些值。

那么 I/O action 会在什么时候触发呢?这就是\acode{main}的作用。一个 I/O action 会在我们把它绑定到 \acode{main} 这个名字并执行程序的
时候触发。

把整个程序限制在只能有一个 I/O action 看起来有限制,这就是为什么需要\textit{do}语法将所有 I/O action 粘合成一个。例如:

\begin{lstlisting}[language=Haskell]
  main = do
    putStrLn "Hello, what's your name?"
    name <- getLine
    putStrLn ("Hey " ++ name ++ ", you rock!")
\end{lstlisting}

有趣,新的语法!这读起来非常像是命令式编程。注意我们写了\textit{do}之后接着一连串指令,就像是在写命令式编程那样。

正因如此,\acode{main}的类型签名总会是\acode{main :: IO something},这里的\acode{something}就是写具体类型。

我们之前没有遇到过的情况是第三行,即\acode{name <- getLine}。它看起来像是从输入中读取一行,并存储至变量\acode{name}中。让我们测试一下
\acode{getLine}的类型:

\begin{lstlisting}[language=Haskell]
  ghci> :t getLine
  getLine :: IO String
\end{lstlisting}

好的,\acode{getLine}是一个 I/O action,其包含了\acode{String}返回值类型。这很合理,因为它将等待用户在终端的输入,然后将输入转换为一个
字符串。那\acode{name <- getLine}是什么呢?你可以将这段代码读作:\textbf{执行一个 I/O action,\acode{getLine}将其结果绑定到
  \acode{name}这个名字}。\acode{getLine}的类型是\acode{IO String},因此\acode{name}的类型也就是\acode{String}。当我们从 I/O
action 中拿取数据时,就一定同时要在另一个 I/O action 中。这就是 Haskell 如何漂亮的区分 pure 和 impure 程序的方法。\acode{getLine}
在这样的意义下是 impure 的,因为执行两次的时候它没法保证会返回一样的值,这就是为什么它需要在一个 \acode{IO} 的类型构建函数中,这样才能在
I/O action 中取出数据。任何一段程序一旦依赖 I/O 数据,那么则会被视为 I/O 代码(原文用 tainted)。

当提到污染,并不是说 I/O action 所提供的返回不能在 pure 代码中使用。只要我们绑定它到一个名字,我们便可以暂时使用它。也就是说
\acode{name <- getLine}的\acode{name}就是一个普通的字符串。

让我们看以下代码是否合法:

\begin{lstlisting}[language=Haskell]
  nameTag = "Hello, my name is " ++ getLine
\end{lstlisting}

该代码不合法是因为\acode{++}需要两个参数都是同样参数类型的列表,左参数为\acode{String}(或\acode{[Char]}),而\acode{getLine}则是
\acode{IO String}。显然不能将两者合并。要从\acode{IO String}中获取值只能在 I/O action 内部做\acode{name <- getLine}。换言之,
如果要处理 impure 数据,那么就需要再 impure 环境中做。

每个 I/O action 执行都有一个封装后的返回。那么之前的代码同样也可以这么写:

\begin{lstlisting}[language=Haskell]
  main = do
    foo <- putStrLn "Hello, what's your name?"
    name <- getLine
    putStrLn ("Hey " ++ name ++ ", you rock!")
\end{lstlisting}

然而\acode{foo}只会得到一个\acode{()}值,因此这么做并无实际意义。注意最后的\acode{putStrLn}并没有绑定至任何事物。这是因为在一个
\textit{do}代码块中,\acode{最后的 action 不能绑定至一个名称},如前面两行那样。我们在之后讲解 Monad 的时候会说明为什么。

除了最后一行,\textit{do}代码块中的每一行皆可进行绑定。因此\acode{putStrLn "BLAH"}可以被写成\acode{_ <- putStrLn "BLAH"},不过
这并没大用,因此不如不写。

初学者可能会这么想:

\begin{lstlisting}[language=Haskell]
  name = getLine
\end{lstlisting}

记住,从 I/O action 中取值,必须将其置于其它 I/O action 内,并使用\acode{<-}将其结果绑定至一个名称。

我们能够在\textit{do}代码块中使用\textit{let}绑定,就如在列表表达式中那样:

\begin{lstlisting}[language=Haskell]
  import Data.Char

  main = do
    putStrLn "What's your first name?"
    firstName <- getLine
    putStrLn "What's your last name?"
    lastName <- getLine

    let bigFirstName = map toUpper firstName
        bigLastName = map toUpper lastName

    putStrLn $ "hey " ++ bigFirstName ++ " " ++ bigLastName ++ ", how are you?"
\end{lstlisting}

注意在\acode{do}代码块中的 I/O actions,同时注意\acode{let}与其名字,缩进在 Haskell 中不会被无视。

接下来是一个持续读取输入行,并在同一行打印翻转后单词的程序。程序会在输入空行后结束:

\begin{lstlisting}[language=Haskell]
  main = do
  line <- getLine
  if null line
    then return ()
    else do
      putStrLn $ reverseWords line
      main

reverseWords :: String -> String
reverseWords = unwords . map reverse . words
\end{lstlisting}

首先我们来看一下\acode{reverseWords},这个普通的函数假如接受了个字符串\acode{"hey there man"},那么先调用\acode{words}来参数
一个字的列表,比如\acode{["hey", "there", "man"]}。接着将\acode{reverse}映射至列表,获得\acode{["yeh", "ereht", "nam"]},
接着再通过\acode{unwords}将列表转换为一个字符串\acode{"yeh ereht nam"}。这里用到了函数组合,如果没有函数组合的话,我们需要这样做
\acode{reverseWords st = unwords (map reverse (words st))}。

我们来看一下在\textit{else}中发生了什么事。由于我们必须要一个 I/O action 在\textit{else}之后,我们使用一个\textit{do}代码块
将两个 I/O action 整合成一个。我们也可以这么写:

\begin{lstlisting}[language=Haskell]
  else (do
    putStrLn $ reverseWords line
    main)
\end{lstlisting}

这样可以更显式的将\textit{do}代码块认做是一个 I/O action,不过就是更丑了点。在\textit{do}代码块中,调用\acode{reverseWords}从
\acode{getLine}而来的行,接着输出至终端。之后就是递归,因为\acode{main}自身就是一个 I/O action。也就意味着又回到了程序一开头。

那么当\acode{null line}为真时呢?我们看到\acode{then return ()}。\textbf{在 Haskell 中的\acode{return}与其他语言的
  \acode{return}完全不同!}它们只是有相同的名字,这会迷惑很多人,但是实际上大相径庭。在命令式语言中,\acode{return}通常结束一个
方法的执行,并将结果返回给调用者。在 Haskell(特别是 I/O action 中),则是利用某个 pure 值制造出 I/O action。所以在 I/O 的情况下,
\acode{return "haha"}的类型是\acode{IO String}。将 pure 值包成 I/O action 有什么意义呢?这是因为一定要 I/O action 来
承载空输入行的情况。因此使用\acode{return ()}做了一个没什么用的 I/O action。

在 I/O \textit{do}代码块中放一个\acode{return}并不会结束执行。下面这个程序就会执行到底:

\begin{lstlisting}[language=Haskell]
  main = do
    return ()
    return "HAHAHA"
    line <- getLine
    return "BLAH BLAH BLAH"
    return 4
    putStrLn line
\end{lstlisting}

所有的这些\acode{return}都用于创造 I/O actions,而其并不做任何事情仅仅是封装一个结构,且该结果被抛弃,因为它们并没有绑定到一个
名字上。\acode{return}可以与\acode{<-}组合在一起用于绑定名称:

\begin{lstlisting}[language=Haskell]
  main = do
    a <- return "hell"
    b <- return "yeah!"
    putStrLn $ a ++ " " ++ b
\end{lstlisting}

可以看到,\acode{return}像是相反的\acode{<-},前者接受一个值,将其放入盒中,后者接受一个盒(并执行它)然后将值取出来,绑定至一个
名称。不过这么做有点多余,特别是在\textit{do}代码块中,可以使用\textit{let}绑定名称:

\begin{lstlisting}[language=Haskell]
  main = do
    let a = "hell"
        b = "yeah"
    putStrLn $ a ++ " " ++ b
\end{lstlisting}

在 I/O \textit{do}代码块中需要\acode{return}的原因有两个:一个是需要一个什么事都不做的 I/O action,或者是不希望这个\textit{do}
代码块形成的 I/O action 的结果值是这个代码块中的最后一个 I/O action。我们希望有一个不同的结果值,所以用\acode{return}来做一个
I/O action 包装想要的结果放在\textit{do}代码块的最后。

在讲下一节的文件之前,让我们看看有哪些实用的函数可以处理 I/O。

\acode{putStr}类似于\acode{putStrLn},前者不会换行:

\begin{lstlisting}[language=Haskell]
  main = do   putStr "Hey, "
              putStr "I'm "
              putStrLn "Andy!"
\end{lstlisting}

\begin{lstlisting}[language=Haskell]
  $ runhaskell putstr_test.hs
  Hey, I'm Andy!
\end{lstlisting}

\acode{putChar}接受一个字符:

\begin{lstlisting}[language=Haskell]
  main = do   putChar 't'
              putChar 'e'
              putChar 'h'
\end{lstlisting}

\begin{lstlisting}[language=Haskell]
  $ runhaskell putchar_test.hs
  teh
\end{lstlisting}

\acode{print}接受任意实现了\acode{Show}的值(意味着知道如何将该值转换为一个字符串):

\begin{lstlisting}[language=Haskell]
  main = do   print True
              print 2
              print "haha"
              print 3.2
              print [3,4,3]
\end{lstlisting}

\begin{lstlisting}[language=Haskell]
  $ runhaskell print_test.hs
  True
  2
  "haha"
  3.2
  [3,4,3]
\end{lstlisting}

\acode{getChar}从输入中读取一个字符:

\begin{lstlisting}[language=Haskell]
  main = do
    c <- getChar
    if c /= ' '
        then do
            putChar c
            main
        else return ()
\end{lstlisting}

\begin{lstlisting}[language=Haskell]
  $ runhaskell getchar_test.hs
  hello sir
  hello
\end{lstlisting}

\acode{when}函数可以在\acode{Control.Monad}中找到(通过\acode{import Control.Monad})。它在\textit{do}代码块中很有意思,
它像是一个流控声明,但其实它是一个普通函数。其接受一个布尔值以及一个 I/O action,如果改布尔值为真,返回同样的 I/O action;如果为假,
则返回\acode{return ()},即什么都不做的 I/O action。下面是使用\acode{when}来改写之前的程序:

\begin{lstlisting}[language=Haskell]
  import Control.Monad

  main = do
      c <- getChar
      when (c /= ' ') $ do
          putChar c
          main
\end{lstlisting}

如你所见,它将\acode{if something then do some I/O action else return ()}这样的模式封装了起来。

\acode{sequence}接受一个列表的 I/O action 并返回一个 I/O action。其类型签名\acode{sequence :: [IO a] -> IO [a]}。例如:

\begin{lstlisting}[language=Haskell]
  main = do
    a <- getLine
    b <- getLine
    c <- getLine
    print [a,b,c]
\end{lstlisting}

等同于:

\begin{lstlisting}[language=Haskell]
  main = do
    rs <- sequence [getLine, getLine, getLine]
    print rs
\end{lstlisting}

由于对一个队列映射一个返回 I/O action 的函数,再\acode{sequence}这个动作太常用了。所以\acode{mapM}与\acode{mapM_}被引入了。
前者接受一个函数与一个列表,映射函数值列表接着再进行 sequences;后者一样,只不过它丢弃返回值。我们不关心 sequenced I/O action 的
返回值时就使用后者。

\begin{lstlisting}[language=Haskell]
  ghci> mapM print [1,2,3]
  1
  2
  3
  [(),(),()]
  ghci> mapM_ print [1,2,3]
  1
  2
  3
\end{lstlisting}

\acode{forever}接受一个 I/O action 返回一个 I/O action,无限循环:

\begin{lstlisting}[language=Haskell]
  import Control.Monad
  import Data.Char

  main = forever $ do
      putStr "Give me some input: "
      l <- getLine
      putStrLn $ map toUpper l
\end{lstlisting}

\acode{forM}类似于\acode{mapM},不同之处在于参数调转了。为什么有用呢?一些创造性的 lambdas 与\textit{do}的使用,让我们可以
这样做:

\begin{lstlisting}[language=Haskell]
  import Control.Monad

  main = do
      colors <- forM [1,2,3,4] (\a -> do
          putStrLn $ "Which color do you associate with the number " ++ show a ++ "?"
          color <- getLine
          return color)
      putStrLn "The colors that you associate with 1, 2, 3 and 4 are: "
      mapM putStrLn colors
\end{lstlisting}

\begin{lstlisting}[language=Haskell]
  $ runhaskell form_test.hs
  Which color do you associate with the number 1?
  white
  Which color do you associate with the number 2?
  blue
  Which color do you associate with the number 3?
  red
  Which color do you associate with the number 4?
  orange
  The colors that you associate with 1, 2, 3 and 4 are:
  white
  blue
  red
  orange
\end{lstlisting}

\subsection*{文件与流}

% TODO

\subsection*{命令行参数}

% TODO

\subsection*{随机}

% TODO

\subsection*{字节串}

% TODO

\subsection*{异常}

% TODO

\end{document}
