\documentclass[./main.tex]{subfiles}

\begin{document}

\subsection*{相信类型}

通过 \acode{:t} 命令可以得知类型:

\begin{lstlisting}[language=Haskell]
  ghci> :t 'a'
  'a' :: Char
  ghci> :t True
  True :: Bool
  ghci> :t "HELLO!"
  "HELLO!" :: [Char]
  ghci> :t (True, 'a')
  (True, 'a') :: (Bool, Char)
  ghci> :t 4 == 5
  4 == 5 :: Bool
\end{lstlisting}

函数同样拥有类型:

\begin{lstlisting}[language=Haskell]
  ghci> :t doubleSmallNumber
  doubleSmallNumber :: (Ord a, Num a) => a -> a
\end{lstlisting}

而对于有多个入参的函数而言:

\begin{lstlisting}[language=Haskell]
  ghci> addThree x y z = x + y + z
  ghci> :t addThree
  addThree :: Num a => a -> a -> a -> a
\end{lstlisting}

参数由 \acode{->} 符分开,并且入参与返回的类型并无差异,之后我们讨论到为什么是由 \acode{->} 分割而不是
\acode{Int, Int, Int -> Int} 或者其他样式的类型。

接下来是一些常规的类型:

\acode{Int}:对于 32 位的机器而言最大值大概是 $2147483647$ 而最小值则是 $2147483647$。

\acode{Integer}:同样也是整数,只不过范围会大很多,而 \acode{Int} 则更高效率。

\acode{Float}:单精度。

\acode{Double}:双精度。

\acode{Bool}:布尔值。

\acode{Char}:字符。

\subsection*{类型变量}

那么 \acode{head} 函数的类型是什么呢?

\begin{lstlisting}[language=Haskell]
  ghci> :t head
  head :: [a] -> a
\end{lstlisting}

这里的 \acode{a} 则是一个 \textbf{类型变量 type variable},这意味着 \acode{a} 可以是任意类型。这非常像其他语言的泛型,
但唯有在 Haskell 中它更为强大,因为它允许我们可以轻易的编写通用的函数,且不使用任何特定的行为的类型。带有类型变量的函数也被
称为 \textbf{多态函数 polymorphic functions}。

\subsection*{Typeclasses 101}

typeclass 类似于一个接口用于定义一些行为。如果一个类型是 typeclass 的一部分,这就意味着它支持并且实现了 typeclass 中所
描述的行为。

那么 \acode{==} 函数的类型签名是什么呢?

\begin{lstlisting}[language=Haskell]
  ghci> :t (==)
  (==) :: Eq a => a -> a -> Bool
\end{lstlisting}

\begin{anote}
  \acode{==} 操作符是一个函数,\acode{+},\acode{*},\acode{-},\acode{/} 以及其它的操作符也都是。如果一个函数只包含
  特殊字符,那么默认情况下它被认做是一个中置函数。如果想要检查它的类型,将其传递给另一个函数或作为前缀函数调用它,那么则需要用
  括号将其包围。
\end{anote}

这里有趣的是 \acode{=>} 符号,在该符号之前的被称为一个 \textbf{类约束 class constraint}。那么上述的类型声明可以被这么
理解:等式函数接受任意两个相同类型的值,并返回一个 \acode{Bool},而这两个值必须是 \acode{Eq} 类的成员(即类约束)。

\acode{Eq} typeclass 提供了一个用于测试是否相等的接口。

而 \acode{elem} 函数则拥有 \acode{(Eq a) => a -> [a] -> Bool} 这样的类型,因为其在数组中使用了 \acode{==} 用于检查
元素是否为期望的值。

一些基础的 typeclasses:

\acode{Eq} 如上所述。

\acode{Ord} 覆盖了所有标准的比较函数例如\acode{>},\acode{<},\acode{>=} 以及 \acode{<=}。\acode{compare} 函数接受
两个同类型的 \acode{Ord} 成员,并返回一个 ordering。\acode{Ordering} 是一个可作为 \acode{GT},\acode{LT} 或是 \acode{EQ}
的类型,分别意为 \textit{大于},\textit{小于}以及\textit{等于}。

\acode{Show} 可以表示为字符串。

\acode{Read} 有点类似于 \acode{Show} 相反的 typeclass,\acode{read} 函数接受一个字符串并返回一个 \acode{Read} 的成员。

\begin{lstlisting}[language=Haskell]
  ghci> read "True" || False
  True
  ghci> read "8.2" + 3.8
  12.0
  ghci> read "5" - 2
  3
  ghci> read "[1,2,3,4]" ++ [3]
  [1,2,3,4,3]
\end{lstlisting}

但是如果尝试一下 \acode{read "4"} 呢?

\begin{lstlisting}[language=Haskell]
  ghci> read "4"
  <interactive>:1:0:
      Ambiguous type variable `a' in the constraint:
        `Read a' arising from a use of `read' at <interactive>:1:0-7
      Probable fix: add a type signature that fixes these type variable(s)
\end{lstlisting}

这里 GHCI 告知它并不知道想要返回什么,通过检查 \acode{read} 的类型签名:

\begin{lstlisting}[language=Haskell]
  ghci> :t read
  read :: Read a => String -> a
\end{lstlisting}

也就是说其返回的是 \acode{Read} 所约束的类型,那么如果在之后没有使用到它,则没有办法知晓其类型。我们可以显式的使用\textbf{类型注解},
即在表达式后面加上 \acode{::} 与指定的一个类型:

\begin{lstlisting}[language=Haskell]
  ghci> read "5" :: Int
  5
  ghci> read "5" :: Float
  5.0
  ghci> (read "5" :: Float) * 4
  20.0
  ghci> read "[1,2,3,4]" :: [Int]
  [1,2,3,4]
  ghci> read "(3, 'a')" :: (Int, Char)
  (3, 'a')
\end{lstlisting}

大多数表达式可以被编译器推导出其类型,但是有时编译器并不知道返回值的类型,例如 \acode{read "5"} 时该为 \acode{Int} 还是 \acode{Float}。
那么为了知道其类型,Haskell 会解析 \acode{read "5"}。然而 Haskell 是一个静态类型语言,因此它需要在代码编译前知道所有类型。

\acode{Enum} 成员是序列化的有序类型 -- 它们可被枚举。\acode{Enum} typeclass 的主要优势是可以使用在列表区间;它们也同样定义了
successors 与 predecessors,即可使用 \acode{succ} 以及 \acode{pred} 函数。在这个类中的类型有:\acode{()},\acode{Bool}
\acode{Char},\acode{Ordering},\acode{Int},\acode{Integer},\acode{Float} 以及 \acode{Double}.

\acode{Bounded} 成员拥有上下界:

\begin{lstlisting}[language=Haskell]
  ghci> minBound :: Int
  -2147483648
  ghci> maxBound :: Char
  '\1114111'
  ghci> maxBound :: Bool
  True
  ghci> minBound :: Bool
  False
\end{lstlisting}

元组的成员如果为 \acode{Bounded} 的一部分,那么元组也是:

\begin{lstlisting}[language=Haskell]
  ghci> maxBound :: (Bool, Int, Char)
  (True,2147483647,'\1114111')
\end{lstlisting}

\acode{Num} 是一个数值类:

\begin{lstlisting}[language=Haskell]
  ghci> :t 20
  20 :: (Num t) => t
  ghci> 20 :: Int
  20
  ghci> 20 :: Integer
  20
  ghci> 20 :: Float
  20.0
  ghci> 20 :: Double
  20.0
\end{lstlisting}

这些类型都在 \acode{Num} typeclass 内。如果检查 \acode{*} 的类型,将会看到:

\begin{lstlisting}[language=Haskell]
  ghci> :t (*)
  (*) :: (Num a) => a -> a -> a
\end{lstlisting}

\acode{Integral} 同样也是数值 typeclass。

\acode{Floating} 包含 \acode{Float} 与 \acode{Double}。

\acode{fromIntegral} 是一个处理数值的常用函数,而其类型为 \acode{fromIntegral :: (Num b, Integral a) => a -> b}。

\end{document}
