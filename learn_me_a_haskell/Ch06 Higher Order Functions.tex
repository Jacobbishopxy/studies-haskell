\documentclass[./main.tex]{subfiles}

\begin{document}

\subsection*{柯里化函数}

在 Haskell 中每个函数实质上仅接受一个参数。那么迄今为止定义的那么多函数是怎么接受多个参数的呢?这是就是\textbf{柯里化函数 curried functions}。

\begin{lstlisting}[language=Haskell]
  ghci> max 4 5
  5
  ghci> (max 4) 5
  5
\end{lstlisting}

两个参数间用空格间隔就是简单的\textbf{函数应用 function application}。空格类似于一个操作符,其拥有最高的优先级。例如\acode{max},其签名为
\acode{max :: (Ord a) => a -> a -> a},可以被重写为\acode{max :: (Ord a) => a -> (a -> a)},可以这么理解:\acode{max}接受一个
\acode{a}并返回(即\acode{->})一个函数,该函数接受一个\acode{a}并返回一个\acode{a}。这就是为什么返回值类型以及函数的参数都是由箭头符
进行分隔的。

那么这样做有什么便利?简单来说如果调用一个仅几个参数的函数,我们得到的是一个\textbf{部分应用 partially applied}的函数,即一个函数接受的参数
与留下未填的参数一样多。

来观测一个简单的函数:

\begin{lstlisting}[language=Haskell]
  multThree :: (Num a) => a -> a -> a -> a
  multThree x y z = x * y * z
\end{lstlisting}

当使用\acode{multThree 3 5 9}或者\acode{((multThree 3) 5) 9}时到底发生了什么?首先,\acode{3}应用至\acode{multThree},因为它们由空格
进行了分隔(最高优先级)。这就创建了一个接受一个参数的函数,并返回了一个函数。接下来\acode{5}被应用至该函数,以此类推。记住我们的函数类型同样也
可以重写成\acode{multThree :: (Num a) => a -> (a -> (a -> a))}。接下来观察:

\begin{lstlisting}[language=Haskell]
  ghci> let multTwoWithNine = multThree 9
  ghci> multTwoWithNine 2 3
  54
  ghci> let multWithEighteen = multTwoWithNine 2
  ghci> multWithEighteen 10
  180
\end{lstlisting}

调用函数时输入不足的参数,实际上实在创造新的函数。那么如果希望创建一个函数接受一个值并将其与\acode{100}进行比较呢?

\begin{lstlisting}[language=Haskell]
  compareWithHundred :: (Num a, Ord a) => a -> Ordering
  compareWithHundred x = compare 100 x
\end{lstlisting}

如果带着\acode{99}调用它,返回一个\acode{GT}。注意\acode{x}同时位于等式的右侧。那么调用\acode{compare 100}返回的是什么呢?它返回一个接受
一个数值参数并将其与\acode{100}进行比较的函数。现在将其重写:

\begin{lstlisting}[language=Haskell]
  compareWithHundred :: (Num a, Ord a) => a -> Ordering
  compareWithHundred = compare 100
\end{lstlisting}

类型声明仍然相同,因为\acode{compare 100}返回一个函数。\acode{compare}的类型是\acode{(Ord a) -> a -> (a -> Ordering)},带着
\acode{100}调用它返回一个\acode{(Num a, Ord a) => a -> Ordering}。这里额外的类约束溜走了,这是因为\acode{100}同样也是\acode{Num}
类的一部分。

中置函数同样可以通过使用分割被部分应用。要分割中置函数,只需将其用圆括号括起来,并只在一侧提供参数:

\begin{lstlisting}[language=Haskell]
  divideByTen :: (Floating a) => a -> a
  divideByTen = (/10)
\end{lstlisting}

调用\acode{divideByTen 200}等同于\acode{200 / 10},等同于\acode{(/10) 200}。

那么如果在 GHCI 中尝试\acode{multThree 3 4}而不是通过\textit{let}将其与名称绑定,或是将其传递至另一个函数呢?

\begin{lstlisting}[language=Haskell]
  ghci> multThree 3 4
  <interactive>:1:0:
      No instance for (Show (t -> t))
        arising from a use of `print' at <interactive>:1:0-12
      Possible fix: add an instance declaration for (Show (t -> t))
      In the expression: print it
      In a 'do' expression: print it
\end{lstlisting}

GHCI 会提示我们表达式生成了一个类型为\acode{a -> a}的函数,但是并不知道该如何将其打印至屏幕。函数并不是\acode{Show} typeclass 的实例,
因此我们并不会得到一个函数的展示。

\subsection*{来点高阶函数}

函数可以接受函数作为其参数,也可以返回函数。

\begin{lstlisting}[language=Haskell]
  applyTwice :: (a -> a) -> a -> a
  applyTwice f x = f (f x)
\end{lstlisting}

首先注意的是类型声明。之前我们是不需要圆括号的,因为\acode{->}是自然地右结合。然而在这里却是强制性的,它们表明了第一个参数是一个接受某物并
返回某物的函数,第二个参数同上所述。我们可以用柯里化函数的方式来进行解读,不过为了避免头疼,我们仅需要说该函数接受两个参数并返回一个值。
这里第一个参数是一个函数(即类型\acode{a -> a}),而第二个参数则是\acode{a}。

函数体非常的简单,仅需要使用参数\acode{f}作为一个函数,通过一个空格将\acode{x}应用至其,接着再应用一次\acode{f}。

\begin{lstlisting}[language=Haskell]
  ghci> applyTwice (+3) 10
  16
  ghci> applyTwice (++ " HAHA") "HEY"
  "HEY HAHA HAHA"
  ghci> applyTwice ("HAHA " ++) "HEY"
  "HAHA HAHA HEY"
  ghci> applyTwice (multThree 2 2) 9
  144
  ghci> applyTwice (3:) [1]
  [3,3,1]
\end{lstlisting}

可以看到单个高阶函数可以被用以多种用途。而在命令式编程中,通常使用的是 for 循环、while 循环、将某物设置为一个变量、检查其状态等等,为了达到
某些行为,还需要用接口将其封装,类似于函数;而函数式编程则使用高阶函数来抽象出相同的模式。

现在让我们实现一个名为\acode{flip}的标准库已经存在的函数,其接受一个函数并返回一个类似于原来函数的函数,仅前两个参数被翻转。简单的实现:

\begin{lstlisting}[language=Haskell]
  filp' :: (a -> b -> c) -> (b -> a -> c)
  filp' f = g
    where
      g x y = f y x
\end{lstlisting}

观察类型声明,\acode{flip'}接受一个函数,该函数接受一个\acode{a}与\acode{b},并返回一个函数,该返回的函数接受一个\acode{b}与\acode{a}。
然而默认情况下函数是柯里化的,第二个圆括号是没有必要的,因为\acode{->}默认是右结合的。\acode{(a -> b -> c) -> (b -> a -> c)}等同于
\acode{(a -> b -> c) -> (b -> (a -> c))},等同于\acode{(a -> b -> c) -> b -> a -> c}。我们可以用更简单方式来定义该函数:

\begin{lstlisting}[language=Haskell]
  filp'' :: (a -> b -> c) -> b -> a -> c
  filp'' f y x = f x y
\end{lstlisting}

这里我们利用了函数都是柯里化的便利。当不带参数\acode{y}与\acode{x}时调用\acode{flip'' f}时,它将返回一个\acode{f},该函数接受两个参数,
只不过它们的位置是翻转的。

\begin{lstlisting}[language=Haskell]
  ghci> flip' zip [1,2,3,4,5] "hello"
  [('h',1),('e',2),('l',3),('l',4),('o',5)]
  ghci> zipWith (flip' div) [2,2..] [10,8,6,4,2]
  [5,4,3,2,1]
\end{lstlisting}

\subsection*{Maps \& filters}

\acode{map}接受一个函数以及一个列表,将该函数应用至列表中的每一个元素中,生产一个新的列表。让我们来看一下类型签名:

\begin{lstlisting}[language=Haskell]
  map :: (a -> b) -> [a] -> [b]
  map _ [] = []
  map f (x : xs) = f x : map f xs
\end{lstlisting}

测试:

\begin{lstlisting}[language=Haskell]
  ghci> map (+3) [1,3,5,1,6]
  [4,6,8,4,9]
  ghci> map (-1) [1,3,5,1,6]

  <interactive>:2:1: error:
      • Could not deduce (Num a0)
          arising from a type ambiguity check for
          the inferred type for ‘it’
        from the context: (Num a, Num (a -> b))
          bound by the inferred type for ‘it’:
                     forall {a} {b}. (Num a, Num (a -> b)) => [b]
          at <interactive>:2:1-20
        The type variable ‘a0’ is ambiguous
        These potential instances exist:
          instance Num Integer -- Defined in ‘GHC.Num’
          instance Num Double -- Defined in ‘GHC.Float’
          instance Num Float -- Defined in ‘GHC.Float’
          ...plus two others
          ...plus one instance involving out-of-scope types
          (use -fprint-potential-instances to see them all)
      • In the ambiguity check for the inferred type for ‘it’
        To defer the ambiguity check to use sites, enable AllowAmbiguousTypes
        When checking the inferred type
          it :: forall {a} {b}. (Num a, Num (a -> b)) => [b]
  ghci> map (subtract 1) [1,3,5,1,6]
  [0,2,4,0,5]
  ghci> map (++ "!") ["BIFF", "BANG", "POW"]
  ["BIFF!","BANG!","POW!"]
  ghci> map (replicate 3) [3..6]
  [[3,3,3],[4,4,4],[5,5,5],[6,6,6]]
  ghci> map (map (^2)) [[1,2], [3,4,5,6], [7,8]]
  [[1,4],[9,16,25,36],[49,64]]
  ghci> map fst [(1,2),(3,5),(6,3),(2,6),(2,5)]
  [1,3,6,2,2]
\end{lstlisting}

原书上一章有提到过\acode{-1}这样的情况,报错的原因是 Haskell 将\acode{-1}识别为负数而不是减法,需要显式调用\acode{subtract}
才能识别为 partial 函数并应用至列表中的各个元素上。

\acode{filter}接受一个子句(该子句是一个函数,用于告知某物是否为真),以及一个列表,并返回满足该子句的元素列表。类型签名如下:

\begin{lstlisting}[language=Haskell]
  filter :: (a -> Bool) -> [a] -> [a]
  filter _ [] = []
  filter p (x : xs)
    | p x = x : filter p xs
    | otherwise = filter p xs
\end{lstlisting}

测试:

\begin{lstlisting}[language=Haskell]
  ghci> filter (>3) [1,5,3,2,1,6,4,3,2,1]
  [5,6,4]
  ghci> filter (==3) [1,2,3,4,5]
  [3]
  ghci> filter even [1..10]
  [2,4,6,8,10]
  ghci> let notNull x = not (null x) in filter notNull [[1,2,3],[],[3,4,5],[2,2],[],[],[]]
  [[1,2,3],[3,4,5],[2,2]]
  ghci> filter (`elem` ['a'..'z']) "u LaUgH aT mE BeCaUsE I aM diFfeRent"
  "uagameasadifeent"
  ghci> filter (`elem` ['A'..'Z']) "i lauGh At You BecAuse u r aLL the Same"
  "GAYBALLS"
\end{lstlisting}

将上一章的\acode{quicksort}中的列表表达式替换为\acode{filter}:

\begin{lstlisting}[language=Haskell]
  quicksort :: (Ord a) => [a] -> [a]
  quicksort [] = []
  quicksort (x : xs) =
    let smallerSorted = quicksort (filter (<= x) xs)
        biggerSorted = quicksort (filter (> x) xs)
     in smallerSorted ++ [x] ++ biggerSorted
\end{lstlisting}

现在尝试一下\textbf{找到 100,100 以下最大能被 3829 的值}:

\begin{lstlisting}[language=Haskell]
  largestDivisible :: (Integral a) => a
  largestDivisible = head (filter p [100000, 99999 ..])
    where
      p x = x `mod` 3829 == 0
\end{lstlisting}

接下来尝试一下\textbf{找到所有奇数平方在 10,000 以下的和},不过首先要介绍一下\acode{takeWhile}函数。该函数接受一个子句以及一个列表,
接着从列表头向后遍历,在子句返回真时返回元素,一旦子句返回假则结束遍历。可以在 GHCI 上用一行来完成任务:

\begin{lstlisting}[language=Haskell]
  ghci> sum (takeWhile (<10000) (filter odd (map (^2) [1..])))
  166650
\end{lstlisting}

当然也可以用列表表达式:

\begin{lstlisting}[language=Haskell]
  ghci> sum (takeWhile (<10000) [n^2 | n <- [1..], odd (n^2)])
  166650
\end{lstlisting}

接下来一个问题是处理考拉兹猜想 Collatz sequences,其数学表达为:

\begin{equation*}
  f(n) =
  \begin{cases}
    n / 2  & \text{if $n \equiv 0$ ($\mod 2$)} \\
    3n + 1 & \text{if $n \equiv 1$ ($\mod 2$)} \\
  \end{cases}
\end{equation*}

我们希望知道的是:\textbf{对于从 1 至 100 的所有数开始,有多少链的长度是大于 15 的?}首先编写一个函数用于生成链:

\begin{lstlisting}[language=Haskell]
  chain :: (Integral a) => a -> [a]
  chain 1 = [1]
  chain n
    | even n = n : chain (n `div` 2)
    | odd n = n : chain (n * 3 + 1)
\end{lstlisting}

因为链的最后一位肯定是 1,也就是边界,那么这就是一个简单的递归函数了。测试:

\begin{lstlisting}[language=Haskell]
  ghci> chain 10
  [10,5,16,8,4,2,1]
  ghci> chain 1
  [1]
  ghci> chain 30
  [30,15,46,23,70,35,106,53,160,80,40,20,10,5,16,8,4,2,1]
\end{lstlisting}

看起来能正常工作,接下来就是获取长度:

\begin{lstlisting}[language=Haskell]
  numLongChains :: Int
  numLongChains = length (filter isLong (map chain [1 .. 100]))
    where
      isLong xs = length xs > 15
\end{lstlisting}

我们将\acode{chain}函数映射至\acode{[1..100]}来获取一个链的列表,接着根据检查长度是否超过 15 的子句来过滤它们、一旦完成过滤,我们就可以看到
结果的列表中还剩多少链。

\begin{anote}
  该函数的类型是\acode{numLongChains :: Int},因为历史原因\acode{length}返回一个\acode{Int}而不是一个\acode{Num a}。如果我们需要返回一个
  更通用的\acode{Num a},可以对返回的长度使用\acode{fromIntegral}。
\end{anote}

使用\acode{map},我们还可以这样做\acode{map (*) [0..]},如果不是因为别的原因要解释柯里化以及偏函数是实值,那么可以将其传递至其它函数,
或者置入列表中(仅仅不能将其变为字符串)。迄今为止,我们只映射了单参数的函数至列表,例如\acode{map (*2) [0..]}类型是\acode{(Num a) => [a]},
我们同样可以这么做\acode{map (*) [0..]}。这里将会对列表中的每个数值应用函数\acode{*},即类型为\acode{(Num a) => a -> a -> a}。将一个参数
应用于需要两个参数的函数将会返回需要一个参数的函数。如果将\acode{*}映射至列表\acode{[0..]},得到的则是一个接受单个参数的函数的列表,即
\acode{(Num a) => [a -> a]}。也就是说\acode{map (*) [0..]}生产一个这样的列表\acode{[(0*),(1*),(2*),(3*),(4*),(5*)..]}。

\begin{lstlisting}[language=Haskell]
  ghci> let listOfFuns = map (*) [0..]
  ghci> (listOfFuns !! 4) 5
  20
\end{lstlisting}

\subsection*{Lambdas}

构建 lambda 的方式是写一个\acode{\\},接着是由空格分隔的参数,再然后是\acode{->}符,最后是函数体。

对于\acode{numLongChains}函数而言,不再需要一个\acode{where}子句:

\begin{lstlisting}[language=Haskell]
  numLongChains' :: Int
  numLongChains' = length (filter (\xs -> length xs > 15) (map chain [1 .. 100]))
\end{lstlisting}

lambdas 也是表达式,上述代码中的表达式\acode{\\xs -> length xs > 15}返回的就是一个函数,其用作于告知列表的长度是否超过 15。

对于不熟悉柯里化以及偏函数应用的人而言,通常会在不必要的地方使用 lambdas。例如表达式\acode{map (+3) [1,6,3,2]}以及
\acode{map (\x -> x + 3) [1,6,3,2]}是等价的,因为\acode{(+3)}以及\acode{(\x -> x + 3)}皆为接受一个值并加上 3 的函数。

与普通函数一样,lambdas 可以接受任意数量的参数:

\begin{lstlisting}[language=Haskell]
  ghci> zipWith (\a b -> (a * 30 + 3) / b) [5,4,3,2,1] [1,2,3,4,5]
  [153.0,61.5,31.0,15.75,6.6]
\end{lstlisting}

与普通函数一样,也可以在 lambdas 中进行模式匹配。唯一不同点在于不能在一个参数内定义若干模式,例如对同样一个参数做\acode{[]}以及\acode{(x:xs)}
模式。如果一个模式匹配再 lambda 中失败了,一个运行时错误则会出现,所以需要特别注意在 lambdas 中进行模式匹配!

\begin{lstlisting}[language=Haskell]
  ghci> map (\(a,b) -> a + b) [(1,2),(3,5),(6,3),(2,6),(2,5)]
  [3,8,9,8,7]
\end{lstlisting}

lambdas 通常都被圆括号包围,除非我们想让它一直延伸到最右。有趣的来了:函数默认情况下是柯里化的,以下两者相等:

\begin{lstlisting}[language=Haskell]
  addThree :: (Num a) => a -> a -> a -> a
  addThree x y z = x + y + z
\end{lstlisting}

\begin{lstlisting}[language=Haskell]
  addThree :: (Num a) => a -> a -> a -> a
  addThree = \x -> \y -> \z -> x + y + z
\end{lstlisting}

\subsection*{fold}

一个 fold 接受一个二元函数,一个起始值(可以称其为 accumulator)以及一个需要被 fold 的列表。二元函数接受两个参数,第一个是 accumulator,
第二个则是列表中第一个(或最后一个)元素,然后再生产一个新的 accumulator。接着二元函数再次被调用,带着新的 accumulator 以及新的列表中第一个
(或最后一个)元素,以此类推。一旦遍历完整个列表,仅剩 accumulator 剩余,即最终答案。

首先让我们看一下\acode{foldl}函数,也被称为左折叠 left fold。

改写\acode{sum}的实现:

\begin{lstlisting}[language=Haskell]
  sum' :: (Num a) => [a] -> a
  sum' xs = foldl (\acc x -> acc + x) 0 xs
\end{lstlisting}

如果考虑到函数是柯里化的,那么可以简化实现:

\begin{lstlisting}[language=Haskell]
  sum' :: (Num a) => [a] -> a
  sum' = foldl (+) 0
\end{lstlisting}

这是因为 lambda 函数\acode{\\acc x -> acc + x}等同于\acode{(+)},所以可以省略掉\acode{xs}这个参数,因为调用\acode{foldl (+) 0}将
返回一个接受列表的函数。

接下来通过左折叠来实现\acode{elem}这个函数:

\begin{lstlisting}[language=Haskell]
  elem' :: (Eq a) => a -> [a] -> Bool
  elem' y ys = foldl (\acc x -> if x == y then True else acc) False ys
\end{lstlisting}

让我们看一下这里究竟做了些什么。这里的起始值与 accumulator 都是布尔值。在处理 fold 时,accumulator 与返回值的类型总是要一致的。这里的起始值
为\acode{False},即假设寻找的值并不在列表中。接着就是检查当前元素是否为需要找到的那个,如果是则将 accumulator 设为\acode{True},不是则不变。

\acode{foldr} 类似于左折叠,只不过 accumulator 是从列表右侧开始。

以下是使用右折叠来实现\acode{map}:

\begin{lstlisting}[language=Haskell]
  map' :: (a -> b) -> [a] -> [b]
  map' f xs = foldr (\x acc -> f x : acc) [] xs
\end{lstlisting}

如果将\acode{(+3)}映射至\acode{[1,2,3]},则是从列表右侧开始,获取最后一个元素,即\acode{3},再应用函数得出\acode{6},接着将其放入 accumulator
头部,即将\acode{[]}变为\acode{6:[]},这样\acode{[6]}现在变成了新的 accumulator(这里的\acode{:}就是将元素添加至头部)。

当然了,我们也可以用左折叠来实现:\acode{map' f xs = foldl (\\acc x -> acc ++ [f x]) [] xs},只不过\acode{++}函数比\acode{:}而言更加昂贵,
因此我们通常从一个列表构建一个新的列表时,会使用右折叠。

\textbf{折叠可以用于实现任意一个想要一次性遍历列表中所有元素的函数,并基于此进行返回。任何时候想要遍历一个列表来返回一些东西时,那么这个时候
  就很可能需要一个 fold。}这也是为什么在函数式编程中,连同 maps 与 filters,fold 是最有用的函数类型。

\acode{foldl1}与\acode{foldr1}类似于\acode{foldl}与\acode{foldr},区别在于不需要提供一个显式的初始值。它们假设首个(或最后的)元素为
起始值,接着从临近的元素开始进行 fold。由于依赖列表至少有一个元素,空列表的情况下它们会导致运行时错误;而\acode{foldl}与\acode{foldr}则
不会。

现在通过 fold 来实现标准库的函数,看看 fold 的强大之处:

\begin{lstlisting}[language=Haskell]
  maximum' :: (Ord a) => [a] -> a
  maximum' = foldr1 (\x acc -> if x > acc then x else acc)

  reverse' :: [a] -> [a]
  reverse' = foldl (\acc x -> x : acc) []

  product' :: (Num a) => [a] -> a
  product' = foldr1 (*)

  filter' :: (a -> Bool) -> [a] -> [a]
  filter' p = foldr (\x acc -> if p x then x : acc else acc) []

  head' :: [a] -> a
  head' = foldr1 (\x _ -> x)

  last' :: [a] -> a
  last' = foldl1 (\_ x -> x)
\end{lstlisting}

\acode{head}更好的实现当然是模式匹配,这里只是使用\acode{fold}进行展示。这里的\acode{reverse'}定义很聪明,从左遍历列表,每次将得到的元素
插入至 accumulator 的头部。\acode{\acc x -> x : acc}看起来像是\acode{:}函数,只不过翻转了参数,这也是为什么可以将\acode{reverse}函数改写
为\acode{foldl (flip (:)) []}。

\acode{scanl}与\acode{scanr}很像\acode{foldl}与\acode{foldr},不同之处在于后者用列表来存储 accumulator 的状态变化;同样\acode{scanl1}
与\acode{scanr1}类似于\acode{foldl1}与\acode{foldr1}。

\begin{lstlisting}[language=Haskell]
  ghci> scanl (+) 0 [3,5,2,1]
  [0,3,8,10,11]
  ghci> scanr (+) 0 [3,5,2,1]
  [11,8,3,1,0]
  ghci> scanl1 (\acc x -> if x > acc then x else acc) [3,4,5,3,7,9,2,1]
  [3,4,5,5,7,9,9,9]
  ghci> scanl (flip (:)) [] [3,2,1]
  [[],[3],[2,3],[1,2,3]]
\end{lstlisting}

Scans 可以被认为是一个函数可被 fold 的过程监控。让我们回答这样一个问题:\textbf{需要多少个元素才能让所有自然数的根之和超过 1000?}获取所有
自然数平方根只需要\acode{map sqrt [1..]},那么想要和,可以使用 fold,但是又因为我们感兴趣的是求和的这个过程,那么这里就可以使用 scan。一旦
完成了 scan,我们就可以看到有多少和是少于 1000 的了。

\begin{lstlisting}[language=Haskell]
  sqrtSums :: Int
  sqrtSums = length (takeWhile (<1000) (scanl1 (+) (map sqrt [1..]))) + 1
\end{lstlisting}

\begin{lstlisting}[language=Haskell]
  ghci> sqrtSums
  131
  ghci> sum (map sqrt [1..131])
  1005.0942035344083
  ghci> sum (map sqrt [1..130])
  993.6486803921487
\end{lstlisting}

这里使用了\acode{takeWhile}而不是\acode{filter},因为后者不能作用于无限列表。尽管我们知道列表是递增的,而\acode{filter}并不知道,因此这里
的\acode{takeWhile}在第一个 sum 大于 1000 时将截断 scan 列表。

\subsection*{通过 \$ 符号进行函数应用}

接下来让我们看一下\acode{\$}函数,也被称为\textit{函数应用 function application}。首先来看一下它的定义:

\begin{lstlisting}[language=Haskell]
  ($) :: (a -> b) -> a -> b
  f $ x = f x
\end{lstlisting}

大多数时候,它是一个便捷的函数使得无需再写很多括号。考虑一下表达式\acode{sum (map sqrt [1..130])},因为\acode{\$}拥有最低的优先级,那么可以
将该表达式重写为\acode{sum \$ map sqrt [1..130]},省了很多键盘敲击!当遇到一个\acode{\$},在其右侧的表达式会作为参数应用于左边的函数。
那么\acode{sqrt 3 + 4 + 9}呢?这会将 9, 4 以及 3 的平方根。那么如何得到 \textit{3 + 4 + 9} 的平方根呢,需要\acode{sqrt (3 + 4 + 9)},
那么如果使用\acode{\$}可以改为\acode{sqrt \$ 3 + 4 + 9}因为\acode{\$}在所有操作符中的优先级最低。这就是为什么可以想象一个\acode{\$}相当于
分别在其右侧以及等式的最右侧编写了一个隐形的圆括号。

那么\acode{sum (filter (> 10) (map (*2) [2..10]))}呢?由于\acode{\$}是右结合的,\acode{f (g (z x))}就相当于\acode{f \$ g \$ z x}。
那么用\acode{\$}重写便是\acode{sum \$ filter (>10) \$ map (*2) [2..10]}。

除开省略掉圆括号,\acode{\$}意味着函数应用可以被\textbf{视为另一个函数},这样的话就可以将其映射 map 至一个列表的函数:

\begin{lstlisting}[language=Haskell]
  ghci> map ($ 3) [(4+), (10*), (^2), sqrt]
  [7.0,30.0,9.0,1.7320508075688772]
\end{lstlisting}

\subsection*{组合函数}

数学里的组合函数定义为 $(f \circ g)(x) = f(g(x))$,意为组合两个函数来产生一个新的函数,当一个参数\acode{x}输入时,相当于带着\acode{x}输入至
\acode{g}再将结果输入至\acode{f}。

Haskell 中的组合函数基本上等同于数学定义中的一样。通过\acode{.}函数来组合函数,其定义为:

\begin{lstlisting}[language=Haskell]
(.) :: (b -> c) -> (a -> b) -> a -> c
f . g = \x -> f (g x)
\end{lstlisting}

注意类型声明。\acode{f}的参数类型与\acode{g}的返回类型一致。

函数组合的用途之一是将函数动态的传递给其他函数。当然了,这点 lambdas 也可以做到,但是很多情况下,组合函数更加简洁精炼。假设我们有一个列表的数值,并
希望将它们全部转为复数。其中一种办法就是将它们全部取绝对值后再添加负号:

\begin{lstlisting}[language=Haskell]
  ghci> map (\x -> negate (abs x)) [5,-3,-6,7,-3,2,-19,24]
  [-5,-3,-6,-7,-3,-2,-19,-24]
\end{lstlisting}

注意到了 lambda 以及组合函数的样式,我们可以将上述重写为:

\begin{lstlisting}[language=Haskell]
  ghci> map (negate . abs) [5,-3,-6,7,-3,2,-19,24]
  [-5,-3,-6,-7,-3,-2,-19,-24]
\end{lstlisting}

完美!组合函数是右结合的,所以我们可以一次性进行多次组合。表达式\acode{f (g (z x))}等同于\acode{(f . g . z) x}。了解到这些以后,我们可以将:

\begin{lstlisting}[language=Haskell]
  ghci> map (\xs -> negate (sum (tail xs))) [[1..5],[3..6],[1..7]]
  [-14,-15,-27]
\end{lstlisting}

转换为:

\begin{lstlisting}[language=Haskell]
  ghci> map (negate . sum . tail) [[1..5],[3..6],[1..7]]
  [-14,-15,-27]
\end{lstlisting}

那么对于接受多个参数的函数而言呢?如果对它们进行组合函数,就需要偏应用它们使得每个函数仅接受一个参数。\acode{sum (replicate 5 (max 6.7 8.9))}
可以被重写为\acode{(sum . replicate 5 . max 6.7) 8.9}或者是\acode{sum . replicate 5 . max 6.7 \$ 8.9}。那么这里实际上发生的是:
一个函数接受了\acode{max 6.7}所接受的参数,并将\acode{replicate 5}应用至其,接着一个函数接受计算的出的结果并对其求和,最后则是带着\acode{8.9}
调用该函数。不过正常来讲,人类的读取应为:将\acode{8.9}应用至\acode{max 6.7},接着应用\acode{replicate 5},最后则是求和。如果想用组合函数
重写一个又很多圆括号的表达式,可以先把最内存函数的最后一个参数放在\acode{\$}后面,然后在不带最后一个参数的情况下,组合其他的函数调用,即在函数之间
加上\acode{.}。如果有一个表达式\acode{replicate 100 (product (map (*3) (zipWith max [1,2,3,4,5] [4,5,6,7,8])))},那么可以重写为
\acode{replicate 100 . product . map (*3) . zipWith max [1,2,3,4,5] [4,5,6,7,8]}。如果表达式的结尾有三个圆括号,转换为组合函数时,
则会有三个组合操作符。

组合函数的另一个常见的用法是以所谓的点自由样式(也称无点样式)来定义函数。拿之前我们写的一个列子:

\begin{lstlisting}[language=Haskell]
  sum' :: (Num a) => [a] -> a
  sum' xs = foldl (+) 0 xs
\end{lstlisting}

\acode{xs}在两边都暴露出来。由于有柯里化,我们可以在两边省略\acode{xs},因为调用\acode{foldl (+) 0}会创建一个接受一个列表的函数。编写函数
\acode{sum' = foldl (+) 0}就是被称为无点样式。那么以下函数如何转换成无点样式呢?

\begin{lstlisting}[language=Haskell]
  fn x = ceiling (negate (tan (cos (max 50 x))))
\end{lstlisting}

我们没法将\acode{x}从等式两边的右侧移除,函数体中的\acode{x}后面有圆括号。\acode{cos (max 50)}显然没有意义。那么将\acode{fn}表达为组合
函数即可:

\begin{lstlisting}[language=Haskell]
  fn = ceiling . negate . tan . cos . max 50
\end{lstlisting}

漂亮!无点样式可以令用户去思考函数的组合方式,而非数据的传递方式,这样更加的简明。你可以将一组简单的函数组合在一起,使之成为一个负责的函数。不过如果
函数过于复杂,再使用无点样式往往会达到反效果。因此构造较长的函数组合链并不好,更好的解决方案则是用\acode{let}语句将中间的运算结果绑定一个名字,
或者说把问题分解成几个小问题再组合到一起。

本章的 maps 与 filters 中,我们写了:

\begin{lstlisting}[language=Haskell]
  oddSquareSum :: Integer
  oddSquareSum = sum (takeWhile (<10000) (filter odd (map (^2) [1..])))
\end{lstlisting}

那么更为函数式的写法则是

\begin{lstlisting}[language=Haskell]
  oddSquareSum :: Integer
  oddSquareSum = sum . takeWhile (<10000) . filter odd . map (^2) $ [1..]
\end{lstlisting}

如果需要可读性,则可以这样:

\begin{lstlisting}[language=Haskell]
  oddSquareSum :: Integer
  oddSquareSum =
      let oddSquares = filter odd $ map (^2) [1..]
          belowLimit = takeWhile (<10000) oddSquares
      in  sum belowLimit
\end{lstlisting}

\end{document}
