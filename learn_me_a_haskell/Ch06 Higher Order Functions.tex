\documentclass[./main.tex]{subfiles}

\begin{document}

\subsection*{柯里化函数}

在 Haskell 中每个函数实质上仅接受一个参数。那么迄今为止定义的那么多函数是怎么接受多个参数的呢?这是就是\textbf{柯里化函数 curried functions}。

\begin{lstlisting}[language=Haskell]
  ghci> max 4 5
  5
  ghci> (max 4) 5
  5
\end{lstlisting}

两个参数间用空格间隔就是简单的\textbf{函数应用 function application}。空格类似于一个操作符,其拥有最高的优先级。例如\acode{max},其签名为
\acode{max :: (Ord a) => a -> a -> a},可以被重写为\acode{max :: (Ord a) => a -> (a -> a)},可以这么理解:\acode{max}接受一个
\acode{a}并返回(即\acode{->})一个函数,该函数接受一个\acode{a}并返回一个\acode{a}。这就是为什么返回值类型以及函数的参数都是由箭头符
进行分隔的。

那么这样做有什么便利?简单来说如果调用一个仅几个参数的函数,我们得到的是一个\textbf{部分应用 partially applied}的函数,即一个函数接受的参数
与留下未填的参数一样多。

来观测一个简单的函数:

\begin{lstlisting}[language=Haskell]
  multThree :: (Num a) => a -> a -> a -> a
  multThree x y z = x * y * z
\end{lstlisting}

当使用\acode{multThree 3 5 9}或者\acode{((multThree 3) 5) 9}时到底发生了什么?首先,\acode{3}应用至\acode{multThree},因为它们由空格
进行了分隔(最高优先级)。这就创建了一个接受一个参数的函数,并返回了一个函数。接下来\acode{5}被应用至该函数,以此类推。记住我们的函数类型同样也
可以重写成\acode{multThree :: (Num a) => a -> (a -> (a -> a))}。接下来观察:

\begin{lstlisting}[language=Haskell]
  ghci> let multTwoWithNine = multThree 9
  ghci> multTwoWithNine 2 3
  54
  ghci> let multWithEighteen = multTwoWithNine 2
  ghci> multWithEighteen 10
  180
\end{lstlisting}

调用函数时输入不足的参数,实际上实在创造新的函数。那么如果希望创建一个函数接受一个值并将其与\acode{100}进行比较呢?

\begin{lstlisting}[language=Haskell]
  compareWithHundred :: (Num a, Ord a) => a -> Ordering
  compareWithHundred x = compare 100 x
\end{lstlisting}

如果带着\acode{99}调用它,返回一个\acode{GT}。注意\acode{x}同时位于等式的右侧。那么调用\acode{compare 100}返回的是什么呢?它返回一个接受
一个数值参数并将其与\acode{100}进行比较的函数。现在将其重写:

\begin{lstlisting}[language=Haskell]
  compareWithHundred :: (Num a, Ord a) => a -> Ordering
  compareWithHundred = compare 100
\end{lstlisting}

类型声明仍然相同,因为\acode{compare 100}返回一个函数。\acode{compare}的类型是\acode{(Ord a) -> a -> (a -> Ordering)},带着
\acode{100}调用它返回一个\acode{(Num a, Ord a) => a -> Ordering}。这里额外的类约束溜走了,这是因为\acode{100}同样也是\acode{Num}
类的一部分。

中缀函数同样可以通过使用分割被部分应用。要分割中缀函数,只需将其用圆括号括起来,并只在一侧提供参数:

\begin{lstlisting}[language=Haskell]
  divideByTen :: (Floating a) => a -> a
  divideByTen = (/10)
\end{lstlisting}

调用\acode{divideByTen 200}等同于\acode{200 / 10},等同于\acode{(/10) 200}。

那么如果在 GHCI 中尝试\acode{multThree 3 4}而不是通过\textit{let}将其与名称绑定,或是将其传递至另一个函数呢?

\begin{lstlisting}[language=Haskell]
  ghci> multThree 3 4
  <interactive>:1:0:
      No instance for (Show (t -> t))
        arising from a use of `print' at <interactive>:1:0-12
      Possible fix: add an instance declaration for (Show (t -> t))
      In the expression: print it
      In a 'do' expression: print it
\end{lstlisting}

GHCI 会提示我们表达式生成了一个类型为\acode{a -> a}的函数,但是并不知道该如何将其打印至屏幕。函数并不是\acode{Show} typeclass 的实例,
因此我们并不会得到一个函数的展示。

\subsection*{一些高级是在于顺序}

函数可以接受函数作为其参数,也可以返回函数。

\begin{lstlisting}[language=Haskell]
  applyTwice :: (a -> a) -> a -> a
  applyTwice f x = f (f x)
\end{lstlisting}

首先注意的是类型声明。之前我们是不需要圆括号的,因为\acode{->}是自然地右结合。然而在这里却是强制性的,它们表明了第一个参数是一个接受某物并
返回某物的函数,第二个参数同上所述。我们可以用柯里化函数的方式来进行解读,不过为了避免头疼,我们仅需要说该函数接受两个参数并返回一个值。
这里第一个参数是一个函数(即类型\acode{a -> a}),而第二个参数则是\acode{a}。

函数体非常的简单,仅需要使用参数\acode{f}作为一个函数,通过一个空格将\acode{x}应用至其,接着再应用一次\acode{f}。

\begin{lstlisting}[language=Haskell]
  ghci> applyTwice (+3) 10
  16
  ghci> applyTwice (++ " HAHA") "HEY"
  "HEY HAHA HAHA"
  ghci> applyTwice ("HAHA " ++) "HEY"
  "HAHA HAHA HEY"
  ghci> applyTwice (multThree 2 2) 9
  144
  ghci> applyTwice (3:) [1]
  [3,3,1]
\end{lstlisting}

可以看到单个高阶函数可以被用以多种用途。而在命令式编程中,通常使用的是 for 循环、while 循环、将某物设置为一个变量、检查其状态等等,为了达到
某些行为,还需要用接口将其封装,类似于函数;而函数式编程则使用高阶函数来抽象出相同的模式。

现在让我们实现一个名为\acode{flip}的标准库已经存在的函数,其接受一个函数并返回一个类似于原来函数的函数,仅前两个参数被翻转。简单的实现:

\begin{lstlisting}[language=Haskell]
  filp' :: (a -> b -> c) -> (b -> a -> c)
  filp' f = g
    where
      g x y = f y x
\end{lstlisting}

观察类型声明,\acode{flip'}接受一个函数,该函数接受一个\acode{a}与\acode{b},并返回一个函数,该返回的函数接受一个\acode{b}与\acode{a}。
然而默认情况下函数是柯里化的,第二个圆括号是没有必要的,因为\acode{->}默认是右结合的。\acode{(a -> b -> c) -> (b -> a -> c)}等同于
\acode{(a -> b -> c) -> (b -> (a -> c))},等同于\acode{(a -> b -> c) -> b -> a -> c}。我们可以用更简单方式来定义该函数:

\begin{lstlisting}[language=Haskell]
  filp'' :: (a -> b -> c) -> b -> a -> c
  filp'' f y x = f x y
\end{lstlisting}

这里我们利用了函数都是柯里化的便利。当不带参数\acode{y}与\acode{x}时调用\acode{flip'' f}时,它将返回一个\acode{f},该函数接受两个参数,
只不过它们的位置是翻转的。

\begin{lstlisting}[language=Haskell]
  ghci> flip' zip [1,2,3,4,5] "hello"
  [('h',1),('e',2),('l',3),('l',4),('o',5)]
  ghci> zipWith (flip' div) [2,2..] [10,8,6,4,2]
  [5,4,3,2,1]
\end{lstlisting}

\subsection*{Maps \& filters}

\acode{map}接受一个函数以及一个列表,将该函数应用至列表中的每一个元素中,生产一个新的列表。让我们来看一下类型签名:

\begin{lstlisting}[language=Haskell]
  map :: (a -> b) -> [a] -> [b]
  map _ [] = []
  map f (x : xs) = f x : map f xs
\end{lstlisting}

测试:

\begin{lstlisting}[language=Haskell]
  ghci> map (+3) [1,3,5,1,6]
  [4,6,8,4,9]
  ghci> map (-1) [1,3,5,1,6]

  <interactive>:2:1: error:
      • Could not deduce (Num a0)
          arising from a type ambiguity check for
          the inferred type for ‘it’
        from the context: (Num a, Num (a -> b))
          bound by the inferred type for ‘it’:
                     forall {a} {b}. (Num a, Num (a -> b)) => [b]
          at <interactive>:2:1-20
        The type variable ‘a0’ is ambiguous
        These potential instances exist:
          instance Num Integer -- Defined in ‘GHC.Num’
          instance Num Double -- Defined in ‘GHC.Float’
          instance Num Float -- Defined in ‘GHC.Float’
          ...plus two others
          ...plus one instance involving out-of-scope types
          (use -fprint-potential-instances to see them all)
      • In the ambiguity check for the inferred type for ‘it’
        To defer the ambiguity check to use sites, enable AllowAmbiguousTypes
        When checking the inferred type
          it :: forall {a} {b}. (Num a, Num (a -> b)) => [b]
  ghci> map (subtract 1) [1,3,5,1,6]
  [0,2,4,0,5]
  ghci> map (++ "!") ["BIFF", "BANG", "POW"]
  ["BIFF!","BANG!","POW!"]
  ghci> map (replicate 3) [3..6]
  [[3,3,3],[4,4,4],[5,5,5],[6,6,6]]
  ghci> map (map (^2)) [[1,2], [3,4,5,6], [7,8]]
  [[1,4],[9,16,25,36],[49,64]]
  ghci> map fst [(1,2),(3,5),(6,3),(2,6),(2,5)]
  [1,3,6,2,2]
\end{lstlisting}

原书上一章有提到过\acode{-1}这样的情况,报错的原因是 Haskell 将\acode{-1}识别为负数而不是减法,需要显式调用\acode{subtract}
才能识别为 partial 函数并应用至列表中的各个元素上。

\acode{filter}接受一个子句(该子句是一个函数,用于告知某物是否为真),以及一个列表,并返回满足该子句的元素列表。类型签名如下:

\begin{lstlisting}[language=Haskell]
  filter :: (a -> Bool) -> [a] -> [a]
  filter _ [] = []
  filter p (x : xs)
    | p x = x : filter p xs
    | otherwise = filter p xs
\end{lstlisting}

测试:

\begin{lstlisting}[language=Haskell]
  ghci> filter (>3) [1,5,3,2,1,6,4,3,2,1]
  [5,6,4]
  ghci> filter (==3) [1,2,3,4,5]
  [3]
  ghci> filter even [1..10]
  [2,4,6,8,10]
  ghci> let notNull x = not (null x) in filter notNull [[1,2,3],[],[3,4,5],[2,2],[],[],[]]
  [[1,2,3],[3,4,5],[2,2]]
  ghci> filter (`elem` ['a'..'z']) "u LaUgH aT mE BeCaUsE I aM diFfeRent"
  "uagameasadifeent"
  ghci> filter (`elem` ['A'..'Z']) "i lauGh At You BecAuse u r aLL the Same"
  "GAYBALLS"
\end{lstlisting}

将上一章的\acode{quicksort}中的列表表达式替换为\acode{filter}:

\begin{lstlisting}[language=Haskell]
  quicksort :: (Ord a) => [a] -> [a]
  quicksort [] = []
  quicksort (x : xs) =
    let smallerSorted = quicksort (filter (<= x) xs)
        biggerSorted = quicksort (filter (> x) xs)
     in smallerSorted ++ [x] ++ biggerSorted
\end{lstlisting}

现在尝试一下\textbf{找到 100,100 以下最大能被 3829 的值}:

\begin{lstlisting}[language=Haskell]
  largestDivisible :: (Integral a) => a
  largestDivisible = head (filter p [100000, 99999 ..])
    where
      p x = x `mod` 3829 == 0
\end{lstlisting}

接下来尝试一下\textbf{找到所有奇数平方在 10,000 以下的和},不过首先要介绍一下\acode{takeWhile}函数。该函数接受一个子句以及一个列表,
接着从列表头向后遍历,在子句返回真时返回元素,一旦子句返回假则结束遍历。可以在 GHCI 上用一行来完成任务:

\begin{lstlisting}[language=Haskell]
  ghci> sum (takeWhile (<10000) (filter odd (map (^2) [1..])))
  166650
\end{lstlisting}

当然也可以用列表表达式:

\begin{lstlisting}[language=Haskell]
  ghci> sum (takeWhile (<10000) [n^2 | n <- [1..], odd (n^2)])
  166650
\end{lstlisting}

接下来一个问题是处理考拉兹猜想 Collatz sequences,其数学表达为:

\begin{equation*}
  f(n) =
  \begin{cases}
    n / 2  & \text{if $n \equiv 0$ ($\mod 2$)} \\
    3n + 1 & \text{if $n \equiv 1$ ($\mod 2$)} \\
  \end{cases}
\end{equation*}

我们希望知道的是:\textbf{对于从 1 至 100 的所有数开始,有多少链的长度是大于 15 的?}首先编写一个函数用于生成链:

\begin{lstlisting}[language=Haskell]
  chain :: (Integral a) => a -> [a]
  chain 1 = [1]
  chain n
    | even n = n : chain (n `div` 2)
    | odd n = n : chain (n * 3 + 1)
\end{lstlisting}

因为链的最后一位肯定是 1,也就是边界,那么这就是一个简单的递归函数了。测试:

\begin{lstlisting}[language=Haskell]
  ghci> chain 10
  [10,5,16,8,4,2,1]
  ghci> chain 1
  [1]
  ghci> chain 30
  [30,15,46,23,70,35,106,53,160,80,40,20,10,5,16,8,4,2,1]
\end{lstlisting}

看起来能正常工作,接下来就是获取长度:

\begin{lstlisting}[language=Haskell]
  numLongChains :: Int
  numLongChains = length (filter isLong (map chain [1 .. 100]))
    where
      isLong xs = length xs > 15
\end{lstlisting}

我们将\acode{chain}函数映射至\acode{[1..100]}来获取一个链的列表,接着根据检查长度是否超过 15 的子句来过滤它们、一旦完成过滤,我们就可以看到
结果的列表中还剩多少链。

\begin{anote}
  该函数的类型是\acode{numLongChains :: Int},因为历史原因\acode{length}返回一个\acode{Int}而不是一个\acode{Num a}。如果我们需要返回一个
  更通用的\acode{Num a},可以对返回的长度使用\acode{fromIntegral}。
\end{anote}

使用\acode{map},我们还可以这样做\acode{map (*) [0..]},如果不是因为别的原因要解释柯里化以及偏函数是实值,那么可以将其传递至其它函数,
或者置入列表中(仅仅不能将其变为字符串)。迄今为止,我们只映射了单参数的函数至列表,例如\acode{map (*2) [0..]}类型是\acode{(Num a) => [a]},
我们同样可以这么做\acode{map (*) [0..]}。这里将会对列表中的每个数值应用函数\acode{*},即类型为\acode{(Num a) => a -> a -> a}。将一个参数
应用于需要两个参数的函数将会返回需要一个参数的函数。如果将\acode{*}映射至列表\acode{[0..]},得到的则是一个接受单个参数的函数的列表,即
\acode{(Num a) => [a -> a]}。也就是说\acode{map (*) [0..]}生产一个这样的列表\acode{[(0*),(1*),(2*),(3*),(4*),(5*)..]}。

\begin{lstlisting}[language=Haskell]
  ghci> let listOfFuns = map (*) [0..]
  ghci> (listOfFuns !! 4) 5
  20
\end{lstlisting}

\subsection*{Lambdas}

% TODO

\end{document}
