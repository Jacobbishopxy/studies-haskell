\documentclass[./main.tex]{subfiles}

\begin{document}

\subsection*{模式匹配}

一个简单案例:

\begin{lstlisting}[language=Haskell]
  sayMe :: (Integral a) => a -> String
  sayMe 1 = "One!"
  sayMe 2 = "Two!"
  sayMe 3 = "Three!"
  sayMe 4 = "Four!"
  sayMe 5 = "Five!"
  sayMe x = "Not between 1 and 5"
\end{lstlisting}

一个递归案例:

\begin{lstlisting}[language=Haskell]
  factorial :: (Integral a) => a ->
  factorial 0 = 1
  factorial n = n * factorial (n - 1)
\end{lstlisting}

模式匹配也可以失败:

\begin{lstlisting}[language=Haskell]
  charName :: Char -> String
  charName 'a' = "Albert"
  charName 'b' = "Broseph"
  charName 'c' = "Cecil"
\end{lstlisting}

当输入并不是期望时:

\begin{lstlisting}[language=Haskell]
  ghci> charName 'a'
  "Albert"
  ghci> charName 'b'
  "Broseph"
  ghci> charName 'h'
  "*** Exception: tut.hs:(53,0)-(55,21): Non-exhaustive patterns in function charName
\end{lstlisting}

即出现了非穷尽的匹配,因此我们总是需要捕获所有模式。

模式匹配也可用作于元组:

\begin{lstlisting}[language=Haskell]
  addVectors :: (Num a) => (a, a) -> (a, a) -> (a, a)
  addVectors a b = (fst a + fst b, snd a + snd b)
\end{lstlisting}

模式匹配也可作用于列表表达式:

\begin{lstlisting}[language=Haskell]
  ghci> let xs = [(1,3), (4,3), (2,4), (5,3), (5,6), (3,1)]
  ghci> [a+b | (a,b) <- xs]
  [4,7,6,8,11,4]
\end{lstlisting}

\begin{anote}
  \acode{x:xs} 模式的使用很常见,特别是递归函数。但是包含\acode{:}的模式只匹配长度为 1 或更多的数组。
\end{anote}

如果希望获取前三个元素以及数组剩余元素,那么可以使用 \acode{x:y:z:zs},那么这样仅匹配有三个或以上的元素的数组。

其它案例:

\begin{lstlisting}[language=Haskell]
  tell :: (Show a) => [a] -> String
  tell [] = "The list is empty"
  tell (x:[]) = "The list has one element: " ++ show x
  tell (x:y:[]) = "The list has two elements: " ++ show x ++ " and " ++ show y
  tell (x:y:_) = "This list is long. The first two elements are: " ++ show x ++ " and " ++ show y
\end{lstlisting}

该函数是安全的,因为它考虑到了空数组,以及若干元素数组的情况。

之前通过列表表达式编写了 \acode{length} 函数,现在可以通过模式匹配再加上递归的方式实现一遍:

\begin{lstlisting}[language=Haskell]
  length' :: (Num b) => [a] -> b
  length' [] = 0
  length' (_ : xs) = 1 + length' xs
\end{lstlisting}

接下来是实现 \acode{sum}:

\begin{lstlisting}[language=Haskell]
  sum' :: (Num a) => [a] -> a
  sum' [] = 0
  sum' (x:xs) = x + sum' xs
\end{lstlisting}

同样还有一种被称为\textit{as 模式}的,即在模式前添加名称以及\acode{@}符号,例如\acode{xs@(x:y:ys)},该模式将匹配
\acode{x:y:ys},同时用户可以轻易的通过\acode{xs}来获取整个数组,而无需重复使用\acode{x:y:ys}进行表达:

\begin{lstlisting}[language=Haskell]
  capital :: String -> String
  capital "" = "Empty string, whoops!"
  capital all@(x : xs) = "The first letter of " ++ all ++ " is " ++ [x]
\end{lstlisting}

最后,用户在模式匹配中不能使用 \acode{++} 符号。

\subsection*{守护!}

守护是一种检测值的某些属性是否为真,看上去像是\acode{if}语句,但是其可读性更强:

\begin{lstlisting}[language=Haskell]
  bmiTell :: (RealFloat a) => a -> String
  bmiTell bmi
    | bmi <= 18.5 = "You're underweight, you emo, you!"
    | bmi <= 25.0 = "You're supposedly normal. Pffft, I bet you're ugly!"
    | bmi <= 30.0 = "You're fat! Lose some weight, fatty!"
    | otherwise = "You're a whale, congratulations!"
\end{lstlisting}

守护是由管道符并接着一个函数名以及函数参数进行定义的。守护本质上就是一个布尔表达式,如果为\acode{True},那么其关联的函数体被执行;
如果为\acode{False},那么检查则会移至下一个守护,以此类推。

大多数时候最后一个守护是\acode{otherwise},其被简单的定义为\acode{otherwise = True}并捕获所有情况。

当然我们可以使用任意参数的函数来守护:

\begin{lstlisting}[language=Haskell]
  bmiTell' :: (RealFloat a) => a -> a -> String
  bmiTell' weight height
    | weight / height ^ 2 <= 18.5 = "You're underweight, you emo, you!"
    | weight / height ^ 2 <= 25.0 = "You're supposedly normal. Pffft, I bet you're ugly!"
    | weight / height ^ 2 <= 30.0 = "You're fat! Lose some weight, fatty!"
    | otherwise = "You're a whale, congratulations!"
\end{lstlisting}

另外我们可以实现自己的\acode{max}与\acode{compare}函数:

\begin{lstlisting}[language=Haskell]
  max' :: (Ord a) => a -> a -> a
  max' a b
    | a > b = a
    | otherwise = b
\end{lstlisting}

\begin{lstlisting}[language=Haskell]
  compare' :: (Ord a) => a -> a -> Ordering
  a `compare'` b
    | a > b = GT
    | a == b = EQ
    | otherwise = LT
\end{lstlisting}

\begin{anote}
  我们不仅可以通过引号来调用函数,也可以使用引号来定义他们,有时这样会更加便于阅读。
\end{anote}

\subsection*{Where!?}

上一节的\acode{bmiTell'}函数中的\acode{weight / height ^ 2}被重复了三遍,可以只计算一次并通过名称来绑定计算结果:

\begin{lstlisting}[language=Haskell]
  bmiTell'' :: (RealFloat a) => a -> a -> String
  bmiTell'' weight height
    | bmi <= 18.5 = "You're underweight, you emo, you!"
    | bmi <= 25.0 = "You're supposedly normal. Pffft, I bet you're ugly!"
    | bmi <= 30.0 = "You're fat! Lose some weight, fatty!"
    | otherwise = "You're a whale, congratulations!"
    where
      bmi = weight / height ^ 2
\end{lstlisting}

我们在守护的结尾添加了\acode{where}并定义了\acode{bmi}这个名称,这里定义的名称对整个守护可见,这样就无需再重复同样代码了。
那么我们可以进行更多的定义:

\begin{lstlisting}[language=Haskell]
  bmiTell''' :: (RealFloat a) => a -> a -> String
  bmiTell''' weight height
    | bmi <= skinny = "You're underweight, you emo, you!"
    | bmi <= normal = "You're supposedly normal. Pffft, I bet you're ugly!"
    | bmi <= fat = "You're fat! Lose some weight, fatty!"
    | otherwise = "You're a whale, congratulations!"
    where
      bmi = weight / height ^ 2
      skinny = 18.5
      normal = 25.0
      fat = 30.0
\end{lstlisting}

当然我们可以通过\textbf{模式匹配}来进行变量绑定!上面\acode{where}中的代码可以改写为:

\begin{lstlisting}[language=Haskell]
    ...
    where
      bmi = weight / height ^ 2
      (skinny, normal, fat) = (18.5, 25.0, 30.0)
\end{lstlisting}

现在让我们编写另一个相当简单的函数用作获取名字首字母:

\begin{lstlisting}[language=Haskell]
  initials :: String -> String -> String
  initials first_name last_name = [f] ++ ". " ++ [l] ++ "."
    where
      (f : _) = first_name
      (l : _) = last_name
\end{lstlisting}

我们可以直接将模式匹配应用于函数参数。

另外,正如我们可以在\acode{where}块中定义约束,我们也可以定义函数:

\begin{lstlisting}[language=Haskell]
  calcBmis :: (RealFloat a) => [(a, a)] -> [a]
  calcBmis xs = [bmi w h | (w, h) <- xs]
    where
      bmi weight height = weight / height ^ 2
\end{lstlisting}

\acode{where}绑定也可以是嵌套的,这在编写函数中很常见:定义一些辅助函数在函数\acode{where}子句,然后这些函数的辅助函数又在
其自身的\acode{where}子句中。

\subsection*{Let 的用法}

与\acode{where}绑定很相似的是\acode{let}绑定。前者是一个语法构造器用于在函数的尾部进行变量绑定,这些变量可供整个函数使用,
包括守护;而后者则是在任意处绑定一个变量,其自身为表达式,不过只在作用域生效,因此不能被守护中访问。与 Haskell 其他任意的构造
一样,\acode{let}绑定也可使用模式匹配:

\begin{lstlisting}[language=Haskell]
  cylinder :: (RealFloat a) => a -> a -> a
  cylinder r h =
    let sideArea = 2 * pi * r * h
        topArea = pi * r ^ 2
     in sideArea + 2 * topArea
\end{lstlisting}

这里的结构是\acode{let <bindings> in <expression>}。在\textit{let}中定义的名称可以在\textit{in}之后的表达式中访问。
这里同样要注意缩进。现在看来\acode{let}仅仅将绑定提前,与\acode{where}的作用无异。

不同点在于\acode{let}绑定是表达式自身,而\acode{where}仅为语法构造。还记得之前提到过的\acode{if else}语句是表达式,可以
在任意处构造:

\begin{lstlisting}[language=Haskell]
  ghci> [if 5 > 3 then "Woo" else "Boo", if 'a' > 'b' then "Foo" else "Bar"]
  ["Woo", "Bar"]
  ghci> 4 * (if 10 > 5 then 10 else 0) + 2
  42
\end{lstlisting}

那么\acode{let}绑定也可以:

\begin{lstlisting}[language=Haskell]
  ghci> 4 * (let a = 9 in a + 1) + 2
  42
\end{lstlisting}

同样可以在当前作用域引入函数:

\begin{lstlisting}[language=Haskell]
  ghci> [let square x = x * x in (square 5, square 3, square 2)]
  [(25,9,4)]
\end{lstlisting}

如果想要绑定若干变量,我们显然不能再列上对齐它们。这就是为什么需要用分号进行分隔:

\begin{lstlisting}[language=Haskell]
  ghci> (let a = 100; b = 200; c = 300 in a*b*c, let foo="Hey "; bar = "there!" in foo ++ bar)
  (6000000,"Hey there!")
\end{lstlisting}

正如之前提到的,可以将模式匹配应用于\textit{let}绑定:

\begin{lstlisting}[language=Haskell]
  ghci> (let (a,b,c) = (1,2,3) in a+b+c) * 100
  600
\end{lstlisting}

当然也可以将\textit{let}绑定置入列表表达式中:

\begin{lstlisting}[language=Haskell]
  calcBmis' :: (RealFloat a) => [(a, a)] -> [a]
  calcBmis' xs = [bmi | (w, h) <- xs, let bmi = w / h ^ 2]
\end{lstlisting}

将\acode{let}置入列表表达式中类似于一个子句,不过它不会对列表进行筛选,而仅仅绑定名称。该名称可被列表表达式的输出函数可见
(即在符号\acode{|}前的部分),以及所有的子句,以及绑定后的部分。因此我们可以让函数继续进行筛选:

\begin{lstlisting}[language=Haskell]
  calcBmis'' :: (RealFloat a) => [(a, a)] -> [a]
  calcBmis'' xs = [bmi | (w, h) <- xs, let bmi = w / h ^ 2, bmi >= 25.0]
\end{lstlisting}

我们不能在\acode{(w, h) <- xs}中使用\acode{bmi},因为它在\textit{let}绑定之前。

在列表表达式中使用\textit{let}绑定可以省略\textit{in}的那部分,这是因为名称的可视范围已经被预定义好了。不过我们还是可以
在一个子句中使用\textit{let in}绑定,该名称仅可在该子句中可见。在 GHCi 中定义函数与常数时,\textit{in}部分同样也可以
省略。如果这么做了,那么该名称可以被整个交互过程中可见。

\begin{lstlisting}[language=Haskell]
  ghci> let zoot x y z = x * y + z
  ghci> zoot 3 9 2
  29
  ghci> let boot x y z = x * y + z in boot 3 4 2
  14
  ghci> boot
  <interactive>:1:0: Not in scope: `boot'
\end{lstlisting}

\subsection*{Case 表达式}

以下两端代码表达的是同样一件事,它们互为可替换的。

\begin{lstlisting}[language=Haskell]
  head' :: [a] -> a
  head' [] = error "No head for empty lists!"
  head' (x:_) = x
\end{lstlisting}

\begin{lstlisting}[language=Haskell]
  head' :: [a] -> a
  head' xs = case xs of [] -> error "No head for empty lists!"
                        (x:_) -> x
\end{lstlisting}

正如所见的那样,case 表达式的语法特别简单:

\begin{lstlisting}[language=Haskell]
  case expression of pattern -> result
  pattern -> result
  pattern -> result
  ...
\end{lstlisting}

\acode{expression}与模式匹配。模式匹配的行为正如预期那样:首个匹配上表达式的那个模式将被使用。如果直到最后都没有合适的模式
被找到,那么将会抛出运行时错误。

函数参数的模式匹配只能在定义函数时完成,而 case 表达式则可以在任意处使用。例如:

\begin{lstlisting}[language=Haskell]
  describeList :: [a] -> String
  describeList xs =
    "The list is " ++ case xs of
      [] -> "empty."
      [x] -> "a singleton list."
      xs -> "a longer list."
\end{lstlisting}

case 表达式可以对表达式中间的模型内容进行模式匹配。函数定义中的模式匹配是 case 表达式的语法糖,因此我们也可以这样定义:

\begin{lstlisting}[language=Haskell]
  describeList' :: [a] -> String
  describeList' xs = "The list is " ++ what xs
    where
      what [] = "empty."
      what [x] = "a singleton list."
      what xs = "a longer list."
\end{lstlisting}

\end{document}
