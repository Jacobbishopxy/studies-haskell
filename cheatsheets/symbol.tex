\PassOptionsToPackage{quiet}{fontspec}
\documentclass[UTF8]{ctexart}

\usepackage[a4paper, total={6in, 8in}]{geometry}
\usepackage{lmodern}
\usepackage{amsfonts}
\usepackage{graphicx}
\usepackage{listings}
\usepackage{xcolor}
\usepackage{tikz}
\usepackage{framed}
\usepackage{amsthm,amssymb}
\usepackage[leqno]{amsmath}
\usepackage{hyperref}

\graphicspath{\subfix{./images/}}

\usepackage{blindtext}

% adding style
\usepackage{../style}
\lstset{style=mystyle}

% adding commands

% a new command which severs arbitrary section number
\newcommand{\asection}[2]{
  \setcounter{section}{#1}
  \addtocounter{section}{-1}
  \section{#2}
}

% a new command which simplify \lstinline
\newcommand{\acode}[1]{
  \lstinline[basicstyle=\ttfamily]{#1}
}

\definecolor{codered}{HTML}{cc0200}

\newcommand{\acodered}[1]{
  \colorbox{codered}{\lstinline[basicstyle=\ttfamily\color{white}]{#1}}
}

\colorlet{shadecolor}{yellow!30}
\newenvironment{anote}{\begin{shaded}\textbf{Note}\par}{\end{shaded}}


% no indent
\setlength{\parindent}{0pt}

%
\newtheoremstyle{sybs}{}{}{}{}{\bfseries}{}{.5em}{{\thmnumber{#2 }}{\thmname{#1}}{\thmnote{ (#3)}}}

\theoremstyle{sybs}
\newtheorem{syb}{}

\title{Cheatsheet of Haskell Symbols}
\author{}
\date{}


\begin{document}

\maketitle

\begin{syb}
  \acode{!} strictness flag
\end{syb}

任何时候一个数据构造函数被应用,当且仅当关联的类型在代数数据类型中包含了严格标志(strictness flag),即用惊叹号\acode{!}表示,
才会对构造函数的每个参数求值。

\begin{lstlisting}[language=Haskell]
  data Vec = Vec !Int
\end{lstlisting}

\begin{syb}
  \acode{!} bang pattern
\end{syb}

GHC2021 开始所支持的严格模式,即消除 thunks。其主要的目的是在模式匹配中加入一个新的语法:

\begin{lstlisting}[language=Haskell]
  pat ::= !pat
\end{lstlisting}

匹配一个表达式\acode{e}至模版\acode{!p}上则是首先计算\acode{e},然后再进行\acode{p}的匹配。

以下定义令\acode{f1}严格在\acode{x}上,如果没有\acode{!}那么它则是惰性的。换言之\acode{f1}这个函数的入参,会在\acode{f1}
被调用的时候第一时间被计算。

\begin{lstlisting}[language=Haskell]
  f1 !x = True
\end{lstlisting}

嵌套的 bang,\acode{f2}在\acode{x}上严格,而\acode{y}没有。换言之\acode{f2}这个参数的入参\acode{x},会在函数被调用时
第一时间被计算,而\acode{y}仍然是惰性的。

\begin{lstlisting}[language=Haskell]
  f2 (!x, y) = [x,y]
\end{lstlisting}

\begin{syb}
  \acode{\#} MagicHash
\end{syb}

详见
\href{https://downloads.haskell.org/ghc/latest/docs/users_guide/exts/primitives.html#glasgow-unboxed}{Unboxed 类型}。
GHC 中大多数类型都是 boxed 的,意为该类型的值被表示成一个堆对象的指针。例如 Haskell 中的\acode{Int},是一个堆对象。而 unboxed
类型,则是该值本身,即没有指针或牵涉到堆分配。

\textbf{MagicHash}还支持一些新的字面量:

\begin{itemize}
  \item $'x'\#$ 类型为 $Char\#$
  \item $"foo"\#$ 类型为 $Addr\#$
  \item $3\#$ 类型为 $Int\#$。通常而言,任何 Haskell 的整数语义加上 $\#$ 后就是一个 $Int\#$ 字面量,例如 $-0x3A\#$ 即 $32\#$
  \item $3\#\#$ 类型为 $Word\#$。通常而言,任何非负 Haskell 的整数语义加上 $\#\#$ 后就是一个 $Word\#$
  \item $3.2\#$ 类型为 $Float\#$
  \item $3.2\#\#$ 类型为 $Double\#$
\end{itemize}


\begin{lstlisting}[language=Haskell]
  data Int = I# Int#
\end{lstlisting}

\begin{syb}
  \acode{\#} OverloadedLabels
\end{syb}

详见
\href{https://downloads.haskell.org/ghc/latest/docs/users_guide/exts/overloaded_labels.html#overloaded-labels}{重载标签}。

\begin{lstlisting}[language=Haskell]
  example = #x (Point 1 2)
\end{lstlisting}

\begin{syb}
  \acode{\#} C pre-processor's directive
\end{syb}

\textbf{CPP}语言扩展。通过命令行加上\acode{-x}前缀:

\begin{lstlisting}[language=Haskell]
  $ ghc -XCPP foo.hs
\end{lstlisting}

也可以使用 LANGUAGE pragma 生效:

\begin{lstlisting}[language=Haskell]
  {-# LANGUAGE CPP #-}
\end{lstlisting}

使用:

\begin{lstlisting}[language=Haskell]
  #include "MachDeps.h"
\end{lstlisting}

\begin{syb}
  \acode{\#} hsc2hs command's operator
\end{syb}

详见
\href{https://downloads.haskell.org/ghc/latest/docs/users_guide/utils.html#input-syntax}{Input syntax}

\begin{lstlisting}[language=Haskell]
  flag = #const VER_MAJORVERSION
\end{lstlisting}

\begin{syb}
  \acode{?}
\end{syb}

% TODO

\begin{lstlisting}[language=Haskell]

\end{lstlisting}

\begin{syb}
  \acode{?}
\end{syb}



\begin{lstlisting}[language=Haskell]

\end{lstlisting}

\begin{syb}
  \acode{?}
\end{syb}



\begin{lstlisting}[language=Haskell]

\end{lstlisting}

\begin{syb}
  \acode{?}
\end{syb}



\begin{lstlisting}[language=Haskell]

\end{lstlisting}

\begin{syb}
  \acode{?}
\end{syb}



\begin{lstlisting}[language=Haskell]

\end{lstlisting}

\begin{syb}
  \acode{?}
\end{syb}



\begin{lstlisting}[language=Haskell]

\end{lstlisting}

\begin{syb}
  \acode{?}
\end{syb}



\begin{lstlisting}[language=Haskell]

\end{lstlisting}

\begin{syb}
  \acode{?}
\end{syb}



\begin{lstlisting}[language=Haskell]

\end{lstlisting}

\begin{syb}
  \acode{?}
\end{syb}



\begin{lstlisting}[language=Haskell]

\end{lstlisting}

\begin{syb}
  \acode{?}
\end{syb}



\begin{lstlisting}[language=Haskell]

\end{lstlisting}

\begin{syb}
  \acode{?}
\end{syb}



\begin{lstlisting}[language=Haskell]

\end{lstlisting}

\begin{syb}
  \acode{?}
\end{syb}



\begin{lstlisting}[language=Haskell]

\end{lstlisting}

\begin{syb}
  \acode{?}
\end{syb}



\begin{lstlisting}[language=Haskell]

\end{lstlisting}

\begin{syb}
  \acode{?}
\end{syb}



\begin{lstlisting}[language=Haskell]

\end{lstlisting}


引用:
\href{https://github.com/takenobu-hs/haskell-symbol-search-cheatsheet}{haskell-symbol-search-cheatsheet}

\end{document}
