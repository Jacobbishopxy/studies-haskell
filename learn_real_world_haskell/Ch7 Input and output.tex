\documentclass[./main.tex]{subfiles}

\begin{document}

\subsection*{Haskell 中的经典 I/O}

略

\subsection*{处理文件以及句柄}

Haskell 为 I/O 定义了相当多的基本函数,其中许多函数与其它编程语言中的函数相似。\acode{System.IO}库中提供了所有的基础 I/O 函数。

使用\acode{openFile}会返回一个文件的\acode{Handle},它用于对文件执行指定的操作。Haskell 提供了例如\acode{hPutStrLn}这样的
函数,其类似于\acode{putStrLn},不同在于接受一个额外的参数 -- 一个\acode{Handle} -- 指定哪个文件被操作。当我们结束时,需要用
\acode{hClose}来结束\acode{Handle}。这些函数都定义在\acode{System.IO}中,“h” 开头的函数对应了几乎所有的非 “h” 开头的函数;
例如\acode{print}打印在屏幕上,而\acode{hPrint}打印至一个文件。

一个例子:

\begin{lstlisting}[language=Haskell]
  import Data.Char (toUpper)
  import System.IO

  main :: IO ()
  main = do
    inh <- openFile "input.txt" ReadMode
    outh <- openFile "output.txt" WriteMode
    mainloop inh outh
    hClose inh
    hClose outh

    putStrLn "whatever"

  mainloop :: Handle -> Handle -> IO ()
  mainloop inh outh = do
    ineof <- hIsEOF inh
    if ineof
      then return ()
      else do
        inpStr <- hGetLine inh
        hPutStrLn outh (map toUpper inpStr)
        mainloop inh outh
\end{lstlisting}

\acode{mainloop}首先检查输入是否结束(EOF),如果没有则读取一行,将该行转为大写后写入输出文件中,再递归的调用\acode{mainloop}。

略

\begin{center}
  \begin{tabular}{|l|l|l|l|l|}
    \hline
    \multicolumn{5}{|c|}{\textbf{可能的 IOMode 值}}                                     \\
    \hline
    IOMode        & \textbf{可读?} & \textbf{可写?} & \textbf{起始位置} & \textbf{说明}       \\
    \hline
    ReadMode      & Yes          & No           & 文件起始          & 文件必须存在            \\
    WriteMode     & No           & Yes          & 文件起始          & 文件存在时会被清空         \\
    ReadWriteMode & Yes          & Yes          & 文件起始          & 文件不存在时创建;否则现有数据保留 \\
    AppendMode    & No           & Yes          & 文件尾部          & 文件不存在时创建;否则现有数据保留 \\
    \hline
  \end{tabular}
\end{center}

\subsubsection*{关闭句柄}

上述例子中我们已经见到了\acode{hClose}用于关闭文件的句柄。在之后的小节中我们将了解缓存 Buffering 这个概念,即 Haskell 在内部
为文件维护了缓存。这样做能显著的提升性能,然而这就意味着在对打开的文件调用\acode{hClose}之前,数据可能不会被刷新到操作系统中。

另一个原因是打开着的文件消耗系统的资源。如果程序运行的时间很长,开开了很多文件但是并没有关闭它们,那么程序很可能会因为资源枯竭而
导致崩溃。

\subsubsection*{Seek 与 Tell}

从一个关联了磁盘文件的\acode{Handle}读写时,操作系统会维护一个关于当前位置的内部记录。每当进行下一层读取的时,操作系统返回下一个
数据块,其起始点为当前位置,且增加的位置反应了读取了多少数据。

你可以使用\acode{hTell}来找到文件中当前的位置。当文件被创建时,它是空的且你的位置将会是 0。在编写 5 个字节后,你的位置则变为了
5,以此类推。\acode{hTell}接受一个\acode{Handle}并返回一个\acode{IO Integer}来表示你的位置。

\acode{hTell}的同伴是\acode{hSeek},它允许你更改文件的位置,其三个参数为:\acode{Handle},\acode{SeekMode}以及一个位置。

\acode{SeekMode}有三个不同类型,用于指定如何解析给定的位置。\acode{AbsoluteSeek}为文件中的精确位置,等同于\acode{hTell};
\acode{RelativeSeek}意为当前位置开始的多少位置,正值向后,负值向前;\acode{SeekFromEnd}则是从文件末尾往前多少的位置。

并不是所有的\acode{Handle}都是可以 seek 的。一个\acode{Handle}通常关联一个文件,但是它还可以关联其它的东西,比如网络连接,
磁带机,或者终端。可以使用\acode{hIsSeekable}来查看给定的\acode{Handle}是否可以 seek。

\subsubsection*{标准输入,输出以及错误}

较早之前我们指出每个非“h”函数通常都会有与其关联的“h”函数用作于任何\acode{Handle}上。实际上非“h”函数只不过是“h”函数的一种缩写。

在\acode{System.IO}中有三个预定义的\acode{Handle}:\acode{stdin}标准输入,通常是键盘;\acode{stdout}标准输出,通常是
显示器;\acode{stderr}标准错误,通常也是显示器。

像是\acode{getLine}类似的函数可以简单的定义成:

\begin{lstlisting}[language=Haskell]
  getLine = hGetLine stdin
  putStrLn = hPutStrLn stdout
  print = hPrint stdout
\end{lstlisting}

\subsubsection*{删除与文件重命名}

\acode{System.Directory}模块提供了两个比较有用的函数:\acode{removeFile}接受单个参数,即文件名,然后删除该文件;
\acode{renameFile}接受两个文件名,一个旧名称以及一个新的名称,如果新的文件名是一个不同的路径,那么可以认为这是一个移动。
旧的文件名必须在调用\acode{renameFile}之前就存在,另外如果新文件已经存在,则重命名后将其删除。

跟很多其它接受一个文件名的函数一样,如果“旧”名称不存在,\acode{renameFile}则会抛出异常。

\acode{System.Directory}模块中还有很多其他的函数,像是创建或移除文件夹,在路径中查找文件列表,测试文件是否存在,等等。

\subsubsection*{临时文件}

程序员频繁的需要临时文件。这些文件被用于存储大量等待计算的数据,可被其它程序所用的数据,等等。

通过名为\acode{openTempFile}的函数(以及相关联的\acode{openBinaryTempFile})可以帮助我们解决不少问题。

\acode{openTempFile}接受两个参数:创建文件的路径,以及一个“template”用于命名文件。路径可以是\acode{.}来代表当前路径,或者
也可以使用\acode{System.Directory.getTemporaryDirectory}来找到操作系统所给出的最佳放置临时文件的位置。template 则将一些
随机字符添加至文件,用以确保结果是真正唯一的。实际上它保证了可以在一个唯一的文件名上工作。

\acode{openTempFile}的返回类型是\acode{IO (FilePath, Handle)}。元组的第一部分是被创建文件的名称,第二部分则是一个模式为
\acode{ReadWriteMode}的打开了文件的\acode{Handle}。当我们使用完文件,我们希望用\acode{hClose}来操作\acode{Handle}用于
关闭文件,接着调用\acode{removeFile}来删除它。接下来我们会看到一个例子用于展示这些函数的用法。

\subsection*{扩展案例:函数式 I/O 以及临时文件}

% TODO

\end{document}
