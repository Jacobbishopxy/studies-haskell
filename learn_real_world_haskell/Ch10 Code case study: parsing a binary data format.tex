\documentclass[./main.tex]{subfiles}

\begin{document}

本章我们将要探讨一个通用任务:解析一个二进制文件。使用该任务有两个目的,第一是借助解析来探讨程序的规划,重构,以及“模板化代码移除”。我们将
展示如何简化重复的代码,并为我们在第 14 章 Monads 中做准备。

我们将使用的文件格式源自 netpbm 套件,这是一个古老而可敬的用于处理位图图像的程序和文件格式集合。这些文件格式具有广泛的时候以及相当容易
解析的双重有点。更重要的是为了方便起见,netpbm 文件没有被压缩。

\subsection*{灰度文件}

netpbm 的灰度文件格式为 PGM(“portable grey map”)。它有两种格式构;“plain”(或“P2”)格式是由 ASCII 编码的,而更常用的“raw”
(“P5”)则是二进制格式。

两种格式的文件都有 header,即一个用于描述格式的“魔力”字符串。对于一个 plain 文件而言,字符串为\acode{P2};raw 文件则是\acode{P5}。
紧随字符串后的则是一个空格然后再是三个数字:图片的宽度,长度,以及最大灰度。这些数字皆为 ASCII 小数,由空格分隔。

接下来的就是图片数据。raw 文件中是二进制的字符串,plain 文件中则是由单空格字符分隔的 ASCII 小数构成。

raw 文件可以包含一系列的图片,一张接着一张,每张由 header 开头;plain 文件仅包含一张图片。

\subsection*{解析一个原始 PGM 文件}

我们的第一个解析函数仅需考虑 raw PGM 文件,同时该解析函数是一个\textit{纯}函数。该函数并不对数据获取负责,仅仅用作于解析。这在
Haskell 中是一个常规操作。通过分离数据读取,我们获得了更加灵活的操作空间。

我们将使用\acode{ByteString}类型来存储 greymap 数据。由于 PGM 文件的头部是 ASCII 文本,而本体是二进制的,我们需要同时引用文本
以及二进制的\acode{ByteString}模块。

\begin{lstlisting}[language=Haskell]
  import qualified Data.ByteString.Lazy as L
  import qualified Data.ByteString.Lazy.Char8 as L8
  import Data.Char (isSpace)
\end{lstlisting}

我们将使用一个直接的数据类型来表示 PGM 图片。

\begin{lstlisting}[language=Haskell]
  data Greymap = Greymap
  { greyWidth :: Int,
    greyHeight :: Int,
    greyMax :: Int,
    greyData :: L.ByteString
  }
  deriving (Eq)
\end{lstlisting}

通常而言,Haskell 的\acode{Show}实例应该生成一个字符串作为展示,同时我们可以通过\acode{read}将其从字符串中转回。然而对于 bitmap
图像文件而言,这会潜在的生产出巨大的文本字符串,例如对一个图片执行\acode{show}。因此我们不会让编译器自动派生一个\acode{Show}实例:
我们编写自己的实现,并将其简化。

\begin{lstlisting}[language=Haskell]
  instance Show Greymap where
    show (Greymap w h m _) = "Greymap " ++ show w ++ "x" ++ show h ++ " " ++ show m
\end{lstlisting}

因为我们的\acode{Show}实例是特意避开了打印 bitmap 数据,因此我们无法编写一个\acode{Read}实例,毕竟我们不能通过\acode{show}的
结果来重新构建一个合法的\acode{Greymap}。

接下来是我们解析函数的类型:

\begin{lstlisting}[language=Haskell]
  parseP5 :: L.ByteString -> Maybe (Greymap, L.ByteString)
\end{lstlisting}

该函数接受一个\acode{ByteString},如果解析成功则返回一个单独解析好的\acode{Greymap},以及剩下还需要解析的字符串。

我们的解析函数需要需要花费一点时间。首先是确保输入的文件是 raw PGM;接着解析文件头剩下的数值;然后再是解析 bitmap 数据。下面是一个
浅显的表达方式:

\begin{lstlisting}[language=Haskell]
  matchHeader :: L.ByteString -> L.ByteString -> Maybe L.ByteString
  matchHeader = undefined

  getNat :: L.ByteString -> Maybe (Int, L.ByteString)
  getNat = undefined

  getBytes :: Int -> L.ByteString -> Maybe (L.ByteString, L.ByteString)
  getBytes = undefined

  -- parse function
  parseP5 :: L.ByteString -> Maybe (Greymap, L.ByteString)
  parseP5 s =
    case matchHeader (L8.pack "PS") s of
      Nothing -> Nothing
      Just s1 ->
        case getNat s1 of
          Nothing -> Nothing
          Just (width, s2) ->
            case getNat (L8.dropWhile isSpace s2) of
              Nothing -> Nothing
              Just (height, s3) ->
                case getNat (L8.dropWhile isSpace s3) of
                  Nothing -> Nothing
                  Just (maxGrey, s4)
                    | maxGrey > 255 -> Nothing
                    | otherwise ->
                        case getBytes 1 s4 of
                          Nothing -> Nothing
                          Just (_, s5) ->
                            case getBytes (width * height) s5 of
                              Nothing -> Nothing
                              Just (bitmap, s6) ->
                                Just (Greymap width height maxGrey bitmap, s6)
\end{lstlisting}

这非常字面化的代码将所有的解析放在了一个长的阶梯型的\acode{case}表达式上。每个函数在消费完其所需的字符串后,都会返回剩下的
\acode{ByteString}。接着解构每个结果,如果解析失败则返回\acode{Nothing}。下面是解析过程中所用到的函数:

\begin{lstlisting}[language=Haskell]
  matchHeader :: L.ByteString -> L.ByteString -> Maybe L.ByteString
  matchHeader prefix str
    | prefix `L8.isPrefixOf` str = Just (L8.dropWhile isSpace (L.drop (L.length prefix) str))
    | otherwise = Nothing

  getNat :: L.ByteString -> Maybe (Int, L.ByteString)
  getNat s = case L8.readInt s of
    Nothing -> Nothing
    Just (num, rest)
      | num <= 0 -> Nothing
      | otherwise -> Just (fromIntegral num, rest)

  getBytes :: Int -> L.ByteString -> Maybe (L.ByteString, L.ByteString)
  getBytes n str =
    let count = fromIntegral n
        both@(prefix, _) = L.splitAt count str
     in if L.length prefix < count
          then Nothing
          else Just both
\end{lstlisting}

% TODO

\subsection*{移除样板代码}

\subsection*{隐式状态}

\subsubsection*{唯一性解析器}

\subsubsection*{Record 语义,更新,以及模式匹配}

\subsubsection*{一个更有趣的解析器}

\subsubsection*{获取与修改解析状态}

\subsubsection*{报告解析异常}

\subsubsection*{链起所有解析器}

\subsection*{函子简介}

\subsubsection*{类型定义上的约束并不好}

\subsubsection*{fmap 的中缀使用}

\subsubsection*{灵活的实例}

\subsubsection*{更多有关函子}

\subsection*{为解析编写函子实例}

\subsection*{为解析使用函子}

\subsection*{重写 PGM 解析器}

\subsection*{未来的方向}

\end{document}
