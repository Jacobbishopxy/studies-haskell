\documentclass[./main.tex]{subfiles}

\begin{document}

本章我们将要探讨一个通用任务:解析一个二进制文件。使用该任务有两个目的,第一是借助解析来探讨程序的规划,重构,以及“模板化代码移除”。
我们将展示如何简化重复的代码,并为我们在第 14 章 Monads 中做准备。

我们将使用的文件格式源自 netpbm 套件,这是一个古老而可敬的用于处理位图图像的程序和文件格式集合。这些文件格式具有广泛的时候以及容易
解析的双重有点。更重要的是为了方便起见,netpbm 文件没有被压缩。

\subsection*{灰度文件}

netpbm 的灰度文件格式为 PGM(“portable grey map”)。它有两种格式构;“plain”(或“P2”)格式是由 ASCII 编码的,而更常用的“raw”
(“P5”)则是二进制格式。

两种格式的文件都有 header,即一个用于描述格式的“魔力”字符串。对于一个 plain 文件而言,字符串为\acode{P2};raw 文件则是\acode{P5}。
紧随字符串后的则是一个空格然后再是三个数字:图片的宽度,长度,以及最大灰度。这些数字皆为 ASCII 小数,由空格分隔。

接下来的就是图片数据。raw 文件中是二进制的字符串,plain 文件中则是由单空格字符分隔的 ASCII 小数构成。

raw 文件可以包含一系列的图片,一张接着一张,每张由 header 开头;plain 文件仅包含一张图片。

\subsection*{解析一个原始 PGM 文件}

我们的第一个解析函数仅需考虑 raw PGM 文件,同时该解析函数是一个\textit{纯}函数。该函数并不对数据获取负责,仅仅用作于解析。这在
Haskell 中是一个常规操作。通过分离数据读取,我们获得了更加灵活的操作空间。

我们将使用\acode{ByteString}类型来存储 greymap 数据。由于 PGM 文件的头部是 ASCII 文本,而本体是二进制的,我们需要同时引用文本
以及二进制的\acode{ByteString}模块。

\begin{lstlisting}[language=Haskell]
  import qualified Data.ByteString.Lazy as L
  import qualified Data.ByteString.Lazy.Char8 as L8
  import Data.Char (isSpace)
\end{lstlisting}

我们将使用一个直接的数据类型来表示 PGM 图片。

\begin{lstlisting}[language=Haskell]
  data Greymap = Greymap
  { greyWidth :: Int,
    greyHeight :: Int,
    greyMax :: Int,
    greyData :: L.ByteString
  }
  deriving (Eq)
\end{lstlisting}

通常而言,Haskell 的\acode{Show}实例应该生成一个字符串作为展示,同时我们可以通过\acode{read}将其从字符串中转回。然而对于 bitmap
图像文件而言,这会潜在的生产出巨大的文本字符串,例如对一个图片执行\acode{show}。因此我们不会让编译器自动派生一个\acode{Show}实例:
我们编写自己的实现,并将其简化。

\begin{lstlisting}[language=Haskell]
  instance Show Greymap where
    show (Greymap w h m _) = "Greymap " ++ show w ++ "x" ++ show h ++ " " ++ show m
\end{lstlisting}

因为我们的\acode{Show}实例是特意避开了打印 bitmap 数据,因此我们无法编写一个\acode{Read}实例,毕竟我们不能通过\acode{show}的
结果来重新构建一个合法的\acode{Greymap}。

接下来是我们解析函数的类型:

\begin{lstlisting}[language=Haskell]
  parseP5 :: L.ByteString -> Maybe (Greymap, L.ByteString)
\end{lstlisting}

该函数接受一个\acode{ByteString},如果解析成功则返回一个单独解析好的\acode{Greymap},以及剩下还需要解析的字符串。

我们的解析函数需要需要花费一点时间。首先是确保输入的文件是 raw PGM;接着解析文件头剩下的数值;然后再是解析 bitmap 数据。下面是一个
浅显的表达方式:

\begin{lstlisting}[language=Haskell]
  matchHeader :: L.ByteString -> L.ByteString -> Maybe L.ByteString
  matchHeader = undefined

  getNat :: L.ByteString -> Maybe (Int, L.ByteString)
  getNat = undefined

  getBytes :: Int -> L.ByteString -> Maybe (L.ByteString, L.ByteString)
  getBytes = undefined

  -- parse function
  parseP5 :: L.ByteString -> Maybe (Greymap, L.ByteString)
  parseP5 s =
    case matchHeader (L8.pack "PS") s of
      Nothing -> Nothing
      Just s1 ->
        case getNat s1 of
          Nothing -> Nothing
          Just (width, s2) ->
            case getNat (L8.dropWhile isSpace s2) of
              Nothing -> Nothing
              Just (height, s3) ->
                case getNat (L8.dropWhile isSpace s3) of
                  Nothing -> Nothing
                  Just (maxGrey, s4)
                    | maxGrey > 255 -> Nothing
                    | otherwise ->
                        case getBytes 1 s4 of
                          Nothing -> Nothing
                          Just (_, s5) ->
                            case getBytes (width * height) s5 of
                              Nothing -> Nothing
                              Just (bitmap, s6) ->
                                Just (Greymap width height maxGrey bitmap, s6)
\end{lstlisting}

这非常字面化的代码将所有的解析放在了一个长的阶梯型的\acode{case}表达式上。每个函数在消费完其所需的字符串后,都会返回剩下的
\acode{ByteString}。接着解构每个结果,如果解析失败则返回\acode{Nothing}。下面是解析过程中所用到的函数:

\begin{lstlisting}[language=Haskell]
  matchHeader :: L.ByteString -> L.ByteString -> Maybe L.ByteString
  matchHeader prefix str
    | prefix `L8.isPrefixOf` str = Just (L8.dropWhile isSpace (L.drop (L.length prefix) str))
    | otherwise = Nothing

  getNat :: L.ByteString -> Maybe (Int, L.ByteString)
  getNat s = case L8.readInt s of
    Nothing -> Nothing
    Just (num, rest)
      | num <= 0 -> Nothing
      | otherwise -> Just (fromIntegral num, rest)

  getBytes :: Int -> L.ByteString -> Maybe (L.ByteString, L.ByteString)
  getBytes n str =
    let count = fromIntegral n
        both@(prefix, _) = L.splitAt count str
     in if L.length prefix < count
          then Nothing
          else Just both
\end{lstlisting}

\subsection*{移除样板代码}

虽然\acode{parseP5}函数能工作,其形态并不令人满意。我们不停地重复\acode{Maybe}值的构建与解构,且仅在匹配\acode{Just}时继续。
所有的这些相似的\acode{case}表达式就像是“模板代码”那样。

我们可以看到有两个模式。首先是很多应用的函数都有相同的类型,接受\acode{ByteString}作为其最后一个参数,并返回\acode{Maybe}。
其次\acode{parseP5}函数中的每一步“阶梯”都会解构一个\acode{Maybe}值,要么失败或是传递未解包的结果给一个函数。

我们可以轻易地用一个函数捕获第二个模式。

\begin{lstlisting}[language=Haskell]
  (>>?) :: Maybe a -> (a -> Maybe b) -> Maybe b
  Nothing >>? _ = Nothing
  Just v >>? f = f v
\end{lstlisting}

\acode{>>?}函数的作用非常简单:它接受一个值作为其右参,以及一个函数作为其左参。如果值非\acode{Nothing},那么便将函数应用在
\acode{Just}值上。我们定义了作为一个操作符的函数,以此可以链接所有函数在一起。最后我们并未提供给\acode{(>>?)}一个优先级声明,
那么默认为\acode{infixl 9}(左结合,最强的操作符优先级)。换言之,\acode{a >>? b >>? c}将会从左至右进行计算,类似于
\acode{(a >>? b) >>? c}。

有了这个链接函数,我们可以再尝试一下编写我们的解析函数。

\begin{lstlisting}[language=Haskell]
  parseP5_take2 :: L.ByteString -> Maybe (Greymap, L.ByteString)
  parseP5_take2 s =
    matchHeader (L8.pack "P5") s
      >>? \s ->
        skipSpace ((), s)
          >>? (getNat . snd)
          >>? skipSpace
          >>? \(width, s) ->
            getNat s
              >>? skipSpace
              >>? \(height, s) ->
                getNat s
                  >>? \(maxGrey, s) ->
                    getBytes 1 s
                      >>? (getBytes (width * height) . snd)
                      >>? \(bitmap, s) -> Just (Greymap width height maxGrey bitmap, s)

  skipSpace :: (a, L.ByteString) -> Maybe (a, L.ByteString)
  skipSpace (a, s) = Just (a, L8.dropWhile isSpace s)
\end{lstlisting}

理解这个函数的关键在于链条的思考。每个\acode{(>>?)}的左侧总是一个\acode{Maybe}值;右侧则是一个返回\acode{Maybe}值的函数。
那么表达式的两侧类型都是\acode{Maybe},即满足下一个\acode{(>>?)}表达式。

另一个增加可读性的工作就是添加了\acode{skipSpace}函数。

\subsection*{隐式状态}

我们的代码仍然在显式的传递一对一对的值,第一个元素作为解析过程中的结果,第二个元素则是当前剩余的\acode{ByteString}。当我们想要
扩展代码时,例如追踪消费了多少字节数量以便报告解析异常位置的时候,我们需要修改八个不同的地方,仅仅是为了传递一个三元组。

即使在少量代码的情况下,这样的需求都会使得代码难以修改。这个问题存在于使用模式匹配从每个元组中提取值时:我们默认了在代码中直接使用
元组。

让我们讲解一下新代码为何是不灵活的。首先修改解析器使用的状态类型。

\begin{lstlisting}[language=Haskell]
  data ParseState = ParseState
  { string :: L.ByteString,
    offset :: Int64
  }
  deriving (Show)
\end{lstlisting}

在我们新的代数数据类型中,我们同时拥有了追踪剩余字符串以及当前偏移量的能力。最重要的变化还是使用了 record 语义:现在可以\textit{避免}
对状态的模式匹配了,而是使用访问函数\acode{string}以及\acode{offset}。

我们给与需要解析的状态一个名称。当我们为某物命名时,它可以变得更有意义。例如,我们可以将解析视为一种函数:其消费一个解析状态,并同时
生产一个新的解析状态,以及一些额外的信息。我们可以直接将其视为一个 Haskell 类型。

\begin{lstlisting}[language=Haskell]
  simpleParse :: ParseState -> (a, ParseState)
  simpleParse = undefined
\end{lstlisting}

为了更好的帮助用户,我们可以在解析失败时报告异常信息。这仅仅需要我们解析器一点小小的改动。

\begin{lstlisting}[language=Haskell]
  betterParse :: ParseState -> Either String (a, ParseState)
  betterParse = undefined
\end{lstlisting}

之前显式的使用状态元组时,要扩展解析器的时候我们很快就发现问题了。为了避免重复,我们将使用\acode{newtype}声明来隐去解析类型的细节。

\begin{lstlisting}[language=Haskell]
  newtype Parse a = Parse
  { runParse :: ParseState -> Either String (a, ParseState)
  }
\end{lstlisting}

记住\acode{newtype}定义就是一个编译时的包装函数,因此它没有任何的运行时负荷。当我们使用该函数时,我们将应用\acode{runParser}
访问函数。

如果我们不从模块中导出\acode{Parse}值,我们可以确保他人不会意外的创建一个解析器,也避免了通过模式匹配来查看内部实现。

\subsubsection*{唯一性解析器}

让我们尝试定义一个简单的解析器,\textit{identity}解析器。它的功能便是将任何传递进其的参数转换为解析器的结果。这种情况下,它更像是
\acode{id}函数。

\begin{lstlisting}[language=Haskell]
  identity :: a -> Parse a
  identity a = Parse (\s -> Right (a, s))
\end{lstlisting}

该函数不会改变解析状态,并使用其参数作为解析的结果。我们将函数体包装成\acode{Parse}类型来满足类型检查器。那么我们该如何使用该包装后
的函数来进行解析呢?

首先我们通过\acode{runParse}函数去除\acode{Parse}的包装,以此获取其中的函数。接着构造一个\acode{ParseState},并以此运行我们的
解析函数。最后则是将解析的结果从最终的\acode{ParseState}中分离。

\begin{lstlisting}[language=Haskell]
  parse :: Parse a -> L.ByteString -> Either String a
  parse parser initState =
    case runParse parser (ParseState initState 0) of
      Left err -> Left err
      Right (result, _) -> Right result
\end{lstlisting}

由于\acode{identity}解析器与\acode{parse}函数都不会测试状态,我们甚至不需要创建一个字符串输入来进行测试。

\begin{lstlisting}[language=Haskell]
  ghci> :l Parse.hs
  [1 of 1] Compiling Main             ( Parse.hs, interpreted )
  Ok, one module loaded.
  ghci> :t parse (identity 1) undefined
  parse (identity 1) undefined :: Num a => Either String a
  ghci> parse (identity 1) undefined
  Right 1
  ghci> parse (identity "foo") undefined
  Right "foo"
\end{lstlisting}

解析器甚至不会检查其不感兴趣的输入,之后我们将见识到其有用之处。

\subsubsection*{Record 语义,更新,以及模式匹配}

\subsubsection*{一个更有趣的解析器}

\subsubsection*{获取与修改解析状态}

\subsubsection*{报告解析异常}

\subsubsection*{链起所有解析器}

\subsection*{函子简介}

\subsubsection*{类型定义上的约束并不好}

\subsubsection*{fmap 的中缀使用}

\subsubsection*{灵活的实例}

\subsubsection*{更多有关函子}

\subsection*{为解析编写函子实例}

\subsection*{为解析使用函子}

\subsection*{重写 PGM 解析器}

\subsection*{未来的方向}

\end{document}
