\documentclass[./main.tex]{subfiles}

\begin{document}

本章我们将要探讨一个通用任务:解析一个二进制文件。使用该任务有两个目的,第一是借助解析来探讨程序的规划,重构,以及“模板化代码移除”。我们将
展示如何简化重复的代码,并为我们在第 14 章 Monads 中做准备。

我们将使用的文件格式源自 netpbm 套件,这是一个古老而可敬的用于处理位图图像的程序和文件格式集合。这些文件格式具有广泛的时候以及相当容易
解析的双重有点。更重要的是为了方便起见,netpbm 文件没有被压缩。

\subsection*{灰度文件}

netpbm 的灰度文件格式为 PGM(“portable grey map”)。它有两种格式构;“plain”(或“P2”)格式是由 ASCII 编码的,而更常用的“raw”
(“P5”)则是二进制格式。

两种格式的文件都有 header,即一个用于描述格式的“魔力”字符串。对于一个 plain 文件而言,字符串为\acode{P2};raw 文件则是\acode{P5}。
紧随字符串后的则是一个空格然后再是三个数字:图片的宽度,长度,以及最大灰度。这些数字皆为 ASCII 小数,由空格分隔。

接下来的就是图片数据。raw 文件中是二进制的字符串,plain 文件中则是由单空格字符分隔的 ASCII 小数构成。

raw 文件可以包含一系列的图片,一张接着一张,每张由 header 开头;plain 文件仅包含一张图片。

\subsection*{解析一个原始 PGM 文件}

\subsection*{移除样板代码}

\subsection*{隐式状态}

\subsubsection*{唯一性解析器}

\subsubsection*{Record 语义,更新,以及模式匹配}

\subsubsection*{一个更有趣的解析器}

\subsubsection*{获取与修改解析状态}

\subsubsection*{报告解析异常}

\subsubsection*{链起所有解析器}

\subsection*{函子简介}

\subsubsection*{类型定义上的约束并不好}

\subsubsection*{fmap 的中缀使用}

\subsubsection*{灵活的实例}

\subsubsection*{更多有关函子}

\subsection*{为解析编写函子实例}

\subsection*{为解析使用函子}

\subsection*{重写 PGM 解析器}

\subsection*{未来的方向}

\end{document}
