\documentclass[./main.tex]{subfiles}

\begin{document}

\subsection*{关联列表}

通常我们需要处理由键索引的无序数据。例如 Unix 管理员会有一系列的数值 UIDs 以及其关联的用户名。该列表的价值在于能够查找给定 UID 的用户名,而不是按照数据的顺序。
换言之,UID 是数据库的键。

在 Haskell 中,有若干方法可以处理类似这样数据的结构。两个最常见的就是关联列表以及由\acode{Data.Map}模块提供的\acode{Map}类型。关联列表很方便,因为它们很简单。
它们就是 Haskell 的列表,因此所有熟悉的列表函数同样可用于关联列表。然而对于较大的数据集,\acode{Map}则比关联列表具有更大的性能优势。

关联列表就是一个包含了(键,值)元组的普通列表。对于 UID 映射至用户名的类型就是\acode{[(Integer, String)]}。

Haskell 拥有一个称为\acode{Data.List.lookup}的内建函数用于关联列表的查询。其类型为\acode{Eq a => a -> [(a, b)] -> Maybe b}。

\begin{lstlisting}[language=Haskell]
  ghci> let al = [(1,"one"),(2,"two"),(3,"three"),(4,"four")]
  ghci> lookup 1 al
  Just "one"
  ghci> lookup 5 al
  Nothing
\end{lstlisting}

\acode{lookup}函数实际很简单,我们可以自己编写一个:

\begin{lstlisting}[language=Haskell]
  myLookup :: (Eq a) => a -> [(a, b)] -> Maybe b
  myLookup _ [] = Nothing
  myLookup key ((thisKey, thisVal) : rest) =
    if key == thisKey
      then Just thisVal
      else myLookup key rest
\end{lstlisting}

让我们来看一个更复杂的关联列表。在 Unix/Linux 机器中存在一个名为\acode{/etc/passwd}的文件用于存储用户名,UIDs,home 路径,已经其它数据。我们将编写一个程序用于
解析该文件,创建一个关联列表,令用户通过给定的 UID 来查找用户名。

\begin{lstlisting}[language=Haskell]
  import Control.Monad (when)
  import Data.List
  import System.Environment (getArgs)
  import System.Exit
  import System.IO

  main = do
    -- Load the command-line arguments
    args <- getArgs

    -- If we don't have the right amount of args, give an error and abort
    when (length args /= 2) $ do
      putStrLn "Syntax: passwd-al filename uid"
      exitFailure

    -- Read the file lazily
    content <- readFile $ head args

    -- Compute the username in pure code
    let username = findByUID content $ read $ args !! 1

    -- Display the result
    case username of
      Just x -> putStrLn x
      Nothing -> putStrLn "Could not find that UID"

    putStrLn "whatever"

  -- Given the entire input and a UID, see if we can find a username.
  findByUID :: String -> Integer -> Maybe String
  findByUID content uid =
    let al = map parseLine . filter (('#' /=) . head) . lines $ content
     in lookup uid al

  -- Convert a colon-separated line into fields
  parseLine :: String -> (Integer, String)
  parseLine input =
    let fields = split ':' input
     in (read (fields !! 2), head fields)

  -- Takes a delimiter and a list. Break up the list based on the delimiter.
  split :: (Eq a) => a -> [a] -> [[a]]
  -- If the input is empty, the result is a list of empty lists.
  split _ [] = [[]]
  split delim str =
    -- Find the part of the list before delim and put it in "before".
    -- The rest of the list, including the leading delim, goes in "remainder".
    let (before, remainder) = span (/= delim) str
     in before : case remainder of
          [] -> []
          -- If there is more data to precess,
          -- call split recursively to process it
          x ->
            split delim $ tail x
\end{lstlisting}

\textbf{注}:原文的\acode{findByUID}函数中的\acode{let al = map parseLine . lines \$ content}需要修改成
\acode{let al = map parseLine . filter (('#' /=) . head) . lines \$ content}才能对\acode{/etc/passwd}文件生效。否则会抛出异常
\acode{Prelude.!!: index too large}。

测试:

\begin{lstlisting}
  % runhaskell passwd-al.hs /etc/passwd 0
  root
  % runhaskell passwd-al.hs /etc/passwd 3
  Could not find that UID
\end{lstlisting}

\subsection*{映射}

\subsection*{函数也是数据}

\subsection*{拓展案列:/etc/passwd}

\subsection*{拓展案列:数值类型}

\subsection*{函数作为数据的优点}

\subsection*{通用序列}

\end{document}
