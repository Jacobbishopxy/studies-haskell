\documentclass[./main.tex]{subfiles}

\begin{document}

构建真实系统意味着需要考虑质量控制,鲁棒性以及正确性。使用正确的质量保证机制,良好的代码类似于一个各项功能完备的精确的仪器。

Haskell 中有着若干工具帮助我们构建这样精准的系统。最显著的当然是语言自带的类型系统,它允许静态的强制执行复杂的不变量 --
使得编写的代码无法违背其约束。除此之外,纯粹性以及多态鼓励模块化,可重构和可测试的代码风格。

在维护代码的正确表达性上测试担任了很重要的角色。Haskell 中主要的测试机制是传统的单元测试(通过 HUnit 库),以及其更强大的
派生品:基于类型的“属性”测试,即 QuickCheck 一个开源的 Haskell 测试框架。基于属性的测试鼓励更高层次的测试方法,是以抽象
不变函数的形式测试,需要满足测试库所生成真实的测试数据。相较于手写测试用例的不足,通过这样测试的代码通常能覆盖所有的边缘用例。

本章我们学习如何使用 QuickCheck 来构建代码的不变性,并测试上一章节编写的代码。另外还会学习如何使用 GHC 代码覆盖工具:HPC。

\subsection*{QuickCheck:基于类型的测试}

为了展示基于属性的测试是如何工作的,我们用一个简单的场景开始:用户编写了一个特殊的排序函数,想要测试其行为。

首先是导入 QuickCheck 库,正如其他所需的包那样(通过以下命令在\acode{.cabal}存在的项目下安装包
\acode{cabal install --lib QuickCheck}):

\begin{lstlisting}[language=Haskell]
  import Test.QuickCheck
  import Data.List
\end{lstlisting}

以及希望被测试的函数 -- 一个自定义的排序:

\begin{lstlisting}[language=Haskell]
  qsort :: (Ord a) => [a] -> [a]
  qsort [] = []
  qsort (x : xs) = qsort lhs ++ [x] ++ qsort rhs
    where
      lhs = filter (< x) xs
      rhs = filter (>= x) xs
\end{lstlisting}

这是一个典型的 Haskell 排序实现:学习函数式编程的优雅,不过没那么高效(这并不是一个替换排序)。现在我们希望检查该函数是否遵循
基本的排序规则。一个不错的不变性测试就是\textit{幂等性} -- 运行一个函数两次的返回是否一致。这个不变性的测试不难:

\begin{lstlisting}[language=Haskell]
  prop_idempotent xs = qsort (qsort xs) == qsort xs
\end{lstlisting}

我们将使用 QuickCheck 来转换带有\acode{prop_}的前缀测试属性,用于区别普通的代码。这个幂等性的属性被简单的编写为 Haskell
函数来表明任何输入的数据被排序后是否具有相等性。我们通过手动测试若干案例的方式来检查它是否合理:

\begin{lstlisting}[language=Haskell]
  ghci> :l QC-basics.hs
  [1 of 2] Compiling Main             ( QC-basics.hs, interpreted )
  Ok, one module loaded.
  ghci> prop_idempotent []
  True
  ghci> prop_idempotent [1,1,1,1]
  True
  ghci> prop_idempotent [1..100]
  True
  ghci> prop_idempotent [1,5,2,1,2,0,9]
  True
\end{lstlisting}

看起来不错,不过手动输入数据还是太麻烦了,同时也违背了函数式编程的高效性:让机器做这事儿!QuickCheck 库带有 Haskell 所有数据
类型的数据生成器可以用于自动化手动输入这个过程。通过类型系统所生成的(临时)随机数据,QuickCheck 使用\acode{Arbitrary}
typeclass 来表示的统一接口。QuickCheck 通常会隐藏数据生成的过程,不过我们仍然可以通过手动的方式运行生成器来获取 QuickCheck
所生成的数据分布。例如,生成一个随机布尔值列表。

\begin{lstlisting}[language=Haskell]
  ghci> generate arbitrary :: IO [Bool]
  [False,False,True,False,True,False,False]
\end{lstlisting}

\textbf{注}:新的 QuickCheck 的\acode{generate}的函数签名是\acode{generate :: Gen a -> IO a}。

QuickCheck 生成的测试数据就像是这样,并通过\acode{quickCheck}函数传递给所选择的属性。属性其本身的类型决定了数据生成器的使用。
\acode{quickCheck}接着检查属性是否满足所有生产的数据。

\begin{lstlisting}[language=Haskell]
  ghci> :t quickCheck
  quickCheck :: Testable prop => prop -> IO ()
  ghci> quickCheck (pro)
  product                   properFraction            propertyForAllShrinkShow
  prop_idempotent           property
  ghci> quickCheck (prop_idempotent :: [Integer] -> Bool)
  +++ OK, passed 100 tests.
\end{lstlisting}

对于生成的 100 个不同的列表,属性都可以适配。在开发测试时,看到每个测试用例中真实生成的数据是很有意义的。为了达到这个效果,可以使用
\acode{verboseCheck}函数来查看更详细的输出。

\subsubsection*{测试属性}

好的库包含了一系列正交的基础,它们彼此之间都有合理的关联。使用 QuickCheck 来指定函数间的关系,通过开发属性间合理交互的函数来帮助我们
找到好的库接口。QuickCheck 这时所扮演的是 API 的 “lint” 工具 -- 其提供了确保我们库 API 合理性的机器支持。

列表排序函数当然拥有若干有意思的属性,它们与其它的列表操作紧密关联。例如:排序列表中第一个元素总是最小的元素。我们可以在 Haskell 中
指定这一特性,通过\acode{List}库的\acode{minimum}函数:

\begin{lstlisting}[language=Haskell]
  prop_minimum xs = head (qsort xs) == minimum xs
\end{lstlisting}

不过会抛出异常:

\begin{lstlisting}[language=Haskell]
  ghci> quickCheck (prop_minimum :: [Integer] -> Bool)
  *** Failed! (after 1 test):
  Exception:
    Prelude.head: empty list
    CallStack (from HasCallStack):
      error, called at libraries/base/GHC/List.hs:1646:3 in base:GHC.List
      errorEmptyList, called at libraries/base/GHC/List.hs:85:11 in base:GHC.List
      badHead, called at libraries/base/GHC/List.hs:81:28 in base:GHC.List
      head, called at QC-basics.hs:21:19 in main:Main
  []
\end{lstlisting}

也就是这个属性只满足非空列表。感谢 QuickCheck 有着一整套属性使得我们可以更加精确的指定我们的不变,剔除不希望考虑到的值。对于非空列表,
我们真实想表达的是:\textit{如果}列表是非空的,\textit{那么}排序列表中的第一个元素是最小的。根据\acode{(==>)}隐式函数,在运行属性
前剔除非法的数据:

\begin{lstlisting}[language=Haskell]
  prop_minimum' xs = not (null xs) ==> head (qsort xs) == minimum xs
\end{lstlisting}

通过分离空列表,现在可以确认属性实际上成立:

\begin{lstlisting}[language=Haskell]
  ghci> quickCheck (prop_minimum' :: [Integer] -> Property)
  +++ OK, passed 100 tests; 18 discarded.
\end{lstlisting}

注意我们将属性的类型从简单的\acode{Bool}结果变为了更泛化的\acode{Property}类型(属性自身现在是一个函数,在测试前筛选非空列表,
而不是简单的一个布尔常数)。

我们现在可以通过其它应该满足的不变来完成排序函数的所有基础属性了:输出是有序的(每个元素都应该小于等于其前者);输出是输入的组合(通过
列表差异函数\acode{(\\\\)};最后被排序的元素是最大的元素;如果找到了两个不同列表的最小元素,那么将两个列表合并后再排序,该元素还是
第一个元素。这些属性可以表示为:

\begin{lstlisting}[language=Haskell]
  prop_ordered xs = ordered (qsort xs)
  where
    ordered [] = True
    ordered [x] = True
    ordered (x : y : xs) = x <= y && ordered (y : xs)

prop_permutation xs = permutation xs (qsort xs)
  where
    permutation xs ys = null (xs \\ ys) && null (ys \\ xs)

prop_maximum xs = not (null xs) ==> last (qsort xs) == maximum xs

prop_append xs ys =
  not (null xs)
    ==> not (null ys)
    ==> head (qsort (xs ++ ys))
    == min (minimum xs) (minimum ys)
\end{lstlisting}

\subsubsection*{测试模型}

另一个技巧就是测试模型的实现。可以将列表排序的实现绑定标准库中的排序函数,同时如果它们表现的一致,就可以得知自定义的排序的正确性了。

\begin{lstlisting}[language=Haskell]
  prop_sort_model xs = sort xs == qsort xs
\end{lstlisting}

这一类基于模型的测试异常强大。通常开发人员会有一个参考现实或原型,虽然不够高效,但却是正确的。这可以保存下来,并用于确保优化后的产品
代码符合这套参考。通过构建一套大型的基于模型的测试,并定期运行它们(例如在每次提交时),我们可以轻松的确保代码的准确性。大型 Haskell
项目通常绑定了与项目本身大小相当的属性套件,在每次更改时都要测试数千个不变量,保证代码符合规范,并确保能按要求运行。

\subsection*{测试用例:指定一个漂亮打印器}

在 Haskell 所构建的大型系统中中,为单独的函数测试它们的自然属性属于基础的构建模块。接下来我们将要学习更复杂的场景:为之前章节所编写的
漂亮打印库构建测试套装。

\subsubsection*{生成测试数据}

回忆一下漂亮打印是基于\acode{Doc}所构建的程序,它是一个代表着良好格式化后的文档的代数数据结构。

\begin{lstlisting}[language=Haskell]
  data Doc
    = Empty
    | Char Char
    | Text String
    | Line
    | Concat Doc Doc
    | Union Doc Doc
    deriving (Show, Eq)
\end{lstlisting}

库本身是实现的一系列函数用于构建与转换该文档类型的值,最后将完成的文档渲染成字符串。

QuickCheck 鼓励采用一种测试方法,开发人员指定的不变量可以满足任何数据类型。为了测试漂亮打印库,我们还需要输入数据源。利用 QuickCheck
中\acode{Arbitrary}类所提供的组合字套件来构建随机数据。该类提供了一个函数,\acode{arbitrary},用于生成每种类型的数据,通过它我们可以
为自定义的数据类型定义数据生成器。

\begin{lstlisting}[language=Haskell]
  class Arbitrary a where
    arbitrary :: Gen a
\end{lstlisting}

\textbf{注}:该 typeclass 由 QuickCheck 库提供。

需要注意的是生成器运行在一个\acode{Gen}环境中,由类型表示。这是一个简单的状态传递单子,用于隐藏贯穿代码的随机数生成器状态。我们会在
稍后的章节中讲解单子,现在我们只需要知道\acode{Gen}定义为单子,可以使用\acode{do}语义来编写可访问隐式随机数源的生成器。为了编写自定义
类型的生成器,我们使用库中定义的一组函数来引入新的随机值,并将它们粘合在一起用以构建我们需要的数据结构类型。关键函数的类型为:

\begin{lstlisting}[language=Haskell]
  elements :: [a] -> Gen a
  choose :: Random a => (a, a) -> Gen a
  oneof :: [Gen a ] -> Gen a
\end{lstlisting}

\textbf{注}:以上三个函数皆由 QuickCheck 库提供。

函数\acode{elements},接受一个列表,返回一个生成器,该生成器将随机返回该列表中的值。\acode{choose}与\acode{oneof}将在稍后用到。
通过它们我们可以为一个简单的类型编写生成器。例如:

\begin{lstlisting}[language=Haskell]
  data Ternary
    = Yes
    | No
    | Unknown
    deriving (Eq, Show)
\end{lstlisting}

为其实现\acode{Arbitrary} typeclass:

\begin{lstlisting}[language=Haskell]
  instance Arbitrary Ternary where
    arbitrary = elements [Yes, No, Unknown]
\end{lstlisting}

另外一种数据生成的方案就是先生成一种 Haskell 的基础类型,再转换成期望的实际类型。通过\acode{choose}来选择 0 至 2 的随机整数,然后再
映射成 ternary 值:

\begin{lstlisting}[language=Haskell]
  instance Arbitrary Ternary where
    arbitrary = do
      n <- choose (0, 2) :: Gen Int
      return $ case n of
        0 -> Yes
        1 -> No
        _ -> Unknown
\end{lstlisting}

对于简单的\textit{sum}类型,这种方法可以很好的工作,因为整数可以很好的映射到数据类型的构造函数上。对于\textit{product}类型
(例如结构或元组),我们需要为每个成员分别生成乘积(递归嵌套类型),再将成员进行合并。例如生成一对随机值:

\begin{lstlisting}[language=Haskell]
  instance (Arbitrary a, Arbitrary b) => Arbitrary (a, b) where
    arbitrary = do
      x <- arbitrary
      y <- arbitrary
      return (x, y)
\end{lstlisting}

\textbf{注}:上述代码同样由 QuickCheck 库提供。

现在来为所有\acode{Doc}类型的变体来编写一个生成器。我们需要将问题进行分解,首先是为每个类型生成随机构造函数,接着根据结果生成每个字段的
组件。这里最复杂的部分就是 union 以及 concatenation 变体了。

首先,我们需要编写一个实例来生成随机字符 -- QuickCheck 并没有默认的字符实例,因为有大量不同的文本编码,我们希望一个字符测试。我们无需考虑
文档中的文本内容,因此简单的字母以及标点符号的生成器就足够了(更丰富的生成器可以基于该生成器进行拓展):

\begin{lstlisting}[language=Haskell]
  newtype DocChar = DocChar Char

  fromDocChar :: DocChar -> Char
  fromDocChar (DocChar x) = x

  instance Arbitrary DocChar where
    arbitrary = elements (DocChar <$> ['A' .. 'Z'] ++ ['a' .. 'z'] ++ " ~!@#$%^&*()")
\end{lstlisting}

\textbf{注}:与原文不同在于使用了\acode{newtype},因为 QuickCheck 已经为\acode{Char}实现了\acode{Arbitrary}。这里\acode{(<\$>)}的
计算优先级是 4,而\acode{(++)}的优先级是 5,因此不需要额外的括号。

有了这些现在就可以编写文档的实例了,即通过枚举构造函数,并填充其字段。我们选择随机整数来表示所需要生成的文档变体,再根据结果进行分配。
生成 concat 与 union 文档节点时,只需要递归\acode{arbitrary},将类型推导交给\acode{Arbitrary}的实例本身:

\begin{lstlisting}[language=Haskell]
  instance Arbitrary Doc where
    arbitrary = do
      n <- choose (1, 6) :: Gen Int
      case n of
        1 -> return Empty
        2 -> do
          (x :: DocChar) <- arbitrary
          return (Char $ fromDocChar x)
        3 -> Text <$> arbitrary
        4 -> return Line
        5 -> do
          x <- arbitrary
          Concat x <$> arbitrary
        6 -> do
          x <- arbitrary
          Union x <$> arbitrary
\end{lstlisting}

\textbf{注}:与原文不同之处在于处理\acode{Char}的步骤,先是显式声明了\acode{x}为\acode{DocChar},然后再使用\acode{fromDocChar}取出
\acode{Char}值。

上述代码非常的直接,我们可以通过\acode{oneof}函数对其进行重写,该函数我们较早之前见过,其作用是在列表中挑选一个后生成生成器(我们也可以
使用单子化的组合子,用\acode{liftM}来避免命名生成器的各种中间结果):

\begin{lstlisting}[language=Haskell]
  instance Arbitrary Doc where
  arbitrary =
    oneof
      [ return Empty,
        fmap (Char . fromDocChar . DocChar) arbitrary,
        fmap Text arbitrary,
        return Line,
        liftM2 Concat arbitrary arbitrary,
        liftM2 Union arbitrary arbitrary
      ]
\end{lstlisting}

\textbf{注}:与原文不同之处在于列表第二个元素\acode{Char}的处理方式上,通过 newtype \acode{DocChar}的\acode{arbitrary}函数再提取出
\acode{Char}来实现调用非默认的\acode{Char} arbitrary。

上述两种对\acode{Doc}实现\acode{Arbitrary}实例的方式,后者更为简洁,仅仅是在生成器列表中进行选择,不过二种表达方式讲述的都是同一件事。我们
可以通过生成一个随机文档列表来进行测试

\begin{lstlisting}[language=Haskell]
  ghci> generate arbitrary :: IO [Doc]
  [Concat (Union (Text "1^_\FS\83431\nir^\DC1IH\t\GS\10583r\"_~t\1007943(\1091024\180519\ETX\53558y\998555") Line) (Union (Text "s4hxR(\1108184") Empty),Empty,Empty,Char '&',Char 'D',Empty,Line,Empty,Line,Char '>',Te
  xt "\99986\25864;'}\CAN",Concat Line Empty,Text "\NAK",Char '&',Union Empty (Concat (Char '\998638') (Union
   (Text "/\t\28481\1025790\95127\1078739x;D") (Concat (Char 'Q') Line))),Empty,Concat Line (Text "Gwj^v\1032477\EOT\58278k},`<\1016570\59993:Y>\1074673q;A \7956\127844"),Empty,Concat Empty (Text "\199725\SIl(<\176480\133060,A\FSu\EM>\DC1\SYN\1102841\RS\STX\DELim\n\"|\1106202De\n#"),Char 'J',Union Empty Line,Concat (Union
   (Concat Line (Char 'E')) (Char '\NUL')) (Concat (Text "78\152486\194828?vy\157411\\1*\DC1\NAK\NULrd\SYN:\t
  ") Line),Text ""]
\end{lstlisting}

\textbf{注}:与原文不同,\acode{generate}的函数签名变了。

结果是各种文档变体混合的列表。之后在每次测试运行时会生成上百个结果。现在可以为文档函数编写一些更泛化的属性了。

\subsubsection*{测试文档构建}

基于文档的两个基础函数是 null 文档常数(一个 nullary 函数),\acode{empty},以及 append 函数。它们的类型是:

\begin{lstlisting}[language=Haskell]
  empty :: Doc
  (<>) :: Doc -> Doc -> Doc
\end{lstlisting}

总之,它们有一个很好的属性:将空列表 append 或 prepend 到第二个列表上,第二个列表保持不变。我们可以将这个不变量声明为一个属性:

\begin{lstlisting}[language=Haskell]
  prop_empty_id x =
  empty <> x == x
    && x <> empty == x
\end{lstlisting}

\textbf{注}:原有 Doc 代码在第五章的\acode{Prettify.hs}文件中(改动后位于 code/Ch11/Prettify2.hs 文件下)。与第五章不一样的地方:
为\acode{Doc}实现了\acode{Semigroup},因此可以对\acode{Doc}使用操作符\acode{(<>)}。

为了确保它是正确的,测试一下:

\begin{lstlisting}[language=Haskell]
  ghci> quickCheck prop_empty_id
  00, passed 100 tests.
\end{lstlisting}

将\acode{quickCheck}改为\acode{verboseCheck}可以看到生成出来的测试文档。

其它的 API 测试:

\begin{lstlisting}[language=Haskell]
  prop_char c = char c == Char c

  prop_text s = text s == if null s then Empty else Text s

  prop_line = line == Line

  prop_double d = double d == text (show d)
\end{lstlisting}

\subsubsection*{以列表作模型}

高阶函数是重用程序的基本粘合剂,漂亮打印库也不例外 -- 自定义的 fold 函数用于内部实现文档拼接以及文档块之间的交错分隔符。文档的\acode{fold}
接受一个文档列表,通过提供的组合函数将它们粘合起来:

\begin{lstlisting}[language=Haskell]
  fold :: (Doc -> Doc -> Doc) -> [Doc] -> Doc
  fold f = foldr f empty
\end{lstlisting}

那么为 fold 编写测试就很简单了。

\begin{lstlisting}[language=Haskell]
  prop_hcat xs = hcat xs == glue xs
    where
      glue [] = empty
      glue (d : ds) = d <> glue ds
\end{lstlisting}

同理对于\acode{punctuate},我们可以通过列表散布(interspersion)的方式来插入标点符号:

\begin{lstlisting}[language=Haskell]
  import Data.List (intersperse)

  prop_punctuate s xs = punctuate s xs == intersperse s xs
\end{lstlisting}

然而对其进行测试时,我们会得到错误:

\begin{lstlisting}[language=Haskell]
  ghci> quickCheck prop_punctuate
  *** Failed! Falsified (after 6 tests and 1 shrink):
  Char ' '
  [Union (Text "\ESC") (Text "RO"),Union Empty (Char '\1097718')]
\end{lstlisting}

漂亮打印库优化掉了多余的空文档,这是模型实现无法做到的,因此我们需要增强模型以匹配实际情况。首先,可以在整个文档列表中散布标点符号文本,然后用
一个小的循环来清理分散在其中的\acode{Empty}文档,像是这样:

\begin{lstlisting}[language=Haskell]
  prop_punctuate' s xs = punctuate s xs == combine (intersperse s xs)
    where
      combine [] = []
      combine [x] = [x]
      combine (x : Empty : ys) = x : combine ys
      combine (Empty : y : ys) = y : combine ys
      combine (x : y : ys) = x `Concat` y : combine ys
\end{lstlisting}

通过 CHCi 来确认结果。

\begin{lstlisting}[language=Haskell]
  ghci> :l QC.hs
  [1 of 4] Compiling Prettify2        ( Prettify2.hs, interpreted )
  [2 of 4] Compiling Arbitrary        ( Arbitrary.hs, interpreted )
  [3 of 4] Compiling Main             ( QC.hs, interpreted )
  Ok, three modules loaded.
  ghci> quickCheck prop_punctuate'
  +++ OK, passed 100 tests.
\end{lstlisting}

\subsubsection*{将它们合在一起}

我们可以将所有的测试放到一个单独文件中,然后使用 QuickCheck 的派生函数来运行。基本的批处理驱动程序通常已经足够好了。

\begin{lstlisting}[language=Haskell]
  {-# LANGUAGE TemplateHaskell #-}
  import Test.QuickCheck.All

  -- All the prop tests before `return []`

  return []
  runTests = $quickCheckAll
\end{lstlisting}

\textbf{注}:根据\href{https://stackoverflow.com/a/42669557/22544285}{StackOverflow}的答案,QuickCheck 在 2.0 之后就没了
\acode{Test.QuickCheck.Batch},因此需要在\acode{QC.hs}文件的头尾加上上述代码。而新的\acode{Run.hs}代码如下:

\begin{lstlisting}[language=Haskell]
  import QC

  main = do
    runTests
    return ()
\end{lstlisting}

通过 GHCi 测试:

\begin{lstlisting}[language=Haskell]
  ghci> :l Run.hs
  [1 of 5] Compiling Prettify2        ( Prettify2.hs, interpreted )
  [2 of 5] Compiling Arbitrary        ( Arbitrary.hs, interpreted )
  [3 of 5] Compiling QC               ( QC.hs, interpreted )
  [4 of 5] Compiling Main             ( Run.hs, interpreted )
  Ok, four modules loaded.
  ghci> main
  === prop_empty_id from ./QC.hs:14 ===
  +++ OK, passed 100 tests.

  === prop_char from ./QC.hs:18 ===
  +++ OK, passed 100 tests.

  === prop_text from ./QC.hs:20 ===
  +++ OK, passed 100 tests.

  === prop_line from ./QC.hs:22 ===
  +++ OK, passed 1 test.

  === prop_double from ./QC.hs:24 ===
  +++ OK, passed 100 tests.

  === prop_hcat from ./QC.hs:28 ===
  +++ OK, passed 100 tests.

  === prop_punctuate from ./QC.hs:33 ===
  *** Failed! Falsified (after 4 tests):
  Char 'H'
  [Concat (Char '1') (Concat (Char '\119225') (Concat (Char 'r') Empty)),Text ""]

  === prop_punctuate' from ./QC.hs:35 ===
  +++ OK, passed 100 tests.
\end{lstlisting}

我们在这里将代码构建为一个独立的测试脚本,实例和属性在它们自己的文件中,做到了与库的源代码分离。这是一个典型的库项目,测试与库本身分开,
并通过模块系统导入库。然后可以编译和执行测试脚本:

\begin{lstlisting}[language=Haskell]
  ghc --make Run.hs
  Loaded package environment from <xxx>
  [1 of 5] Compiling Prettify2        ( Prettify2.hs, Prettify2.o, Prettify2.dyn_o )
  [2 of 5] Compiling Arbitrary        ( Arbitrary.hs, Arbitrary.o, Arbitrary.dyn_o )
  [3 of 5] Compiling QC               ( QC.hs, QC.o, QC.dyn_o )
  [4 of 5] Compiling Main             ( Run.hs, Run.o )
  [5 of 5] Linking Run
\end{lstlisting}

接着运行:

\begin{lstlisting}[language=Haskell]
  ./Run
  === prop_empty_id from ./QC.hs:14 ===
  +++ OK, passed 100 tests.

  === prop_char from ./QC.hs:18 ===
  +++ OK, passed 100 tests.

  === prop_text from ./QC.hs:20 ===
  +++ OK, passed 100 tests.

  === prop_line from ./QC.hs:22 ===
  +++ OK, passed 1 test.

  === prop_double from ./QC.hs:24 ===
  +++ OK, passed 100 tests.

  === prop_hcat from ./QC.hs:28 ===
  +++ OK, passed 100 tests.

  === prop_punctuate from ./QC.hs:33 ===
  *** Failed! Falsified (after 4 tests):
  Union (Char '\1089590') (Char 't')
  [Line,Text "\129350\167965\EOT"]

  === prop_punctuate' from ./QC.hs:35 ===
  +++ OK, passed 100 tests.
\end{lstlisting}

\subsection*{通过 HPC 测量测试覆盖率}

HPC (Haskell Program Coverage) 是编译器的扩展,用于观察给定程序运行期间实际执行的代码部分。这在测试的上下文中很有用,因为它可以
让我们精确的观察哪些函数、分支以及表达式被求值。HPC 附带了一个简单的工具来生成有用的程序覆盖图,这可以放大测试套件中的弱点。

我们需要在编译时添加\acode{-fhpc}标记:

\begin{lstlisting}[language=Haskell]
  ghc -fhpc Run.hs --make
  Loaded package environment from /Users/jacobxie/.ghc/aarch64-darwin-9.4.7/environments/default
  [1 of 5] Compiling Prettify2        ( Prettify2.hs, Prettify2.o, Prettify2.dyn_o ) [HPC flags changed]
  [2 of 5] Compiling Arbitrary        ( Arbitrary.hs, Arbitrary.o, Arbitrary.dyn_o ) [HPC flags changed]
  [3 of 5] Compiling QC               ( QC.hs, QC.o, QC.dyn_o ) [Source file changed]
  [4 of 5] Compiling Main             ( Run.hs, Run.o ) [HPC flags changed]
  [5 of 5] Linking Run [Objects changed]
\end{lstlisting}

在测试运行期间,程序的跟踪记录被写入至当前目录下的 .tix 与 .mix 文件中。然后命令行工具 hpc 使用这些文件来显示有关的各种统计信息。
首先,可以使用 hpc 报告标志获得运行期间测试代码的摘要。我们将排除测试程序本身(使用\acode{--exclude}标志),以便只关注库的代码:

\begin{lstlisting}[language=Haskell]
  hpc report Run --exclude=Main --exclude=QC
  26% expressions used (54/202)
   0% boolean coverage (0/3)
        0% guards (0/3), 3 unevaluated
      100% 'if' conditions (0/0)
      100% qualifiers (0/0)
  21% alternatives used (8/37)
   0% local declarations used (0/4)
  44% top-level declarations used (16/36)
\end{lstlisting}

最后一行可以看到 42\% 的定义在测试运行期间被评估。第一次尝试还算不错。随着我们从库中测试越来越多的函数,这个数字将会上升。文本版本的
测试摘要对于快速总结很有用,弹药真正了解发生了什么,那么就可以使用\acode{markup}标记:

\begin{lstlisting}[language=Haskell]
  hpc markup Run --exclude=Main --exclude=QC
  Writing: Arbitrary.hs.html
  Writing: Prettify2.hs.html
  Writing: hpc_index.html
  Writing: hpc_index_fun.html
  Writing: hpc_index_alt.html
  Writing: hpc_index_exp.html
\end{lstlisting}

略

\end{document}
