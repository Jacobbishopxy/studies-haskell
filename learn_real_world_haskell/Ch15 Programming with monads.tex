\documentclass[./main.tex]{subfiles}

\begin{document}

\subsection*{练习:关联列表}

Web 客户端与服务端之间经常通过简单的键值对列表进行信息传输。

\begin{lstlisting}
  name=Attila+%42The+Hun%42&occupation=Khan
\end{lstlisting}

这里的编码名为\acode{application/x-www-form-urlencoded},同时非常便于理解。每个键值对都被一个“\&”符号分隔。在一个兼职对中,键为“=”符号之前的所有字符,而值为
之后的所有字符。

显然可以将一个\acode{String}作为键,但是 HTTP 并不清楚该键是否必须跟着一值。可以通过\acode{Maybe String}来表示一个模糊的值。如果值为\acode{Nothing},即无值
可展示。当使用\acode{Just}包裹一个值时则以为着有值。使用\acode{Maybe}让我们可以区分“无值”与“空值”。

Haskell 程序员使用类型为\acode{[(a, b)]}的\textit{关联列表},可以视作关联列表中的每个元素都是键与值的关联。

假设我们想用这些列表中的一个来填充一个数据结构。

\begin{lstlisting}[language=Haskell]
  data MovieReview = MovieReview
    { revTitle :: String,
      revUser :: String,
      revReview :: String
    }
\end{lstlisting}

从一个朴素的函数开始:

\begin{lstlisting}[language=Haskell]
  simpleReview :: [(String, Maybe String)] -> Maybe MovieReview
  simpleReview alist =
    case lookup "title" alist of
      Just (Just title@(_ : _)) ->
        case lookup "user" alist of
          Just (Just user@(_ : _)) ->
            case lookup "review" alist of
              Just (Just review@(_ : _)) ->
                Just (MovieReview title user review)
              _ -> Nothing -- no review
          _ -> Nothing -- no user
      _ -> Nothing -- no title
\end{lstlisting}

当关联列表包含了所有必要值且不为空值时,它将返回一个\acode{MovieReview}。

我们对\acode{Maybe}单子已经很熟悉了,因此可以简化一下上述的阶梯式代码:

\begin{lstlisting}[language=Haskell]
  maybeReview :: [(String, Maybe [Char])] -> Maybe MovieReview
  maybeReview alist = do
    title <- lookup1 "title" alist
    user <- lookup1 "user" alist
    review <- lookup1 "review" alist
    return $ MovieReview title user review

  lookup1 :: Eq a1 => a1 -> [(a1, Maybe [a2])] -> Maybe [a2]
  lookup1 key alist =
    case lookup key alist of
      Just (Just s@(_ : _)) -> Just s
      _ -> Nothing
\end{lstlisting}

尽管这看起来简洁多了,但仍然在重复自身。我们可以利用\acode{MoviewReview}构造函数作为一个普通的纯函数,通过\textit{lifting}它至单子:

\begin{lstlisting}[language=Haskell]
  liftedReview :: [(String, Maybe [Char])] -> Maybe MovieReview
  liftedReview alist =
    liftM3
      MovieReview
      (lookup1 "title" alist)
      (lookup1 "user" alist)
      (lookup1 "review" alist)
\end{lstlisting}

这里仍然有一些重复,不过更难简化。

\subsection*{泛化的 lifting}

尽管使用\acode{liftM3}得以简化代码,但是我们无法使用 liftM 家族的函数来解决更泛化的问题,因为标准库仅定义到了\acode{liftM5}。虽然可以自定义此类型的变体函数,
但是这个数量仍然是一个问题。

假设一个构造函数或者纯函数接收十个参数,并决定坚持使用标准库,这可能就不太合适了。

在\acode{Control.Monad}中,有一个名为\acode{ap}的函数拥有着有趣的类型签名。

\begin{lstlisting}[language=Haskell]
  ghci> :m +Control.Monad
  ghci> :type ap
  ap :: (Monad m) => m (a -> b) -> m a -> m b
\end{lstlisting}

我们可能会疑惑谁会将单参数纯函数放在单子中,且为什么。不过回想一下\textit{所有的} Haskell 函数实际上只接受一个参数,这里开始将看到其与\acode{MovieReview}构造函数
的管理。

\begin{lstlisting}[language=Haskell]
  ghci> :type MovieReview
  MovieReview :: String -> String -> String -> MovieReview
\end{lstlisting}

我们当然可以简单的将类型写作\acode{String -> (String -> (String -> MovieReview))}。如果使用旧的\acode{liftM}将\acode{MovieReview}提升至\acode{Maybe}
单子,那么我们将会得到类型为\acode{Maybe (String -> (String -> (String -> MovieReview)))}的值。现在可以看出来该类型适用于单个参数的\acode{ap}。我们可以
一次将这个传递给\acode{ap},并继续链式执行,直到得到该定义。

\begin{lstlisting}[language=Haskell]
  apReview :: [(String, Maybe [Char])] -> Maybe MovieReview
  apReview alist =
    MovieReview
      `liftM` lookup1 "title" alist
      `ap` lookup1 "user" alist
      `ap` lookup1 "review" alist
\end{lstlisting}

注:以下为步骤拆解后的类型变化。

\begin{lstlisting}[language=Haskell]
  MovieReview :: String -> ( String -> String -> MovieReview )
  MovieReview `liftM` :: Maybe String -> Maybe ( String -> String -> MovieReview )
  MovieReview `liftM` lookup1 "title" alist :: Maybe ( String -> String -> MovieReview )
  MovieReview `liftM` lookup1 "title" alist `ap` :: Maybe String -> Maybe ( String -> MovieReview )
  MovieReview `liftM` lookup1 "title" alist `ap` lookup1 "user" alist :: Maybe ( String -> MovieReview )
\end{lstlisting}

我们可以像这样把\acode{ap}的应用链接起来,只要有需要就可以多次链接,从而绕过\acode{liftM}系列函数。

看待\acode{ap}的另一种有用的方式是,它是我们熟悉的\acode{(\$)}操作符的一元等价物:可以把\acode{ap}读作\textit{apply}。当比较两者的函数签名时可知:

\begin{lstlisting}[language=Haskell]
  ghci> :type ($)
  ($) :: (a -> b) -> a -> b
  ghci> :type ap
  ap :: (Monad m) => m (a -> b) -> m a -> m b
\end{lstlisting}

实际上,\acode{ap}通常被定义为\acode{liftM2 id}或是\acode{liftM2 (\$)}。

\subsection*{寻找其它方案}

以下是某人的电话号码:

\begin{lstlisting}[language=Haskell]
  data Context = Home | Mobile | Business deriving (Eq, Show)

  type Phone = String

  albulena :: [(Context, String)]
  albulena = [(Home, "+355-652-55512")]

  nils :: [(Context, String)]
  nils =
    [ (Mobile, "+47-922-55-512"),
      (Business, "+47-922-12-121"),
      (Home, "+47-925-55-121"),
      (Business, "+47-922-25-551")
    ]

  twalumba :: [(Context, String)]
  twalumba = [(Business, "+260-02-55-5121")]
\end{lstlisting}

假设我们想要通过打电话来联系某人。我们不希望通过商务号码,更倾向于使用家庭号码(如果存在的话)而不是移动电话。

\begin{lstlisting}[language=Haskell]
  onePersonalPhone :: [(Context, Phone)] -> Maybe Phone
  onePersonalPhone ps =
    case lookup Home ps of
      Nothing -> lookup Mobile ps
      Just n -> Just n
\end{lstlisting}

当然我们可以使用\acode{Maybe}作为返回类型,我们无法考虑到某人可能拥有多个号码的可能性。因此我们需要一个列表:

\begin{lstlisting}[language=Haskell]
  allBusinessPhones :: [(Context, Phone)] -> [Phone]
  allBusinessPhones ps = map snd numbers
    where
      numbers =
        case filter (contextIs Business) ps of
          [] -> filter (contextIs Mobile) ps
          ns -> ns

  contextIs :: Eq a => a -> (a, b) -> Bool
  contextIs a (b, _) = a == b
\end{lstlisting}

注意这两个函数的\acode{case}表达式结构类似:一个替代方法处理第一次查找返回空值,而另一个方法处理非空情况。

\begin{lstlisting}[language=Haskell]
  ghci> onePersonalPhone twalumba
  Nothing
  ghci> onePersonalPhone albulena
  Just "+355-652-55512"
  ghci> allBusinessPhones nils
  ["+47-922-12-121","+47-922-25-551"]
\end{lstlisting}

Haskell 的\acode{Control.Monad}模块定义了一个 typeclass,\acode{MonadPlus},它让我们可以出\acode{case}表达式中抽象出公共模式。

\begin{lstlisting}[language=Haskell]
  ghci> import Control.Monad
  ghci> :i MonadPlus
  type MonadPlus :: (* -> *) -> Constraint
  class (GHC.Base.Alternative m, Monad m) => MonadPlus m where
    mzero :: m a
    mplus :: m a -> m a -> m a
          -- Defined in ‘GHC.Base’
  instance MonadPlus IO -- Defined in ‘GHC.Base’
  instance MonadPlus [] -- Defined in ‘GHC.Base’
  instance MonadPlus Maybe -- Defined in ‘GHC.Base’
\end{lstlisting}

\acode{mzero}代表一个空值,而\acode{mplus}则是将两个结果合并成一个。我们现在可以使用\acode{mplus}来完全的移除\acode{case}表达式了。

\begin{lstlisting}[language=Haskell]
  oneBusinessPhone :: [(Context, Phone)] -> Maybe Phone
  oneBusinessPhone ps = lookup Business ps `mplus` lookup Mobile ps

  allPersonalPhones :: [(Context, Phone)] -> [Phone]
  allPersonalPhones ps =
    map snd $
      filter (contextIs Home) ps
        `mplus` filter (contextIs Mobile) ps
\end{lstlisting}

这些函数中由于我们知道\acode{lookup}返回一个类型为\acode{Maybe}的值,以及\acode{filter}返回一个列表,那么\acode{mplus}使用的版本就很明显了。

更有趣的是,我们可以使用\acode{mzero}与\acode{mplus}来编写任何对\acode{MonadPlus}实例都有用的函数。以下是标准查找函数的例子:

\begin{lstlisting}[language=Haskell]
  lookup :: (Eq a) => a -> [(a, b)] -> Maybe b
  lookup _ []                      = Nothing
  lookup k ((x,y):xys) | x == k    = Just y
                       | otherwise = lookup k xys
\end{lstlisting}

我们可以轻易地泛化返回类型至任意\acode{MonadPlus}的实例:

\begin{lstlisting}[language=Haskell]
  lookupM :: (MonadPlus m, Eq a) => a -> [(a, b)] -> m b
  lookupM _ [] = mzero
  lookupM k ((x, y) : xys)
    | x == k = return y `mplus` lookupM k xys
    | otherwise = lookupM k xys
\end{lstlisting}

如果结果类型是\acode{Maybe},即要么没有结果,要么一个结果;如果类型是列表,即所有结果;或者其它更适合于\acode{MonadPlus}的奇异实例。

\subsubsection*{mplus 并不是加法}

\subsubsection*{MonadPlus 的规则}

\subsubsection*{失败安全的 MonadPlus}

\subsection*{隐藏管道的冒险}

\subsubsection*{支持随机数}

\subsubsection*{再一次尝试}

\subsection*{从实现中分离接口}

\subsubsection*{若干参数的 typeclasses}

\subsubsection*{函数式依赖}

\subsubsection*{完善我们的模块}

\subsubsection*{对单子接口进行编程}

\subsection*{reader 单子}

\subsection*{返回自动推导}

\subsection*{隐藏 IO 单子}

\subsubsection*{使用 newtype}

\subsubsection*{为意想不到的用途设计}

\subsubsection*{使用 typeclasses}

\subsubsection*{隔离与测试}

\subsubsection*{writer 单子与列表}

\subsubsection*{任意 I/O 的重新访问}

\end{document}
