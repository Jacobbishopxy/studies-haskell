\documentclass[./main.tex]{subfiles}

\begin{document}

\subsection*{练习:关联列表}

Web 客户端与服务端之间经常通过简单的键值对列表进行信息传输。

\begin{lstlisting}
  name=Attila+%42The+Hun%42&occupation=Khan
\end{lstlisting}

这里的编码名为\acode{application/x-www-form-urlencoded},同时非常便于理解。每个键值对都被一个“\&”符号分隔。在一个兼职对中,键为“=”符号之前的所有字符,而值为
之后的所有字符。

显然可以将一个\acode{String}作为键,但是 HTTP 并不清楚该键是否必须跟着一值。可以通过\acode{Maybe String}来表示一个模糊的值。如果值为\acode{Nothing},即无值
可展示。当使用\acode{Just}包裹一个值时则以为着有值。使用\acode{Maybe}让我们可以区分“无值”与“空值”。

Haskell 程序员使用类型为\acode{[(a, b)]}的\textit{关联列表},可以视作关联列表中的每个元素都是键与值的关联。

假设我们想用这些列表中的一个来填充一个数据结构。

\begin{lstlisting}[language=Haskell]
  data MovieReview = MovieReview
    { revTitle :: String,
      revUser :: String,
      revReview :: String
    }
\end{lstlisting}

从一个朴素的函数开始:

\begin{lstlisting}[language=Haskell]
  simpleReview :: [(String, Maybe String)] -> Maybe MovieReview
  simpleReview alist =
    case lookup "title" alist of
      Just (Just title@(_ : _)) ->
        case lookup "user" alist of
          Just (Just user@(_ : _)) ->
            case lookup "review" alist of
              Just (Just review@(_ : _)) ->
                Just (MovieReview title user review)
              _ -> Nothing -- no review
          _ -> Nothing -- no user
      _ -> Nothing -- no title
\end{lstlisting}

当关联列表包含了所有必要值且不为空值时,它将返回一个\acode{MovieReview}。

我们对\acode{Maybe}单子已经很熟悉了,因此可以简化一下上述的阶梯式代码:

\begin{lstlisting}[language=Haskell]
  maybeReview :: [(String, Maybe [Char])] -> Maybe MovieReview
  maybeReview alist = do
    title <- lookup1 "title" alist
    user <- lookup1 "user" alist
    review <- lookup1 "review" alist
    return $ MovieReview title user review

  lookup1 :: Eq a1 => a1 -> [(a1, Maybe [a2])] -> Maybe [a2]
  lookup1 key alist =
    case lookup key alist of
      Just (Just s@(_ : _)) -> Just s
      _ -> Nothing
\end{lstlisting}

尽管这看起来简洁多了,但仍然在重复自身。我们可以利用\acode{MoviewReview}构造函数作为一个普通的纯函数,通过\textit{lifting}它至单子:

\begin{lstlisting}[language=Haskell]
  liftedReview :: [(String, Maybe [Char])] -> Maybe MovieReview
  liftedReview alist =
    liftM3
      MovieReview
      (lookup1 "title" alist)
      (lookup1 "user" alist)
      (lookup1 "review" alist)
\end{lstlisting}

这里仍然有一些重复,不过更难简化。

\subsection*{泛化的 lifting}

% TODO

\subsection*{寻找其它方案}

\subsubsection*{mplus 并不是加法}

\subsubsection*{MonadPlus 的规则}

\subsubsection*{失败安全的 MonadPlus}

\subsection*{隐藏管道的冒险}

\subsubsection*{支持随机数}

\subsubsection*{再一次尝试}

\subsection*{从实现中分离接口}

\subsubsection*{若干参数的 typeclasses}

\subsubsection*{函数式依赖}

\subsubsection*{完善我们的模块}

\subsubsection*{对单子接口进行编程}

\subsection*{reader 单子}

\subsection*{返回自动推导}

\subsection*{隐藏 IO 单子}

\subsubsection*{使用 newtype}

\subsubsection*{为意想不到的用途设计}

\subsubsection*{使用 typeclasses}

\subsubsection*{隔离与测试}

\subsubsection*{writer 单子与列表}

\subsubsection*{任意 I/O 的重新访问}

\end{document}
