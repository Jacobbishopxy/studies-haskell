\documentclass[./main.tex]{subfiles}

\begin{document}

\subsection*{练习:关联列表}

Web 客户端与服务端之间经常通过简单的键值对列表进行信息传输。

\begin{lstlisting}
  name=Attila+%42The+Hun%42&occupation=Khan
\end{lstlisting}

这里的编码名为\acode{application/x-www-form-urlencoded},同时非常便于理解。每个键值对都被一个“\&”符号分隔。在一个兼职对中,键为“=”符号之前的所有字符,而值为
之后的所有字符。

显然可以将一个\acode{String}作为键,但是 HTTP 并不清楚该键是否必须跟着一值。可以通过\acode{Maybe String}来表示一个模糊的值。如果值为\acode{Nothing},即无值
可展示。当使用\acode{Just}包裹一个值时则以为着有值。使用\acode{Maybe}让我们可以区分“无值”与“空值”。

Haskell 程序员使用类型为\acode{[(a, b)]}的\textit{关联列表},可以视作关联列表中的每个元素都是键与值的关联。

假设我们想用这些列表中的一个来填充一个数据结构。

\begin{lstlisting}[language=Haskell]
  data MovieReview = MovieReview
    { revTitle :: String,
      revUser :: String,
      revReview :: String
    }
\end{lstlisting}

从一个朴素的函数开始:

\begin{lstlisting}[language=Haskell]
  simpleReview :: [(String, Maybe String)] -> Maybe MovieReview
  simpleReview alist =
    case lookup "title" alist of
      Just (Just title@(_ : _)) ->
        case lookup "user" alist of
          Just (Just user@(_ : _)) ->
            case lookup "review" alist of
              Just (Just review@(_ : _)) ->
                Just (MovieReview title user review)
              _ -> Nothing -- no review
          _ -> Nothing -- no user
      _ -> Nothing -- no title
\end{lstlisting}

当关联列表包含了所有必要值且不为空值时,它将返回一个\acode{MovieReview}。

我们对\acode{Maybe}单子已经很熟悉了,因此可以简化一下上述的阶梯式代码:

\begin{lstlisting}[language=Haskell]
  maybeReview :: [(String, Maybe [Char])] -> Maybe MovieReview
  maybeReview alist = do
    title <- lookup1 "title" alist
    user <- lookup1 "user" alist
    review <- lookup1 "review" alist
    return $ MovieReview title user review

  lookup1 :: Eq a1 => a1 -> [(a1, Maybe [a2])] -> Maybe [a2]
  lookup1 key alist =
    case lookup key alist of
      Just (Just s@(_ : _)) -> Just s
      _ -> Nothing
\end{lstlisting}

尽管这看起来简洁多了,但仍然在重复自身。我们可以利用\acode{MoviewReview}构造函数作为一个普通的纯函数,通过\textit{lifting}它至单子:

\begin{lstlisting}[language=Haskell]
  liftedReview :: [(String, Maybe [Char])] -> Maybe MovieReview
  liftedReview alist =
    liftM3
      MovieReview
      (lookup1 "title" alist)
      (lookup1 "user" alist)
      (lookup1 "review" alist)
\end{lstlisting}

这里仍然有一些重复,不过更难简化。

\subsection*{泛化的 lifting}

尽管使用\acode{liftM3}得以简化代码,但是我们无法使用 liftM 家族的函数来解决更泛化的问题,因为标准库仅定义到了\acode{liftM5}。虽然可以自定义此类型的变体函数,
但是这个数量仍然是一个问题。

假设一个构造函数或者纯函数接收十个参数,并决定坚持使用标准库,这可能就不太合适了。

在\acode{Control.Monad}中,有一个名为\acode{ap}的函数拥有着有趣的类型签名。

\begin{lstlisting}[language=Haskell]
  ghci> :m +Control.Monad
  ghci> :type ap
  ap :: (Monad m) => m (a -> b) -> m a -> m b
\end{lstlisting}

我们可能会疑惑谁会将单参数纯函数放在单子中,且为什么。不过回想一下\textit{所有的} Haskell 函数实际上只接受一个参数,这里开始将看到其与\acode{MovieReview}构造函数
的管理。

\begin{lstlisting}[language=Haskell]
  ghci> :type MovieReview
  MovieReview :: String -> String -> String -> MovieReview
\end{lstlisting}

我们当然可以简单的将类型写作\acode{String -> (String -> (String -> MovieReview))}。如果使用旧的\acode{liftM}将\acode{MovieReview}提升至\acode{Maybe}
单子,那么我们将会得到类型为\acode{Maybe (String -> (String -> (String -> MovieReview)))}的值。现在可以看出来该类型适用于单个参数的\acode{ap}。我们可以
一次将这个传递给\acode{ap},并继续链式执行,直到得到该定义。

\begin{lstlisting}[language=Haskell]
  apReview :: [(String, Maybe [Char])] -> Maybe MovieReview
  apReview alist =
    MovieReview
      `liftM` lookup1 "title" alist
      `ap` lookup1 "user" alist
      `ap` lookup1 "review" alist
\end{lstlisting}

注:以下为步骤拆解后的类型变化。

\begin{lstlisting}[language=Haskell]
  MovieReview :: String -> ( String -> String -> MovieReview )
  MovieReview `liftM` :: Maybe String -> Maybe ( String -> String -> MovieReview )
  MovieReview `liftM` lookup1 "title" alist :: Maybe ( String -> String -> MovieReview )
  MovieReview `liftM` lookup1 "title" alist `ap` :: Maybe String -> Maybe ( String -> MovieReview )
  MovieReview `liftM` lookup1 "title" alist `ap` lookup1 "user" alist :: Maybe ( String -> MovieReview )
\end{lstlisting}

我们可以像这样把\acode{ap}的应用链接起来,只要有需要就可以多次链接,从而绕过\acode{liftM}系列函数。

看待\acode{ap}的另一种有用的方式是,它是我们熟悉的\acode{(\$)}操作符的一元等价物:可以把\acode{ap}读作\textit{apply}。当比较两者的函数签名时可知:

\begin{lstlisting}[language=Haskell]
  ghci> :type ($)
  ($) :: (a -> b) -> a -> b
  ghci> :type ap
  ap :: (Monad m) => m (a -> b) -> m a -> m b
\end{lstlisting}

实际上,\acode{ap}通常被定义为\acode{liftM2 id}或是\acode{liftM2 (\$)}。

\subsection*{寻找其它方案}

以下是某人的电话号码:

\begin{lstlisting}[language=Haskell]
  data Context = Home | Mobile | Business deriving (Eq, Show)

  type Phone = String

  albulena :: [(Context, String)]
  albulena = [(Home, "+355-652-55512")]

  nils :: [(Context, String)]
  nils =
    [ (Mobile, "+47-922-55-512"),
      (Business, "+47-922-12-121"),
      (Home, "+47-925-55-121"),
      (Business, "+47-922-25-551")
    ]

  twalumba :: [(Context, String)]
  twalumba = [(Business, "+260-02-55-5121")]
\end{lstlisting}

假设我们想要通过打电话来联系某人。我们不希望通过商务号码,更倾向于使用家庭号码(如果存在的话)而不是移动电话。

\begin{lstlisting}[language=Haskell]
  onePersonalPhone :: [(Context, Phone)] -> Maybe Phone
  onePersonalPhone ps =
    case lookup Home ps of
      Nothing -> lookup Mobile ps
      Just n -> Just n
\end{lstlisting}

当然我们可以使用\acode{Maybe}作为返回类型,我们无法考虑到某人可能拥有多个号码的可能性。因此我们需要一个列表:

\begin{lstlisting}[language=Haskell]
  allBusinessPhones :: [(Context, Phone)] -> [Phone]
  allBusinessPhones ps = map snd numbers
    where
      numbers =
        case filter (contextIs Business) ps of
          [] -> filter (contextIs Mobile) ps
          ns -> ns

  contextIs :: Eq a => a -> (a, b) -> Bool
  contextIs a (b, _) = a == b
\end{lstlisting}

注意这两个函数的\acode{case}表达式结构类似:一个替代方法处理第一次查找返回空值,而另一个方法处理非空情况。

\begin{lstlisting}[language=Haskell]
  ghci> onePersonalPhone twalumba
  Nothing
  ghci> onePersonalPhone albulena
  Just "+355-652-55512"
  ghci> allBusinessPhones nils
  ["+47-922-12-121","+47-922-25-551"]
\end{lstlisting}

Haskell 的\acode{Control.Monad}模块定义了一个 typeclass,\acode{MonadPlus},它让我们可以出\acode{case}表达式中抽象出公共模式。

\begin{lstlisting}[language=Haskell]
  ghci> import Control.Monad
  ghci> :i MonadPlus
  type MonadPlus :: (* -> *) -> Constraint
  class (GHC.Base.Alternative m, Monad m) => MonadPlus m where
    mzero :: m a
    mplus :: m a -> m a -> m a
          -- Defined in ‘GHC.Base’
  instance MonadPlus IO -- Defined in ‘GHC.Base’
  instance MonadPlus [] -- Defined in ‘GHC.Base’
  instance MonadPlus Maybe -- Defined in ‘GHC.Base’
\end{lstlisting}

\acode{mzero}代表一个空值,而\acode{mplus}则是将两个结果合并成一个。我们现在可以使用\acode{mplus}来完全的移除\acode{case}表达式了。

\begin{lstlisting}[language=Haskell]
  oneBusinessPhone :: [(Context, Phone)] -> Maybe Phone
  oneBusinessPhone ps = lookup Business ps `mplus` lookup Mobile ps

  allPersonalPhones :: [(Context, Phone)] -> [Phone]
  allPersonalPhones ps =
    map snd $
      filter (contextIs Home) ps
        `mplus` filter (contextIs Mobile) ps
\end{lstlisting}

这些函数中由于我们知道\acode{lookup}返回一个类型为\acode{Maybe}的值,以及\acode{filter}返回一个列表,那么\acode{mplus}使用的版本就很明显了。

更有趣的是,我们可以使用\acode{mzero}与\acode{mplus}来编写任何对\acode{MonadPlus}实例都有用的函数。以下是标准查找函数的例子:

\begin{lstlisting}[language=Haskell]
  lookup :: (Eq a) => a -> [(a, b)] -> Maybe b
  lookup _ []                      = Nothing
  lookup k ((x,y):xys) | x == k    = Just y
                       | otherwise = lookup k xys
\end{lstlisting}

我们可以轻易地泛化返回类型至任意\acode{MonadPlus}的实例:

\begin{lstlisting}[language=Haskell]
  lookupM :: (MonadPlus m, Eq a) => a -> [(a, b)] -> m b
  lookupM _ [] = mzero
  lookupM k ((x, y) : xys)
    | x == k = return y `mplus` lookupM k xys
    | otherwise = lookupM k xys
\end{lstlisting}

如果结果类型是\acode{Maybe},即要么没有结果,要么一个结果;如果类型是列表,即所有结果;或者其它更适合于\acode{MonadPlus}的奇异实例。

\subsubsection*{mplus 并不是加法}

尽管\acode{mplus}函数包含了“plus”,但这并不意味着将两个值求和。

根据单子定义\acode{mplus}\textit{可能}会实现类似于加法的操作。例如立标单子中的\acode{mplus}是作为\acode{(++)}操作符实现的。

\begin{lstlisting}[language=Haskell]
  ghci> [1,2,3] `mplus` [4,5,6]
  [1,2,3,4,5,6]
\end{lstlisting}

然而切换到其他单子,类似加法的行为并不成立:

\begin{lstlisting}[language=Haskell]
  ghci> Just 1 `mplus` Just 2
  Just 1
\end{lstlisting}

\subsubsection*{MonadPlus 的规则}

\acode{MonadPlus} typeclass 的实例相较于普通的单子规则还需要遵循某些其他简单的规则。

如果\acode{mzero}出现在绑定表达式的左侧,则实例必须短路。换言之,表达式\acode{mzeor >>= f}的计算结果必须与\acode{mzero}单独计算的结果相同。

\begin{lstlisting}[language=Haskell]
  mzero >>= f == mzero
\end{lstlisting}

如果\acode{mzero}出现在序列表达式的\textit{右侧},那么该实例必须短路。

\begin{lstlisting}[language=Haskell]
  v >> mzero == mzero
\end{lstlisting}

\subsubsection*{失败安全的 MonadPlus}

早在“The Monad typeclass”章节中提到的\acode{fail}函数,被告知不要使用它:在很多\acode{Monad}中,它被实现为对\acode{error}的调用,这会产生令人不快的后果。

\acode{MonadPlus} typeclass 为我们提供了一种更温和的方式来失败计算,而不会出现\acode{fail}或\acode{error}。上面介绍的规则允许我们在任何需要的地方在代码中
引入\acode{mzero},并在该点上短路。

在\acode{Control.Monad}模块中,标准函数\acode{guard}将此理念打包成了方便的样式。

\begin{lstlisting}[language=Haskell]
  guard        :: (MonadPlus m) => Bool -> m ()
  guard True   =  return ()
  guard False  =  mzero
\end{lstlisting}

下面是个简单的例子,一个函数接受一个值\acode{x}并计算它对另一个数字\acode{n}取模。如果结果为零返回\acode{x},否则返回当前单子的\acode{mzero}。

\begin{lstlisting}[language=Haskell]
  x `zeroMod` n = guard ((x `mod` n) == 0) >> return x
\end{lstlisting}

\subsection*{隐藏管道的冒险}

略。

我们给单子取名为\acode{Supply},将执行函数\acode{runSupply}提供一个列表;需要确保列表中每一个元素都是唯一的。

\begin{lstlisting}[language=Haskell]
  runSupply :: Supply s a -> [s] -> (a, [s])
\end{lstlisting}

单子并不会关心内部的值:它们有可能是随机数,临时文件的名称,或是 HTTP cookies 的 ID。

单子内部,消费者每次需求一个值,\acode{next}则会从列表中获取下一个元素并给到消费者。每个值都会被\acode{Maybe}构造函数包装,以防列表长度不够。

\begin{lstlisting}[language=Haskell]
  next :: Supply s (Maybe s)
\end{lstlisting}

为了隐藏管道,模块声明时仅导出类型构造函数,执行函数,以及\acode{next}操作函数:

\begin{lstlisting}[language=Haskell]
  module Supply (Supply, next, runSupply) where
\end{lstlisting}

管道非常的简单:使用\acode{newtype}声明来包装一个\acode{State}单子:

\begin{lstlisting}[language=Haskell]
  import Control.Monad.State

  newtype Supply s a = S (State [s] a)
\end{lstlisting}

这里型参\acode{s}是我们将要提供的唯一值类型,而\acode{a}是为了类型成为单子而必须提供的通常类型参数。

我们对\acode{Supply}类型的\acode{newtype}的使用和模块头文件联合起来防止用户使用\acode{State}单子的\acode{get}和\acode{set}操作。由于模块没有导出\acode{S}
的构造函数,所以外部无法通过编程的方式看到包装的\acode{State}单子,也无法访问它。

此刻有了类型\acode{Supply},我们需要它来创建\acode{Monad} typeclass 实例。我们可以遵循\acode{(>>=)}以及\acode{return}的通常模式,但这只是存粹的样板代码。
我们所要做的是包装盒解包\acode{State}单子的\acode{(>>=)}版本,并使用\acode{S}值构造函数返回。

\begin{lstlisting}[language=Haskell]
  unwrapS :: Supply s a -> State [s] a
  unwrapS (S s) = s

  instance Functor (Supply s) where
    fmap f s = S $ fmap f $ unwrapS s

  instance Applicative (Supply s) where
    pure = S . return
    f <*> a = S $ unwrapS f <*> unwrapS a

  instance Monad (Supply s) where
    s >>= m = S $ unwrapS s >>= unwrapS . m
\end{lstlisting}

注:与原文不同,实现单子实例前还需分别实现函子实例与应用函子实例。

Haskell 程序员不喜欢样板文件,且可以肯定的是,GHC 有一个可爱的语言扩展用于消除这些样板。使用它需要再源文件顶部添加:

\begin{lstlisting}[language=Haskell]
  {-# LANGUAGE GeneralisedNewtypeDeriving #-}
\end{lstlisting}

通常而言,我们只能自动派生一些标准 typeclass 的实例,例如\acode{Show}和\acode{Eq}。顾名思义,\acode{GeneralisedNewtypeDeriving}扩展了派生 typeclass
实例的能力,且它是特定于\acode{newtype}声明的。如果我们包装的类型是任何 typeclass 的实例,扩展可以自动将我们的新类型作为该 typeclass 的实例,如下所示:

\begin{lstlisting}[language=Haskell]
  newtype Supply s a = S (State [s] a)
    deriving (Monad)
\end{lstlisting}

那么接下来就是\acode{next}与\acode{runSupply}的定义了:

\begin{lstlisting}[language=Haskell]
  runSupply :: Supply s a -> [s] -> (a, [s])
  runSupply (S m) xs = runState m xs

  next :: Supply s (Maybe s)
  next = S $ do
    st <- get
    case st of
      [] -> return Nothing
      (x : xs) -> do
        put xs
        return $ Just x
\end{lstlisting}

加载模块至\textbf{ghci}测试一下:

\begin{lstlisting}[language=Haskell]
  ghci> :l Supply
  [1 of 1] Compiling Supply           ( Supply.hs, interpreted )
  Ok, one module loaded.
  ghci> runSupply next [1,2,3]
  (Just 1,[2,3])
  ghci> import Control.Monad
  ghci> runSupply (liftM2 (,) next next) [1,2,3]
  ((Just 1,Just 2),[3])
  ghci> runSupply (liftM2 (,) next next) [1]
  ((Just 1,Nothing),[])
\end{lstlisting}

我们还可以验证\acode{State}单子是否泄漏了。

\begin{lstlisting}[language=Haskell]
  ghci> :browse Supply
  type role Supply nominal nominal
  type Supply :: * -> * -> *
  newtype Supply s a = S (State [s] a)
  runSupply :: Supply s a -> [s] -> (a, [s])
  next :: Supply s (Maybe s)
  ghci> :info Supply
  type role Supply nominal nominal
  type Supply :: * -> * -> *
  newtype Supply s a = S (State [s] a)
          -- Defined at Supply.hs:12:1
  instance Applicative (Supply s) -- Defined at Supply.hs:32:10
  instance Functor (Supply s) -- Defined at Supply.hs:29:10
  instance Monad (Supply s) -- Defined at Supply.hs:36:10
\end{lstlisting}

\subsubsection*{支持随机数}

如果想要使用\acode{Supply}单子作为随机数的源,那么我们将会遇到一个小困难。理想情况下,我们希望能够为它提供无限流式的随机数。我们可以在\acode{IO}单子中获得一个
\acode{StdGen},但是当完成时必须“放回”一个不同的\acode{StdGen}。如果不这么做,那么下一次获取\acode{StdGen}的代码将获得相同的状态,即产生一样的随机数,这可是
灾难性的事故。

从\acode{System.Random}模块中可知,目前的需求很难被调和。我们可以使用\acode{getStdRandom},其类型可以确保获得一个\acode{StdGen}时,又会放回一个。

\begin{lstlisting}[language=Haskell]
  ghci> :type getStdRandom
  getStdRandom :: (StdGen -> (a, StdGen)) -> IO a
\end{lstlisting}

在给到一个随机值后,可以使用\acode{random}来获取一个新的\acode{StdGen};可以使用\acode{randoms}来获取一个随机数的无限列表。但是我们该怎么得到一个无限的随机数
列表和一个新的\acode{StdGen}呢?

答案就在\acode{RandomGen} typeclass 的\acode{split}函数内,该函数接受一个随机数生成器,并将其转换为两个生成器。像这样拆分随机生成器是最不寻常的事:它在纯函数
设置中显然非常有用,但本质上既不是必须的,也不是由非纯语言提供的。

使用\acode{split}函数时,可以使用\acode{StdGen}来生成一个无限随机数列表用于\acode{runSupply},另一个则是用于\acode{IO}单子:

\begin{lstlisting}[language=Haskell]
  import Supply
  import System.Random hiding (next)

  randomsIO :: (Random a) => IO [a]
  randomsIO = getStdRandom $ \g ->
    let (a, b) = split g in (randoms a, b)
\end{lstlisting}

如果我们正确的编写了这个函数,那么示例应该在每次调用时打印一个不同的随机数:

\begin{lstlisting}[language=Haskell]
  ghci> :l RandomSupply.hs
  [1 of 2] Compiling Supply           ( Supply.hs, interpreted )
  [2 of 2] Compiling RandomSupply     ( RandomSupply.hs, interpreted )
  Ok, two modules loaded.
  ghci> (fst . runSupply next) `fmap` randomsIO
  Just (-8154423328023582499)
  ghci> (fst . runSupply next) `fmap` randomsIO
  Just (-4209314352233312889)
\end{lstlisting}

回忆一下,\acode{runSupply}函数即返回执行一元操作的结果,也返回列表中未使用的剩余部分。由于我们向它传递了一个随机数的无限列表,因此使用\acode{fst}以确保在
\textbf{ghci}尝试打印结果时不会被随机数淹没。

\subsubsection*{再一次尝试}

将函数应用与一对元组中的一个元素,并在不改变另一个原始元素的情况下构造一个新元组的模式,这在 Haskell 代码中很常见,以至于它已经变成了标准代码。

\acode{Control.Arrow}模块中有两个函数,\acode{first}与\acode{second},即实现了该操作。

\begin{lstlisting}[language=Haskell]
  ghci> :m +Control.Arrow
  ghci> first (+3) (1,2)
  (4,2)
  ghci> second odd ('a',1)
  ('a',True)
\end{lstlisting}

\subsection*{从实现中分离接口}

之前的小节中,我们见识到了在使用\acode{State}用于维护\acode{Supply}的状态时,是如何隐藏实现的。

另一个让代码更模块化的方式则是分离其\textit{接口} -- 即代码可以做的,与\textit{实现} -- 即代码如何做的。

\acode{System.Random}模块中的标准随机数生成器是很低效的。如果使用\acode{randomsIO}函数来提供随机数,那么\acode{next}操作的性能不会很好。

一个简单高效的处理方式就是为\acode{Supply}提供一个更好的随机数据源。现在让我们将这个想法放在一旁,而去考虑另一种方式,一种在很多设置中有用的方式。我们将单子
可以执行的操作与它使用的 typeclass 的工作方式分离开来。

\begin{lstlisting}[language=Haskell]
  class (Monad m) => MonadSupply s m | m -> s where
    next :: m (Maybe s)
\end{lstlisting}

该 typeclass 定义了任何 supply 单子必须实现的接口。他需要仔细检查,因为它使用了几个暂不熟悉的 Haskell 语言扩展。我们将在接下来的章节中逐一介绍。

\subsubsection*{若干参数的 typeclasses}

我们该如何阅读代码切片\acode{MonadSupply s m}这个 typeclass 呢?如果添加圆括号,那么一个相同的表达式就是\acode{(MonadSupply s) m},这更清晰一点。换言之,
给定某身为\acode{Monad}的类型变量\acode{m},我们可以使其成为\acode{MonadSupply s}的实例。有别于通常的 typeclass,它有一个\textit{参数}。

语言扩展允许一个 typeclass 拥有多个参数,其名称为\acode{MultiParamTypeClasses}。参数\acode{s}的作用与同名的\acode{Supply}类型参数相同:它表示下一个函数
传递的值的类型。

注意,我们不需要在\acode{MonadSupply s}的定义中提到\acode{(>>=)}或\acode{return},因为 typeclass 的上下文(superclass)要求\acode{MonadSupply s}
必须是\acode{Monad}。

\subsubsection*{函数式依赖}

% TODO

\subsubsection*{完善我们的模块}

\subsubsection*{对单子接口进行编程}

\subsection*{reader 单子}

\subsection*{返回自动推导}

\subsection*{隐藏 IO 单子}

\subsubsection*{使用 newtype}

\subsubsection*{为意想不到的用途设计}

\subsubsection*{使用 typeclasses}

\subsubsection*{隔离与测试}

\subsubsection*{writer 单子与列表}

\subsubsection*{任意 I/O 的重新访问}

\end{document}
