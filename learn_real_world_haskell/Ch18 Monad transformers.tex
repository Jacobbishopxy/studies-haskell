\documentclass[./main.tex]{subfiles}

\begin{document}

\subsection*{动机:避免样板}

单子提供了一种强大的方式来构建具有效果的计算。每个标准单子都专门做一件事。在实际代码中,我们经常需要能够一次使用几个效果。

回忆一下第十章开发的\acode{Parse}类型。在介绍单子的时候,我们提到了这个类型是一个伪装的状态单子。我们的单子比标准的\acode{State}单子更加的复杂,因为它使用了
\acode{Either}类型来容忍解析错误。这个例子中,如果一个解析提前出现了,我们是希望停止解析而不是继续带着破损的状态。我们的单子结合了携带状态的效果和提前退出的效果。

正常的\acode{State}单子不允许我们以这种方式逃离;它仅携带状态。它使用\acode{fail}的默认实现:即调用\acode{error},它跑出一个在纯代码中无法捕获的异常。因此,
\acode{State}单子允许失败,而这种能力实际上没有任何用处。(再次声明,我们建议尽量避免使用\acode{fail}!)

如果能够以某种方式采用标准的\acode{State}单子,并在其中添加错误处理,而不需要大量手工构建自定义单子,那将是最为理想的。\acode{mtl}库中的标准单子不允许我们组合
它们。相反,该库提供了一组\textit{单子转换}来实现相同的结果。

一个单子转换类似于普通的单子,但它不是一个独立的实体:相反,它修改底层单子的行为。\acode{mtl}库中的大多数单子都有等效的转换。根据惯例,单子转换的版本具有相同的名称,
末尾有一个\acode{T}。例如,与\acode{State}等价的转换就是\acode{StateT};它将可变状态添加到底层单子中。\acode{WriterT}单子转换使得在堆叠在另一个单子上时
写入数据成为可能。

\subsection*{简单单子转换案例}

在介绍单子转换之前,看一下用已经熟悉的技术编写的函数。下面的函数递归到一个目录树,并返回它在树的每级找到的条目数列表。

\begin{lstlisting}[language=Haskell]

  module CountEntries
    ( listDirectory,
      countEntriesTrad,
    )
  where

  import Control.Monad (forM, liftM)
  import System.Directory (doesDirectoryExist, getDirectoryContents)
  import System.FilePath ((</>))

  listDirectory :: FilePath -> IO [FilePath]
  listDirectory = liftM (filter notDots) . getDirectoryContents
    where
      notDots p = p /= "." && p /= ".."

  countEntriesTrad :: FilePath -> IO [(FilePath, Int)]
  countEntriesTrad path = do
    contents <- listDirectory path
    rest <- forM contents $ \name -> do
      let newName = path </> name
      isDir <- doesDirectoryExist newName
      if isDir
        then countEntriesTrad newName
        else return []
    return $ (path, length contents) : concat rest
\end{lstlisting}

现在我们来看看如何使用 writer 单子来实现同样的目的。由于这个单子允许我们记录任何需要的值,因此不需要显式的构建结果。

因为我们的函数必须在\acode{IO}单子中执行,这样它才能遍历目录,所以不能直接使用\acode{Writer}单子。相反,我们使用\acode{WriterT}向\acode{IO}添加记录功能。

普通的\acode{Writer}单子有两个类型参数,因此这里更为合适的写法是\acode{Writer w a}。第一个参数\acode{w}是要记录的值的类型,而\acode{a}则是\acode{Monad}
typeclass 通常所需要的类型。因此\acode{Writer [(FilePath, Int)] a}是一个记录目录名称和大小列表的 writer 单子。

\acode{WriterT}转换拥有类似的结构,不过它添加了另一个类型参数\acode{m}:这正是增强其行为的底层单子。\acode{WriterT}的完整签名为\acode{WriterT w m a}。

因为要遍历目录,这需要访问\acode{IO}单子,所以将 Writer 堆栈在\acode{IO}单子的顶部。单子转换和底层单子的组合将具有\acode{WriterT [(FilePath, Int)] IO a}
类型。这个单子转换和单子的堆栈本身就是一个单子。

\begin{lstlisting}[language=Haskell]
  module CountEntriesT
    ( countEntries,
    )
  where

  import Control.Monad
  import Control.Monad.Trans
  import Control.Monad.Writer
  import CountEntries (listDirectory)
  import System.Directory (doesDirectoryExist)
  import System.FilePath ((</>))

  countEntries :: FilePath -> WriterT [(FilePath, Int)] IO ()
  countEntries path = do
    contents <- liftIO . listDirectory $ path
    tell [(path, length contents)]
    forM_ contents $ \name -> do
      let newName = path </> name
      isDir <- liftIO . doesDirectoryExist $ newName
      when isDir $ countEntries newName
\end{lstlisting}

这段代码与之前的版本没有太大的不同。我们使用\acode{liftIO}在必要的地方公开\acode{IO}单子,并使用\acode{tell}来记录对目录的访问。

要使代码能被运行还必须使用\acode{WriterT}的一个执行函数:

\begin{lstlisting}[language=Haskell]
  ghci> :t runWriterT
  runWriterT :: WriterT w m a -> m (a, w)
  ghci> :t execWriterT
  execWriterT :: Monad m => WriterT w m a -> m w
\end{lstlisting}

这些函数执行操作,移除\acode{WriteT}包装并给出一个封装在底层单子中的结果。\acode{runWriterT}函数既提供操作的结果也提供运行时记录的内容,而\acode{execWriteT}
会丢弃结果,值提供记录的内容。

\begin{lstlisting}[language=Haskell]
  ghci> :l CountEntriesT.hs
  [1 of 2] Compiling CountEntries     ( CountEntries.hs, interpreted )
  [2 of 2] Compiling CountEntriesT    ( CountEntriesT.hs, interpreted )
  Ok, two modules loaded.
  ghci> :t countEntries ".."
  countEntries ".." :: WriterT [(FilePath, Int)] IO ()
  ghci> :t execWriterT (countEntries "..")
  execWriterT (countEntries "..") :: IO [(FilePath, Int)]
  ghci> take 4 `liftM` execWriterT (countEntries "..")
  [("..",6),("../dist-newstyle",2),("../dist-newstyle/cache",8),("../dist-newstyle/tmp",0)]
\end{lstlisting}

我们在\acode{IO}上使用\acode{WriterT},因为没有\acode{IOT}单子转换。每当我们使用一个或多个单子转换的\acode{IO}单子时,\acode{IO}总是在堆栈的底部。

\subsection*{单子和单子转换中的常见模式}

%

\begin{lstlisting}[language=Haskell]

\end{lstlisting}



\begin{lstlisting}[language=Haskell]

\end{lstlisting}



\begin{lstlisting}[language=Haskell]

\end{lstlisting}



\begin{lstlisting}[language=Haskell]

\end{lstlisting}



\begin{lstlisting}[language=Haskell]

\end{lstlisting}



\begin{lstlisting}[language=Haskell]

\end{lstlisting}



\begin{lstlisting}[language=Haskell]

\end{lstlisting}



\begin{lstlisting}[language=Haskell]

\end{lstlisting}



\begin{lstlisting}[language=Haskell]

\end{lstlisting}



\begin{lstlisting}[language=Haskell]

\end{lstlisting}



\begin{lstlisting}[language=Haskell]

\end{lstlisting}



\begin{lstlisting}[language=Haskell]

\end{lstlisting}



\begin{lstlisting}[language=Haskell]

\end{lstlisting}



\begin{lstlisting}[language=Haskell]

\end{lstlisting}



\begin{lstlisting}[language=Haskell]

\end{lstlisting}



\begin{lstlisting}[language=Haskell]

\end{lstlisting}



\begin{lstlisting}[language=Haskell]

\end{lstlisting}



\begin{lstlisting}[language=Haskell]

\end{lstlisting}



\begin{lstlisting}[language=Haskell]

\end{lstlisting}



\begin{lstlisting}[language=Haskell]

\end{lstlisting}



\begin{lstlisting}[language=Haskell]

\end{lstlisting}



\begin{lstlisting}[language=Haskell]

\end{lstlisting}



\begin{lstlisting}[language=Haskell]

\end{lstlisting}



\begin{lstlisting}[language=Haskell]

\end{lstlisting}



\begin{lstlisting}[language=Haskell]

\end{lstlisting}



\begin{lstlisting}[language=Haskell]

\end{lstlisting}



\begin{lstlisting}[language=Haskell]

\end{lstlisting}



\begin{lstlisting}[language=Haskell]

\end{lstlisting}



\begin{lstlisting}[language=Haskell]

\end{lstlisting}



\begin{lstlisting}[language=Haskell]

\end{lstlisting}



\begin{lstlisting}[language=Haskell]

\end{lstlisting}



\begin{lstlisting}[language=Haskell]

\end{lstlisting}



\begin{lstlisting}[language=Haskell]

\end{lstlisting}



\begin{lstlisting}[language=Haskell]

\end{lstlisting}



\begin{lstlisting}[language=Haskell]

\end{lstlisting}



\begin{lstlisting}[language=Haskell]

\end{lstlisting}



\begin{lstlisting}[language=Haskell]

\end{lstlisting}



\begin{lstlisting}[language=Haskell]

\end{lstlisting}

\subsection*{堆叠多个单子转换}

\subsubsection*{隐藏工序}

\subsection*{向下移动堆栈}

\subsubsection*{当需要显式 lifting 时}

\subsection*{通过构建一个单子转换来了解它}

\subsubsection*{创建一个单子转换}
\subsubsection*{更多的 typeclass 实例}

\subsubsection*{将 Parse 类型替换为单子堆栈}

\subsection*{转换堆栈的顺序很重要}

\subsection*{透视单子和单子转换}

\subsubsection*{对于纯代码的干扰}

\subsubsection*{超出定义的顺序}

\subsubsection*{运行时开支}

\subsubsection*{笨拙的接口}

\subsubsection*{将全部合起来}

\end{document}
