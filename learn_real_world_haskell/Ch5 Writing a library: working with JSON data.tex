\documentclass[./main.tex]{subfiles}

\begin{document}

\subsection*{在 Haskell 中表示 JSON}

首先是在 Haskell 中定义 JSON 这个数据,这里使用代数数据类型来表达 JSON 类型的范围。

\begin{lstlisting}[language=Haskell]
  data JValue
    = JString String
    | JNumber Double
    | JBool Bool
    | JNull
    | JObject [(String, JValue)]
    | JArray [JValue]
    deriving (Eq, Ord, Show)
\end{lstlisting}

对于每种 JSON 类型,我们都提供了独立的值构造函数。测试:

\begin{lstlisting}[language=Haskell]
  ghci> :l SimpleJSON
  [1 of 1] Compiling Main             ( SimpleJSON.hs, interpreted )
  Ok, one module loaded.
  ghci> JString "foo"
  JString "foo"
  ghci> JNumber 2.7
  JNumber 2.7
  ghci> :type JBool True
  JBool True :: JValue
\end{lstlisting}

构造一个从\acode{JValue}获取字符串的函数:

\begin{lstlisting}[language=Haskell]
  getString :: JValue -> Maybe String
  getString (JString s) = Just s
  getString _ = Nothing
\end{lstlisting}

测试:

\begin{lstlisting}[language=Haskell]
  ghci> :reload
  [1 of 1] Compiling Main             ( SimpleJSON.hs, interpreted )
  Ok, one module loaded.
  ghci> getString (JString "hello")
  Just "hello"
  ghci> getString (JNumber 3)
  Nothing
\end{lstlisting}

接下来是其它类型的访问函数:

\begin{lstlisting}[language=Haskell]
  getInt (JNumber n) = Just n
  getInt _ = Nothing

  getDouble (JNumber n) = Just n
  getDouble _ = Nothing

  getBool (JBool n) = Just n
  getBool _ = Nothing

  getObject (JObject o) = Just o
  getObject _ = Nothing

  getArray (JArray a) = Just a
  getArray _ = Nothing

  isNull v = v == JNull
\end{lstlisting}

\acode{truncate}函数可以让浮点类型或者有理数去掉小数点后变为整数:

\begin{lstlisting}[language=Haskell]
  ghci> truncate 5.8
  5
  ghci> :module +Data.Ratio
  ghci> truncate (22 % 7)
  3
\end{lstlisting}

\subsection*{Haskell 模块详解}

一个 Haskell 源文件包含了单个\textit{模块}的定义。模块允许我们在它其内部进行定义,并由其它模块访问:

\begin{lstlisting}[language=Haskell]
  module SimpleJSON
    ( JValue (..),
      getString,
      getInt,
      getDouble,
      getBool,
      getObject,
      getArray,
      isNull,
    )
  where
\end{lstlisting}

如果省略了导出(即圆括号以及其所包含的名称),那么该模块中的所有名称都会被导出。

\subsection*{编译 Haskell 源}

编译一个源文件:

\begin{lstlisting}[language=Bash]
  ghc -c SimpleJSON.hs
\end{lstlisting}

\acode{-c}选项告诉\textbf{ghc}仅生成对象代码。如果省略了该选项,那么编译器则会尝试生成一整个可执行文件。这会导致失败,因为
我们并没有一个\acode{main}函数,即 GHC 所认为的一个独立程序的执行入口。

编译后会得到两个新文件:\acode{SimpleJSON.hi}与\acode{SimpleJSON.o}。前者是一个\textit{接口 interface}文件,即
\textbf{ghc}以机器码的形式存储模块导出的名称信息;后者是一个\textit{对象 object}文件,其包含了生产的机器码。

\subsection*{生成一个 Haskell 程序,导入模块}

添加一个\acode{Main.hs}文件,其内容如下:

\begin{lstlisting}[language=Haskell]
  module Main where

  import SimpleJSON

  main = print (JObject [("foo", JNumber 1), ("bar", JBool False)])
\end{lstlisting}

与原文\acode{module Main () where}的不同之处在于,现在的\acode{Main}后不再需要一个\acode{()}。

接下来是编译\acode{main}函数:

\begin{lstlisting}[language=Bash]
  ghc -o simple Main.hs
\end{lstlisting}

与原文\acode{ghc -o simple Main.hs SimpleJSON.o}不同,现在没了\acode{SimpleJSON.o}这个文件,加上后会报错重复
symbol 的编译错误(因为在\acode{Main.hs}中已经做了\acode{import SimpleJSON}导入了)。

这次省略掉了\acode{-c}选项,因此编译器尝试生成一个可执行。生成可执行的这个过程被称为\textit{链接 linking}(与 C++ 一样),
即在一次编译中链接源文件与可执行文件。

这里给了\textbf{ghc}一个新选项\acode{-o},其接受一个参数:可执行文件的名称,这里是\acode{simple},执行它:

\begin{lstlisting}[language=Bash]
  ./simple
  JObject [("foo",JNumber 1.0),("bar",JBool False)]
\end{lstlisting}

\subsection*{打印 JSON 数据}

现在我们希望将 Haskell 的值渲染成 JSON 数据,创建一个\acode{PutJSON.hs}文件:

\begin{lstlisting}[language=Haskell]
  module PutJSON where

  import Data.List (intercalate)
  import SimpleJSON

  renderJValue :: JValue -> String
  renderJValue (JString s) = show s
  renderJValue (JNumber n) = show n
  renderJValue (JBool True) = "true"
  renderJValue (JBool False) = "false"
  renderJValue JNull = "null"
  renderJValue (JObject o) = "{" ++ pairs o ++ "}"
    where
      pairs [] = ""
      pairs ps = intercalate ", " (map renderPair ps)
      renderPair (k, v) = show k ++ ": " ++ renderJValue v
  renderJValue (JArray a) = "[" ++ values a ++ "]"
    where
      values [] = ""
      values vs = intercalate ", " (map renderJValue vs)
\end{lstlisting}

好的 Haskell 风格需要分隔纯代码与 I/O 代码。我们的\acode{renderJValue}函数不会与外界交互,但是仍然需要一个打印的函数:

\begin{lstlisting}[language=Haskell]
  putJValue :: JValue -> IO ()
  putJValue = putStrLn . renderJValue
\end{lstlisting}

\subsection*{类型推导是把双刃剑}

假设我们编写了一个自认为返回\acode{String}的函数,但是并不为其写类型签名:

\begin{lstlisting}[language=Haskell]
  upcaseFirst (c:cs) = toUpper c -- forgot ":cs" here
\end{lstlisting}

这里希望的是单词首字母大写,但是忘记了将剩余的字符放进结果中。我们认为函数的类型是\acode{String -> String},但是编译器则会
将其视为\acode{String -> Char}。假设我们尝试在其他地方调用该函数:

\begin{lstlisting}[language=Haskell]
  camelCase :: String -> String
  camelCase xs = concat (map upcaseFirst (words xs))
\end{lstlisting}

那么当我们尝试编译该代码或者是加载进\textbf{ghci},我们并不会得到明显的错误信息:

\begin{lstlisting}[language=Haskell]
  ghci> :load Trouble
  [1 of 1] Compiling Main             ( Trouble.hs, interpreted )

  Trouble.hs:9:27:
      Couldn't match expected type `[Char]' against inferred type `Char'
        Expected type: [Char] -> [Char]
        Inferred type: [Char] -> Char
      In the first argument of `map', namely `upcaseFirst'
      In the first argument of `concat', namely
          `(map upcaseFirst (words xs))'
  Failed, modules loaded: none.
\end{lstlisting}

注意这里的报错是在使用\acode{upcaseFirst}函数处,那么假设我们认为\acode{upcaseFirst}的定义与类型是正确的,那么查找到真正的
错误可能会花掉一些时间。

\subsection*{更加泛用的渲染}

我们将更为泛用的打印模块称为\acode{Prettify},那么其源文件即\acode{Prettify.hs}。

为了使\acode{Prettify}满足实际需求,我们还要一个新的 JSON 渲染器来使用\acode{Prettify}的API。在\acode{Prettify}模块中将
使用一个抽象类型\acode{Doc}。基于建立在抽象类型的泛用渲染库,我们可以选择灵活高效的实现。

\acode{PrettyJSON.hs}示例:

\begin{lstlisting}[language=Haskell]
  renderJValue :: JValue -> Doc
  renderJValue (JBool True)  = text "true"
  renderJValue (JBool False) = text "false"
  renderJValue JNull         = text "null"
  renderJValue (JNumber num) = double num
  renderJValue (JString str) = string str
\end{lstlisting}

这里的\acode{text},\acode{double}以及\acode{string}都会在\acode{Prettify}模块中提供。

\subsection*{开发 Haskell 代码而不发疯}

一个用于快速开发程序框架的技巧就是编写占位符,或者类型与函数的\textit{根 stub}版本。例如上述\acode{string},\acode{text}
以及\acode{double}函数将被\acode{Prettify}模块提供。如果我们没有提供这些函数或者\acode{Doc}类型,那么“早点编译,经常编译”
这个尝试就会失败。为了避免这个问题现在让我们编写一个不做任何事情的根程序。

\begin{lstlisting}[language=Haskell]
  import SimpleJSON

  data Doc = ToBeDefined deriving (Show)

  string :: String -> Doc
  string str = undefined

  text :: String -> Doc
  text str = undefined

  double :: Double -> Doc
  double num = undefined
\end{lstlisting}

特殊值\acode{undefined}有一个\acode{a}类型,无论在哪里使用它,总是会有类型检查。如果尝试计算,则会使程序崩溃:

\begin{lstlisting}[language=Haskell]
  ghci> :type undefined
  undefined :: a
  ghci> undefined
  *** Exception: Prelude.undefined
  ghci> :type double
  double :: Double -> Doc
  ghci> double 3.14
  *** Exception: Prelude.undefined
\end{lstlisting}

尽管还不能运行根代码,但是编译器的类型检查器可以确保我们的程序类型正确。

\subsection*{漂亮的打印字符串}

当需要打印一个漂亮的字符串时,我们必须遵循 JSON 的转义规则。字符串就是一系列被包裹在引号中的字符们。\acode{PrettyJSON.hs}:

\begin{lstlisting}[language=Haskell]
  string :: String -> Doc
  string = enclose '"' '"' . hcat . map oneChar
\end{lstlisting}

以及\acode{enclose}函数将一个\acode{Doc}值简单的包裹在一个开始与结束字符之间:

\begin{lstlisting}[language=Haskell]
  enclose :: Char -> Char -> Doc -> Doc
  enclose left right x = char left <> x <> char right
\end{lstlisting}

这里提供的\acode{<>}函数定义在\acode{Prettify}库中,它需要两个\acode{Doc}值,即\acode{Doc}版本的\acode{(++)}。还是
先在根文件中定义:

\begin{lstlisting}[language=Haskell]
  (<>) :: Doc -> Doc -> Doc
  a <> b = undefined

  char :: Char -> Doc
  char c = undefined
\end{lstlisting}

我们的库还需要提供\acode{hcat},将若干\acode{Doc}值合成成为一个(类似于列表的\acode{concat}):

\begin{lstlisting}[language=Haskell]
  hcat :: [Doc] -> Doc
  hcat xs = undefined
\end{lstlisting}

我们的\acode{string}函数应用\acode{oneChar}函数到字符串中的每个字符,将它们连接,然后用引号包装。而\acode{oneChar}函数
则是用来转义或包装一个独立的字符。在\acode{PrettyJSON.hs}中:

\begin{lstlisting}[language=Haskell]
  oneChar :: Char -> Doc
  oneChar c = case lookup c simpleEscapes of
    Just r -> text r
    Nothing
      | mustEscape c -> hexEscape c
      | otherwise -> char c
    where
      mustEscape c = c < ' ' || c == '\x7f' || c > '\xff'

  simpleEscapes :: [(Char, String)]
  simpleEscapes = zipWith ch "\b\n\f\r\t\\\"/" "bnfrt\\\"/"
    where
      ch a b = (a, ['\\', b])
\end{lstlisting}

这里的\acode{simpleEscapes}是一个列表的二元组,我们称其为\textit{关联 association}列表,或者简称\acode{alist}。每个
\acode{alist}的元素都将字符关联了它的转义表达,测试:

\begin{lstlisting}[language=Haskell]
  ghci> take 4 simpleEscapes
  [('\b',"\\b"),('\n',"\\n"),('\f',"\\f"),('\r',"\\r")]
\end{lstlisting}

我们的\acode{case}表达式尝试查看字符是否匹配关联列表。如果匹配则返回匹配值,如若不然则需要以更复杂的方式来转义该字符。只有当
两种转义都不需要时,才会返回普通字符。保守起见,我们输出的唯一未转义字符是可打印的 ASCII 字符。

更复杂的转义包含了将一个字符转为字符串\acode{"\\u"}并跟随四个十六进制的字符用于表达 Unicode 字符的数值。仍然是
\acode{PrettyJSON.hs}:

\begin{lstlisting}[language=Haskell]
  smallHex :: Int -> Doc
  smallHex x =
    text "\\u"
      <> text (replicate (4 - length h) '0')
      <> text h
    where
      h = showHex x ""
\end{lstlisting}

这里的\acode{showHex}函数需要从\acode{Numeric}库中加载,其用于返回一个值的十六进制:

\begin{lstlisting}[language=Haskell]
  ghci> showHex 114111 ""
  "1bdbf"
\end{lstlisting}

\acode{replicate}函数则是由 Prelude 提供:

\begin{lstlisting}[language=Haskell]
  ghci> replicate 5 "foo"
  ["foo","foo","foo","foo","foo"]
\end{lstlisting}

\acode{smallHex}提供的四位编码只能表示最大\acode{0xffff}的 Unicode 字符,而有效的 Unicode 字符的范围可以达到
\acode{0x10ffff}。为了正确的表达一个超出了\acode{0xffff}的 JSON 字符串,我们遵循一些复杂的规则将其分为两部分。这使我们
有机会对 Haskell 的数执行一些位级操作。还是\acode{PrettyJSON.hs}:

\begin{lstlisting}[language=Haskell]
  astral :: Int -> Doc
  astral n = smallHex (a + 0xd800) <> smallHex (b + 0xdc00)
    where
      a = (n `shiftR` 10) .&. 0x3ff
      b = n .&. 0x3ff
\end{lstlisting}

这里\acode{shiftR}函数和\acode{(.&.)}函数都源自\acode{Data.Bits}模块,前者将一个数移动右一位,后者则是执行一个字节层面的
两值\textit{and}操作。

\begin{lstlisting}[language=Haskell]
  ghci> 0x10000 `shiftR` 4   :: Int
  4096
  ghci> 7 .&. 2   :: Int
  2
\end{lstlisting}

现在有了\acode{smallHex}与\acode{astral},我们可以提供\acode{hexEscape}的定义了:

\begin{lstlisting}[language=Haskell]
  hexEscape :: Char -> Doc
  hexEscape c
    | d < 0x10000 = smallHex d
    | otherwise = astral (d - 0x10000)
    where
      d = ord c
\end{lstlisting}

其中\acode{ord}由\acode{Data.Char}模块提供。

\subsection*{数组与对象,以及模块头}

% TODO

\end{document}
